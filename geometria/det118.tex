\documentclass{article}
\usepackage[utf8]{inputenc}
\usepackage[italian]{babel}
\usepackage{amsmath}
\usepackage{amsfonts}
\usepackage{amssymb}
\usepackage{xfrac}
\newcommand*{\qed}{\blacksquare}
\newcommand*{\M}[3]{\mathcal{M}_{#1\times#2} \left(#3\right)}
\newcommand*{\MR}[2]{\M{#1}{#2}{\mathbb{R}}}
\newcommand*{\MC}[2]{\M{#1}{#2}{\mathbb{C}}}
\newcommand*{\MK}[2]{\M{#1}{#2}{\mathbb{K}}}
\newcommand*{\T}[1]{{#1}^\text{T}}  % Trasposta di una matrice
\newcommand*{\mm}[1]{\begin{pmatrix}#1\end{pmatrix}}
\newcommand*{\m}[1]{\begin{bmatrix}#1\end{bmatrix}}
\DeclareMathOperator{\Tr}{Tr}  % Traccia di una matrice
\title{Geometria - Esercizio 4.24}
\author{Riccardo Bergamaschi}
\date{19/10/2023}
\begin{document}
\maketitle
\section*{Metodo 1: Gauss (rigoroso)}
\[\begin{aligned}
    A = &\m{
        1 & -2 &  3 & -2 & -2 \\
        2 & -1 &  1 &  3 &  2 \\
        1 &  1 &  2 &  1 &  1 \\
        1 & -4 & -3 & -2 & -5 \\
        3 & -2 &  2 &  2 & -2 \\
    }\\\mm{\\A_2-2A_1\\A_3-A_1\\A_4-A_1\\A_5-3A_1}\leadsto&\m{
        1 & -2 &  3 & -2 & -2 \\
        0 &  3 & -5 &  7 &  6 \\
        0 &  3 & -1 &  3 &  3 \\
        0 & -2 & -6 &  0 & -3 \\
        0 &  4 & -7 &  8 &  4 \\
    }\\\mm{\\\\A_3-A_2\\A_4+\sfrac{2}{3}A_2\\A_5-\sfrac{4}{3}A_2}\leadsto&\m{
        1 & -2 &  3 & -2 & -2 \\
        0 &  3 & -5 &  7 &  6 \\
        0 &  0 &  4 & -4 & -3 \\
        0 &  0 & -\sfrac{28}{3} & \sfrac{14}{3} & 1 \\
        0 &  0 & -\sfrac{1}{3}  & -\sfrac{4}{3} & -4 \\
    }\\\mm{\\\\\\A_4+\sfrac{7}{3}A_3\\A_5+\sfrac{1}{12}A_3}\leadsto&\m{
        1 & -2 &  3 & -2 & -2 \\
        0 &  3 & -5 &  7 &  6 \\
        0 &  0 &  4 & -4 & -3 \\
        0 &  0 &  0 & -\sfrac{14}{3} & -6 \\
        0 &  0 &  0 & -\sfrac{5}{3} & -\sfrac{17}{4} \\
    }\\\mm{\\\\\\\\A_5-\sfrac{5}{14}A_4}\leadsto&\m{
        1 & -2 &  3 & -2 & -2 \\
        0 &  3 & -5 &  7 &  6 \\
        0 &  0 &  4 & -4 & -3 \\
        0 &  0 &  0 & -\sfrac{14}{3} & -6 \\
        0 &  0 &  0 & 0 & \sfrac{-59}{28} \\
    }
\end{aligned}\]
\[\det{A}=\det\m{
    1 & -2 &  3 & -2 & -2 \\
    0 &  3 & -5 &  7 &  6 \\
    0 &  0 &  4 & -4 & -3 \\
    0 &  0 &  0 & -\sfrac{14}{3} & -6 \\
    0 &  0 &  0 & 0 & \sfrac{-59}{28} \\
}=1\cdot3\cdot4\cdot\left(-\frac{14}{3}\right)\cdot\left(-\frac{59}{28}\right)=118\]

\section*{Metodo 1: Operazioni elementari}
Come Gauss, ma più efficiente (forse?).

\[\begin{aligned}
    \det\m{
        1 & -2 &  3 & -2 & -2 \\
        2 & -1 &  1 &  3 &  2 \\
        1 &  1 &  2 &  1 &  1 \\
        1 & -4 & -3 & -2 & -5 \\
        3 & -2 &  2 &  2 & -2 \\
    }&=\det\m{
        1 & -2 &  3 & -2 & -2 \\
        0 &  3 & -5 &  7 &  6 \\
        0 &  3 & -1 &  3 &  3 \\
        0 & -2 & -6 &  0 & -3 \\
        0 &  4 & -7 &  8 &  4 \\
    }\\&=\det\m{
        3 & -5 &  7 &  6 \\
         3 & -1 &  3 &  3 \\
        -2 & -6 &  0 & -3 \\
         4 & -7 &  8 &  4 \\
    }\\&=\det\m{
         0 & -4 &  4 &  3 \\
         3 & -1 &  3 &  3 \\
        -2 & -6 &  0 & -3 \\
         0 & -19 & 8 & -2 \\
    }\\&=\det\m{
        0 & -4 &  4 &  3 \\
        1 & -7 &  3 &  6 \\
       -2 & -6 &  0 & -3 \\
        0 & -19 & 8 & -2 \\
    }\\&=\det\m{
        0 & -4 &  4 &  3 \\
        1 & -7 &  3 &  6 \\
        0 & -6-14 &  0+6 & -3+12 \\
        0 & -19 & 8 & -2 \\
    }
\end{aligned}\]

\end{document}
