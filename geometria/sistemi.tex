\documentclass{article}
\usepackage[utf8]{inputenc}
\usepackage[italian]{babel}
\usepackage{amsmath}
\usepackage{amsfonts}
\usepackage{amssymb}
\usepackage{xfrac}
\makeatletter
\renewcommand*\env@matrix[1][*\c@MaxMatrixCols c]{%
  \hskip -\arraycolsep
  \let\@ifnextchar\new@ifnextchar
  \array{#1}}
\makeatother
\newcommand*{\qed}{\blacksquare}
\newcommand*{\M}[3]{\mathcal{M}_{#1\times#2} \left(#3\right)}
\newcommand*{\MR}[2]{\M{#1}{#2}{\mathbb{R}}}
\newcommand*{\MC}[2]{\M{#1}{#2}{\mathbb{C}}}
\newcommand*{\MK}[2]{\M{#1}{#2}{\mathbb{K}}}
\newcommand*{\T}[1]{{#1}^\text{T}}  % Trasposta di una matrice
\newcommand*{\sys}[1]{\left\{\begin{array}{@{}l@{}}#1\end{array}\right.}
\newcommand*{\m}[1]{\begin{bmatrix}#1\end{bmatrix}}
\DeclareMathOperator{\Tr}{Tr}  % Traccia di una matrice
\DeclareMathOperator{\rg}{rg}  % Rango di una matrice
\title{Metodo di risoluzione dei sistemi lineari}
\author{Riccardo Bergamaschi}
\date{\today}
\begin{document}
\maketitle
\section*{Soluzioni all'indietro, direttamente sulla matrice completa}

È possibile riscrivere l'algoritmo delle soluzioni all'indietro
in termini di operazioni elementari di riga e di colonna.

\subsection*{Osservazione}
Sia $M$ una matrice ridotta a scala.
Allora, se sommiamo un multiplo di una qualche riga $M_i$
a una qualche altra riga $M_j$ con $j<i$, la matrice che ne
risulta è ancora ridotta a scala.

\subsection*{Caso semplice: una e una sola soluzione}
Sia $SX=b$ un sistema lineare ridotto a scala,
compatibile e con $S$ matrice quadrata ($n$ equazioni, $n$ incognite).
Allora, $S|b$ è della forma:
\[\begin{bmatrix}[cccc|c]
    s_{11} & s_{21} & \dots & s_{1n} & b_0 \\
    0      & s_{22} & \dots & s_{2n} & b_1 \\
    \vdots & \vdots & \ddots & \vdots & \vdots \\
    0      & 0      & \dots & s_{nn} & b_n \\
\end{bmatrix}\]
Ovvero, $S$ è triangolare superiore.
L'idea è quella di arrivare ad $S = \text{Id}_n$ tramite operazioni
elementari di riga: in questo modo, il vettore dei coefficienti sarà
esattamente il vettore soluzione.

Possiamo per prima cosa rendere tutti gli elementi della diagonale
principale pari ad $1$, dividendo ogni riga per il relativo perno
(che è sempre $\ne 0$). La matrice sarà allora della forma:

\[\begin{bmatrix}[cccc|c]
    1      & s_{21} & \dots & s_{1n} & b_0 \\
    0      & 1      & \dots & s_{2n} & b_1 \\
    \vdots & \vdots & \ddots & \vdots & \vdots \\
    0      & 0      & \dots & 1      & b_n \\
\end{bmatrix}\]

Ora, vogliamo mandare $s_{21}$ a $0$. Per fare questo,
sottraiamo alla prima riga $s_{21}$ volte la seconda.
Analogamente, per mandare a $0$ $s_{31}$ ed $s_{32}$ sottrarremo
alla seconda e alla terza riga gli adeguati multipli di $S_3$.
Questo procedimento permette di ridurre la matrice dei coefficienti
alla matrice identità:

\[\begin{bmatrix}[cccc|c]
    1      & 0      & \dots  & 0      & b_0 \\
    0      & 1      & \dots  & 0      & b_1 \\
    \vdots & \vdots & \ddots & \vdots & \vdots \\
    0      & 0      & \dots  & 1      & b_n \\
\end{bmatrix}\]

A questo punto, l'unica soluzione del sistema sarà evidente.

\subsubsection*{Esempio}
Risolviamo il seguente sistema lineare:
\[\left\{\begin{array}{@{}l@{}}
    x - y + 2z = 1 \\
    3x + y + 3z = 6 \\
    x + 3y + z = -1 \\
\end{array}\right.\,\]
Per prima cosa, scriviamo e riduciamo a scala la matrice completa:
\[
    \begin{bmatrix}[ccc|c]
        1 & -1 & 2 & 1 \\
        3 & 1 & 3 & 6 \\
        1 & 3 & 1 & -1 \\
    \end{bmatrix} \leadsto \begin{bmatrix}[ccc|c]
        1 & -1 & 2 & 1 \\
        0 & 4 & -3 & 3 \\
        0 & 4 & -1 & -2 \\
    \end{bmatrix} \leadsto \begin{bmatrix}[ccc|c]
        1 & -1 & 2 & 1 \\
        0 & 4 & -3 & 3 \\
        0 & 0 & 2 & -5 \\
    \end{bmatrix}
\]
Possiamo osservare che il sistema è compatibile
e ammette una e una sola soluzione.

Ora applichiamo l'algoritmo sopra descritto:
\[\begin{aligned}
    \begin{bmatrix}[ccc|c]
        1 & -1 & 2 & 1 \\
        0 & 4 & -3 & 3 \\
        0 & 0 & 2 & -5 \\
    \end{bmatrix} &\leadsto \begin{bmatrix}[ccc|c]
        1 & -1 & 2 & 1 \\
        0 & 1 & -\sfrac{3}{4} & \sfrac{3}{4} \\
        0 & 0 & 1 & -\sfrac{5}{2} \\
    \end{bmatrix} \\ &\leadsto \begin{bmatrix}[ccc|c]
        1 & 0 & \sfrac{5}{4} & \sfrac{7}{4} \\
        0 & 1 & -\sfrac{3}{4} & \sfrac{3}{4} \\
        0 & 0 & 1 & -\sfrac{5}{2} \\
    \end{bmatrix} \\ &\leadsto \begin{bmatrix}[ccc|c]
        1 & 0 & \sfrac{5}{4} & \sfrac{7}{4} \\
        0 & 1 & 0 & -\sfrac{9}{8} \\
        0 & 0 & 1 & -\sfrac{5}{2} \\
    \end{bmatrix} \\ &\leadsto \begin{bmatrix}[ccc|c]
        1 & 0 & 0 & \sfrac{39}{8} \\
        0 & 1 & 0 & -\sfrac{9}{8} \\
        0 & 0 & 1 & -\sfrac{5}{2} \\
    \end{bmatrix}
\end{aligned}\]

Allora l'unica soluzione del sistema è $\begin{bmatrix}
    \sfrac{39}{8} \\ -\sfrac{9}{8} \\ -\sfrac{5}{2}
\end{bmatrix}$.


\[\begin{aligned}
    \begin{bmatrix}[cccc|c]
        2 & -1 & 1 & -1 & 3 \\
        0 & 5 & 5 & 3 & -1 \\
        0 & 0 & 6 & -2 & 2 \\
    \end{bmatrix} &\leadsto \begin{bmatrix}[cccc|c]
        2 & -1 & 1 & -1 & 3 \\
        0 & 5 & 5 & 3 & -1 \\
        0 & 0 & 3 & -1 & 1 \\
    \end{bmatrix} \\&\leadsto \begin{bmatrix}[cccc|c]
        2 & 4 & 6 & 2 & 2 \\
        0 & 5 & 5 & 3 & -1 \\
        0 & 0 & 3 & -1 & 1 \\
    \end{bmatrix} \\&\leadsto \begin{bmatrix}[cccc|c]
        1 & 2 & 3 & 1 & 1 \\
        0 & 5 & 5 & 3 & -1 \\
        0 & 0 & 3 & -1 & 1 \\
    \end{bmatrix} \\&\leadsto \begin{bmatrix}[cccc|c]
        1 & 2 & 3 & 1 & 1 \\
        0 & 1 & -1 & 1 & -3 \\
        0 & 0 & 3 & -1 & 1 \\
    \end{bmatrix} \\&\leadsto \begin{bmatrix}[cccc|c]
        1 & 0 & 5 & -1 & 7 \\
        0 & 1 & -1 & 1 & -3 \\
        0 & 0 & 3 & -1 & 1 \\
    \end{bmatrix}
\end{aligned}\]

\pagebreak

\subsection*{Esercizio 5.4}

\[A = \begin{bmatrix}
    1 & t & 1 \\
    t & t & t \\
    1 & t & t \\
\end{bmatrix}
\overset{A_2-tA_1}{\underset{A_3-A_1}{\longrightarrow}}
\begin{bmatrix}
    1 & t & 1 \\
    0 & t(1-t) & 0 \\
    0 & 0 & t-1 \\
\end{bmatrix}
\]

Se $a_{22} = t(1-t) \ne 0$, anche $a_{33} = t-1 \ne 0$,
e quindi la matrice è ridotta a scala. In tal caso, ha rango 3.

Se invece $a_{22} = t(1-t) = 0$, allora $t=0\vee t=1$.

Se $t=0$:
\[\begin{bmatrix}
    1 & 0 & 1 \\
    0 & 0 & 0 \\
    0 & 0 & -1 \\
\end{bmatrix}
\overset{A_2\leftrightarrow A_3}{\longrightarrow}
\begin{bmatrix}
    1 & 0 & 1 \\
    0 & 0 & -1 \\
    0 & 0 & 0 \\
\end{bmatrix}\]

In questo caso, la matrice ha rango 2.

Infine, se $t = 1$:
\[\begin{bmatrix}
    1 & 1 & 1 \\
    0 & 0 & 0 \\
    0 & 0 & 0 \\
\end{bmatrix}\]

In questo caso, la matrice ha rango 1.

In conclusione:

\[\rg{A} = \begin{cases}
    1 & t = 1 \\
    2 & t = 0 \\
    3 & t \notin \left\{ 0; 1 \right\} \\
\end{cases}\]



\subsection*{Esercizio 5.6}


\subsection*{Esercizio 5.7}

\pagebreak
\subsection*{Esercizio 9}
\[\sys{
    x_1 + kx_2 - x_3 = 0\\
    x_2 - h = 0\\
    x_1 - x_2 - kx_3 = h\\
    3x_1 - kx_2 = 0\\
}\]
\[\begin{aligned}
\m{[ccc|c]
    1 & k & -1 & 0 \\
    0 & 1 & 0 & h \\
    1 & -1 & -k & h \\
    3 & -k & 0 & 0 \\
}&\leadsto\m{[ccc|c]
    1 & k & -1 & 0 \\
    0 & 1 & 0 & h \\
    0 & -k-1 & -k+1 & h \\
    0 & -4k & 3 & 0 \\
}\\&\leadsto\m{[ccc|c]
    1 & k & -1 & 0 \\
    0 & 1 & 0 & h \\
    0 & 0 & -k+1 & (k+2)h \\
    0 & 0 & 3 & 4kh \\
}\\&\leadsto\m{[ccc|c]
    1 & k & -1 & 0 \\
    0 & 1 & 0 & h \\
    0 & 0 & 3 & 4kh \\
    0 & 0 & -k+1 & (k+2)h \\
}\\&\leadsto\m{[ccc|c]
    1 & k & -1 & 0 \\
    0 & 1 & 0 & h \\
    0 & 0 & 3 & 4kh \\
    0 & 0 & 0 & h(4k^2-11k+6)/3 \\
}
\end{aligned}\]

Se $(-2k^2+3k+2)h = 0$, le due rette sono incidenti.
Altrimenti, sono sghembe.

\end{document}
