\documentclass{article}
\usepackage[utf8]{inputenc}
\usepackage[italian]{babel}
\usepackage{amsmath}
\usepackage{amsfonts}
\usepackage{amssymb}
\usepackage{xfrac}
\makeatletter
\renewcommand*\env@matrix[1][*\c@MaxMatrixCols c]{%
\hskip -\arraycolsep
\let\@ifnextchar\new@ifnextchar
\array{#1}}
\makeatother
\newcommand*{\qed}{\blacksquare}
\newcommand*{\M}[3]{\mathcal{M}_{#1\times#2} \left(#3\right)}
\newcommand*{\MR}[2]{\M{#1}{#2}{\mathbb{R}}}
\newcommand*{\MC}[2]{\M{#1}{#2}{\mathbb{C}}}
\newcommand*{\MK}[2]{\M{#1}{#2}{\mathbb{K}}}
\newcommand*{\T}[1]{{#1}^\text{T}}  % Trasposta di una matrice
\newcommand*{\sys}[1]{\left\{\begin{array}{@{}l@{}}#1\end{array}\right.}
\newcommand*{\m}[1]{\begin{bmatrix}#1\end{bmatrix}}
\DeclareMathOperator{\Tr}{Tr}  % Traccia di una matrice
\DeclareMathOperator{\rg}{rg}  % Rango di una matrice
\begin{document}

Pag. 22 n. 3

Punto 1.
\[W=L\left(\m{1\\1\\1\\0},\m{1\\1\\0\\1},\m{1\\1\\2\\-1}\right)\]

Determiniamo una base di $W$. Riducendo a scala, otteniamo che il rango della
matrice $2$ e i perni corrispondono ai primi due vettori. Una base di $W$ è
allora:

\[B_W = \left\{\m{1\\1\\1\\0},\m{1\\1\\0\\1}\right\}\]

\textbf{Punto 2.} Determiniamo equazioni cartesiane per $B_W$:

\[
    \m{[cc|c] 1&1&x_1\\1&1&x_2\\1&0&x_3\\0&1&x_4}
    \longrightarrow
    \m{[cc|c] 1&1&x_1\\0&-1&x_3-x_1\\0&0&x_2-x_1\\0&0&x_4+x_3-x_1}
\]

Allora:
\[W = \left\{
    \m{x_1\\x_2\\x_3\\x_4}\in\mathbb{R}^4:
    x_2-x_1 = 0 \wedge x_4+x_3-x_1=0
\right\}\]

\[U: \sys{x_1-x_2+x_3-x_4=0\\x_2-3x_3+x_4=0}\]
\[U = \left\{
    \m{2x_3\\3x_3-x_4\\x_3\\x_4}:
    x_3,x_4\in\mathbb{R}
\right\}\]
\[U: x_3\m{2\\3\\1\\0}+x_4\m{0\\-1\\0\\1},\quad x_3,x_4\in\mathbb{R}\]

\textbf{Punto 4. $U+W=?$}

\[U+W=L\left(\m{1\\1\\1\\0},\m{1\\1\\0\\1},\m{2\\3\\1\\0},\m{0\\-1\\0\\1}\right)\]

\[
    \m{
        1&1&2&0\\
        1&1&3&-1\\
        1&0&1&0\\
        0&1&0&1\\
    }\longrightarrow\m{
        1&1&2&0\\
        0&1&0&-1\\
        0&0&1&1\\
        0&0&0&0\\
    }
\]

Il rango è $3$, quindi l'ultimo vettore è combinazione linerare degli altri
La dimensione di $U+W$ è $3$, e di conseguenza non può essere scritto come
somma diretta degli altri.

\[\mathbb{R}^4\]

\[W = L\left(\m{1\\1\\k\\k},\m{1\\1\\1\\1},\m{1\\h\\h\\1}\right)\]
\[U = L\left(\m{1\\1\\1\\h},\m{1\\k\\h\\1}\right)\]

\subsection*{12.}

\[\begin{aligned}
    T:&\mathbb{R}^3&\longrightarrow&\mathbb{R}^4\\
      &\m{1\\0\\0}&\longmapsto&\m{1\\0\\-1\\3}\\
      &\m{0\\1\\0}&\longmapsto&\m{0\\3\\4\\1}\\
      &\m{0\\0\\1}&\longmapsto&\m{1\\2\\1\\1}\\
\end{aligned}\]


\end{document}