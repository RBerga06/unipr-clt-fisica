\documentclass{article}
\usepackage[utf8]{inputenc}
\usepackage[italian]{babel}
\usepackage{amsmath}
\usepackage{amssymb}
\usepackage{siunitx}
\usepackage{tabularray}
\usepackage{graphicx}
\usepackage{float}
\usepackage[bottom]{footmisc}
\usepackage[page]{appendix}
\usepackage[mathscr]{euscript}  % per il corsivo
\usepackage[labelformat=simple, justification=centering]{subfig}
\renewcommand{\thesubfigure}{}
\newcommand*{\diam}{\varnothing}
\newcommand*{\best}[1]{{#1}_\text{best}}
\newcommand*{\bestp}[1]{{\left(#1\right)}_\text{best}}
\newcommand*{\pbest}[1]{\left({#1}_\text{best}\right)}
\newcommand*{\pbestp}[1]{\left({\left(#1\right)}_\text{best}\right)}
\newcommand*{\errrel}[1]{\frac{\delta #1}{{#1}_\text{best}}}
\title{
  Laboratorio di Fisica 1\\
  R11: Calorimetro ad azoto liquido
}
\author{Gruppo 15: Bergamaschi Riccardo, Graiani Elia, Moglia Simone}
\date{14/05/2024 – 21/05/2024}
\makeindex
\begin{document}

\maketitle

\begin{abstract}
  Mediante un calorimetro ad azoto liquido, il gruppo di lavoro
  ha misurato i calori specifici di quattro campioni;
  preliminarmente, è stato necessario determinare il
  calore latente di vaporizzazione dell'azoto.
\end{abstract}

\setcounter{section}{-1}
\section{Materiali e strumenti di misura utilizzati}
\begin{center}
\begin{tblr}{
  width=\textwidth,
  colspec={ X[2,m,j]X[1,m,c]X[1,m,c]X[1,m,c] },
  vlines,
}
  \hline
  \textbf{Strumento di misura} & \textbf{Soglia} & \textbf{Portata} & \textbf{Sensibilità} \\
  \hline
  Amperometro & $\qty{0.001}{A}$ & N./A. & $\qty{0.001}{A}$ \\
  \hline[dashed]
  Voltmetro & $\qty{0.01}{V}$ & N./A. & $\qty{0.01}{V}$ \\
  \hline[dashed]
  Cronometro & $\qty{0.01}{s}$ & $\qty{99.99}{s}$ & $\qty{0.01}{s}$ \\
  \hline[dashed]
  Bilancia di precisione & $\qty{0.01}{g}$ & $\qty{4000.0}{g}$ &
    $\qty{0.01}{g}$\footnotemark[1] \\
  \hline
\end{tblr}
\footnotetext[1]{
  Per misure superiori a $\qty{2000.00}{g}$, la sensibilità
  è $\qty{0.1}{g}$
}
\begin{tblr}{
  width=\textwidth,
  colspec={ X[m,j]X[3,m,j] },
  vlines,
}
  \hline
  \textbf{Altro} & \textbf{Descrizione/Note} \\
  \hline
  Calorimetro & {
    Quasi adiabatico,
    dotato di un coperchio con un foro centrale per potervi immergere
    i materiali e permettere la fuoriuscita dell'azoto gassoso.
  } \\
  \hline[dashed]
  Azoto liquido & Contenuto nel calorimetro \\
  \hline[dashed]
  Resistenza & {
    Fissata all'interno del coperchio, fornisce calore all'azoto liquido.
  } \\
  \hline[dashed]
  Videocamera & {
    Utilizzata per acquisire i dati mostrati dal
    cronometro e della bilancia di precisione
    contemporaneamente.
  } \\
  \hline[dashed]
  Generatore & {
    Fornisce corrente elettrica al circuito, composto dall'amperometro
    (collegato in serie) e da voltmetro e resistenza (collegati in
    parallelo).
  } \\
  \hline[dashed]
  Quattro campioni metallici noti & { Li chiameremo $\Xi,\Delta,\aleph,\nabla$. } \\
  \hline
\end{tblr}
\end{center}

\pagebreak

\section{Misura del calore latente di vaporizzazione dell'azoto}

\subsection{Esperienza e procedimento di misura}

\begin{enumerate}
  \item
    Posto il calorimetro sopra alla bilancia di precisione, avviamo
    l'acquisizione del filmato.
  \item
    Dopo almeno una decina di secondi, accendiamo il generatore
    in modo da fornire calore all'azoto per mezzo della resistenza.
  \item
    Mediante il voltmetro e l'amperometro, misuriamo, rispettivamente,
    la differenza di potenziale ($\Delta V$) ai capi della resistenza
    e l'intensità di corrente ($i$) sviluppate dal generatore\footnote{
In entrambe le misurazioni, abbiamo rilevato
$\Delta V = (3.27\pm0.01)\,\unit{V}$ e $i = (1.636\pm0.001)\,\unit{A}$.
    }.
  \item
    Passato circa un minuto, spegniamo il generatore per interrompere
    lo scambio di calore e, dopo almeno un'altra decina di secondi,
    terminiamo la registrazione del filmato.
\end{enumerate}

Il gruppo di lavoro ha effettuato questi passaggi due volte:
la prima, chiudendo il foro del coperchio con un tappo;
la seconda, lasciandolo aperto.

\subsection{Analisi dei dati raccolti}

Essendo l'azoto a temperatura di ebollizione, possiamo esprimere la
quantità di calore assorbito ($\delta Q$) in un intervallo di tempo
$\Delta t$ in funzione della massa di azoto evaporata ($-\Delta m$):
\[\delta Q = - \lambda_\text{vap} \Delta m\] dove la costante
$\lambda_\text{vap}$ è detta “calore latente di vaporizzazione”.

Assumendo le dispersioni di energia trascurabili, possiamo
considerare $\delta Q$ pari al calore sviluppato dalla resistenza per
effetto Joule. Vale allora:
\[ \delta Q = \mathscr{P}\cdot\Delta t = i\cdot\Delta V\cdot\Delta t\]

da cui:
\[
  \lambda_{vap} = \frac{i\cdot\Delta V\cdot\Delta t}{-\Delta m}
    = \frac{i\cdot\Delta V}{-\,\gamma}
  \qquad \text{avendo posto} \quad
  \gamma = \frac{\Delta m}{\Delta t}.
\]

Visionando il filmato, il gruppo di lavoro ha raccolto, a intervalli
di tempo regolari, la misura della massa di azoto liquido indicata
dalla bilancia.

Chiaramente, non essendo il calorimetro perfettamente adiabatico,
la massa di azoto diminuisce anche quando la resistenza è spenta.
Per tenere conto di questo errore sistematico, abbiamo effettuato
tre regressioni lineari (con coefficienti angolari $\alpha_1,\beta,\alpha_2$)
sui dati raccolti, rispettivamente prima,
durante e dopo lo scambio di calore con la resistenza. Abbiamo quindi calcolato:
\[\gamma = \beta - \frac{\alpha_1 + \alpha_2}{2}.\]

\pagebreak
Di seguito riportiamo graficamente i dati acquisiti, accompagnati dalle rette
di regressione e dai valori di $\gamma$ e $\lambda_\text{vap}$ calcolati.

\vspace{2mm}
\emph{
  \textbf{Nota.} La struttura di entrambi i grafici è la seguente:
  \begin{itemize}
    \item in blu, i dati raccolti con la resistenza accesa e la
      relativa retta di regressione (la cui zona di incertezza,
      estremamente ridotta, è rappresentata in azzurro);
    \item in rosso, i dati raccolti con la resistenza spenta e le
      rispettive rette di regressione (le cui zone di incertezza
      sono rappresentate in rosa);
    \item sono inoltre riportate le barre di errore,
      tuttavia così ridotte da risultare invisibili.
  \end{itemize}
}

\subsubsection{Prima acquisizione: foro chiuso}
\begin{figure}[H]
  \includegraphics[trim={2.5cm 0.6cm 3cm 1cm},clip,width=\textwidth]{img/g_azoto2.png}
\end{figure}
\[\begin{aligned}
  \alpha_{1,\text{chiuso}} &= (-59.2\pm0.5)\cdot10^{-6}\,\unit{kg\per s}
  \qquad
  \beta_{\text{chiuso}}\!\!\!\!&=(-&83.06\pm0.05)\cdot10^{-6}\,\unit{kg\per s}
  \\
  \alpha_{2,\text{chiuso}} &= (-71.3\pm0.5)\cdot10^{-6}\,\unit{kg\per s}
  \qquad
  \gamma_\text{chiuso}\!\!\!\!&=(-&17.8\pm0.6)\cdot10^{-6}\,\unit{kg\per s}
\end{aligned}\]
\[
  \lambda_\text{vap,chiuso} = (3.00\pm0.11)\cdot10^5\,\unit{J \per kg}
\]

\pagebreak
\subsubsection{Seconda acquisizione: foro aperto}
\begin{figure}[H]
  \includegraphics[trim={2.5cm 0.6cm 3cm 1cm},clip,width=\textwidth]{img/g_azoto3.png}
\end{figure}
\[\begin{aligned}
  \alpha_{1,\text{aperto}} &= (-54.1\pm0.6)\cdot10^{-6}\,\unit{kg\per s}
  \qquad
  \beta_{\text{aperto}}\!\!\!\!&=(-&78.28\pm0.07)\cdot10^{-6}\,\unit{kg\per s}
  \\
  \alpha_{2,\text{aperto}} &= (-57.0\pm0.2)\cdot10^{-6}\,\unit{kg\per s}
  \qquad
  \gamma_\text{aperto}\!\!\!\!&=(-&22.7\pm0.5)\cdot10^{-6}\,\unit{kg\per s}
\end{aligned}\]
\[
  \lambda_\text{vap,aperto} = (2.35\pm0.06)\cdot10^5\,\unit{J \per kg}
\]

\subsection{Conclusioni}
Confrontando i due valori di $\lambda_\text{vap}$ così ottenuti
con il valore atteso $\lambda_\text{vap,atteso} =
(1.9856\pm0.0001)\cdot10^5\,\unit{J\per kg}$,
possiamo osservare che nessuno dei due risulta compatibile con
quest'ultimo.

\vspace{2mm}
Per quanto riguarda $\lambda_\text{vap,chiuso}$, il gruppo di
lavoro ritiene che la presenza del tappo abbia impedito a una
parte significativa dell'azoto gassoso di fuoriuscire dal sistema,
portando la bilancia a misurare, in ogni momento, una massa di
azoto superiore a quella ancora in fase liquida. Ma, soprattutto,
la variazione di massa rispetto al tempo misurata mediante la
retta di regressione è risultata essere minore di quella effettiva,
portandoci a sovrastimare $\lambda_\text{vap}$ (ricordiamo che
$\gamma$ è al denominatore).

Questa ipotesi è coerente col fatto che $\lambda_\text{vap,aperto}$,
pur risultando anch'esso non compatibile, si avvicina di più al valore
di $\lambda_\text{vap,atteso}$.

\vspace{2mm}
Invece, riguardo a $\lambda_\text{vap,aperto}$, il gruppo di
lavoro ritiene che, durante l'esperienza, la resistenza non fosse
completamente immersa nell'azoto liquido: di conseguenza, parte
del calore sviluppato per effetto Joule è stato disperso,
probabilmente assorbito dall'azoto già in fase gassosa.
Di conseguenza, nell'analisi di cui sopra il calore assorbito
dall'azoto è stato sovrastimato – e, con esso, anche
$\lambda_\text{vap}$.

Questa ipotesi è sostenuta dal fatto che, come è possibile
osservare dal grafico, la massa totale iniziale di azoto si
aggirava attorno ai $\qty{308}{g}$, quando il calorimetro ne
avrebbe potuto contenere ben di più.

\section{Misura dei calori specifici dei campioni}

\subsection{Esperienza e procedimento di misura}

Per ogni campione:
\begin{enumerate}
  \item
    Come prima, avviamo la cattura del filmato dopo aver posizionato il calorimetro
    sopra alla bilancia di precisione.
  \item
    Dopo circa una decina di secondi, lo inseriamo nel calorimetro, che poi andiamo
    a chiudere tramite un tappo in plastica.
  \item
    Attesi uno o due minuti per assicurarci che lo scambio di calore con l'azoto
    sia terminato, possiamo rimuovere il campione dal calorimetro e, successivamente,
    interrompere la registrazione.

\end{enumerate}

\subsection{Analisi dei dati raccolti e conclusioni}
  Analogamente a quanto detto prima, possiamo esprimere la quantità di calore
  assorbito dall'azoto come: \[\delta Q = - \lambda_\text{vap} \Delta m\].

  Considerando il calorimetro totalmente adiabatico, questa quantità di calore
  è pari a quella ceduta dal campione. Possiamo quindi scrivere:
  \[\delta Q = c m \Delta T\].

  Da queste due relazioni è possibile ricavare un espressione per $c$:
  \[
    c = \frac{\lambda (-\gamma) \Delta t}{m \Delta T}
     \qquad \text{con} \quad
    \gamma = \frac{\Delta m}{\Delta t}.
  \]
  \emph{\textbf{Notazione.}
  Con il simbolo $c$ non indichiamo direttamente il calore specifico, in quanto dipende dalla temperatura;
  intendiamo, invece, una sua "media" tra $\qty{-196}{\degree C}$ e $\qty{25}{\degree C}$.
  }

\begin{center}
  \begin{tblr}{ |c|c|c|c|c| }
    \hline
    Oggetto & $L\;\;(\unit{cm})$ & $\diam\;\;(\unit{mm})$ & $m\;\;(\unit{g})$ & $I_\text{CM}\;\;(10^{-5}\,\unit{kg\,m^2})$ \\
    \hline
    Asta & $60.0\pm0.1$ & $5.94\pm0.01$ & $45.82\pm0.01$ & $568.5\pm1.5$ \\
    \hline[dashed]
    Rotore & N./A. & $13.41\pm0.01$ & $22.4\pm0.1^*$ & $0.058\pm0.001^*$ \\
    \hline
  \end{tblr}
\end{center}\begin{center}
  \begin{tblr}{ |c|c|c|c|c|c| }
    \hline
    $i$ & $m_i\;\;(\unit{g})$ & $d_i^\text{\,ext}\;\;(\unit{mm})$ & $d_i^\text{\,int}\;\;(\unit{mm})$ & $h_i\;\;(\unit{mm})$ & $I_{\text{CM},i}\;(\unit{mg\,m^2})$ \\
    \hline
    A & $115.95\pm0.01$ & $29.95 \pm 0.05$ & $6.20 \pm 0.05$ & $19.93 \pm 0.01$ & $10.62\pm0.03$ \\
    \hline[dashed]
    B & $115.86\pm0.01$ & $29.95 \pm 0.05$ & $6.20 \pm 0.05$ & $19.89 \pm 0.01$ & $10.59\pm0.03$\\
    \hline[dashed]
    C & $71.46\pm0.01$ & $29.95 \pm 0.05$ & $6.20 \pm 0.05$ & $12.08 \pm 0.01$ & $5.047\pm0.018$\\
    \hline
  \end{tblr}
\end{center}

\emph{$[^*]$ Valori dati}

\pagebreak
%\subsubsection{La misura di $T$}
Il periodo dell'oscillazione è stato misurato individuando $N+1$ zeri
consecutivi di $\theta(t)$, diciamo $\left\{t_0,t_1,\dots,t_N\right\}$.
Allora, poiché tra uno zero e l'altro corre metà periodo, è possibile
calcolare $T$ in questo modo: $T = \frac{2}{N}(t_N - t_0)$

Il gruppo di lavoro ha scelto $N$ di volta in volta, in modo tale che
fosse proporzionale al numero di oscillazioni compiute dal pendolo
prima di fermarsi. Complessivamente, $N$ ha assunto valori da $30$ a
$180$.

\vspace{2mm}
%\subsubsection{Il calcolo di $g$}

Come descritto sopra, il gruppo di lavoro ha calcolato, per ogni
configurazione $\Gamma$,
i valori di $\frac{I_z^\text{tot}}{Mr_\text{CM}}$
e $\frac{T^2}{4\pi^2}$, riportati nel grafico seguente.

Come è possibile osservare dalla relazione che le lega, la dipendenza
tra queste due grandezze è lineare: questo ci permette di determinare
il valore di $g$ come coefficiente angolare di una retta di regressione.

\begin{center}
\begin{figure}[H]
  %\includegraphics[trim={2cm 1cm 2cm 2.1cm},clip,width=\textwidth]{img/regressione.png}
  \caption[]{\emph{
    In rosso, la retta di regressione lineare e in rosa,
    appena visibile, la sua regione di incertezza.
    (le barre di errore sull'ascissa sono così ridotte
    da risultare invisibili)
  }}
\end{figure}
\end{center}

\begin{itemize}
  \item Intercetta $= (0.003 \pm 0.005)\;\unit{m}$
  \item Coefficiente angolare $g = (9.68 \pm 0.13)\;\unit{m\per s^2}$
\end{itemize}
\pagebreak
I risultati della regressione lineare sono chiaramente compatibili
con i valori attesi. Infatti:
\begin{itemize}
  \item Secondo il modello fisico utilizzato, l'intercetta dovrebbe
  essere nulla; in effetti, $(0.003\pm0.005)\;\unit{m}$ è compatibile
  con $\qty{0}{m}$.
  \item Il valore di $g$ atteso è $\qty{9.806}{m\per s^2}$; si può
  osservare facilmente che il valore misurato,
  $(9.68\pm0.13)\;\unit{m \per s^2}$, è compatibile con esso.
\end{itemize}

Possiamo pertanto concludere che l'esperienza ha avuto successo:
mediante l'apparato sperimentale abbiamo ottenuto una misura di $g$
compatibile con quella attesa.

\end{document}
