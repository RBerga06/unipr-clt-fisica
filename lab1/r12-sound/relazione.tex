\documentclass{article}
\usepackage[utf8]{inputenc}
\usepackage[italian]{babel}
\usepackage{amsmath}
\usepackage{amssymb}
\usepackage{siunitx}
\usepackage{tabularray}
\usepackage{graphicx}
\usepackage{float}
% \usepackage{minted}
\usepackage[bottom]{footmisc}
\usepackage[page]{appendix}
\newcommand*{\diam}{\varnothing}
\newcommand*{\best}[1]{{#1}_\text{best}}
\newcommand*{\bestp}[1]{{\left(#1\right)}_\text{best}}
\newcommand*{\pbest}[1]{\left({#1}_\text{best}\right)}
\newcommand*{\pbestp}[1]{\left({\left(#1\right)}_\text{best}\right)}
\newcommand*{\errrel}[1]{\frac{\delta #1}{{#1}_\text{best}}}
\title{
  Laboratorio di Fisica 1\\
  R12: Tubo di Kundt
}
\author{Gruppo 15: Bergamaschi Riccardo, Graiani Elia, Moglia Simone}
\date{28/05/2024 – 04/06/2024}
\makeindex
\begin{document}

\maketitle

\begin{abstract}
  Il gruppo di lavoro ha misurato la velocità del suono mediante
  lo studio di onde stazionarie in una colonna d'aria.

\end{abstract}

\setcounter{section}{-1}
\section{Materiali e strumenti di misura utilizzati}
\begin{center}
\begin{tblr}{
  width=\textwidth,
  colspec={ X[2,m,j]X[1,m,c]X[1,m,c]X[1,m,c] },
  vlines,
}
  \hline
  \textbf{Strumento di misura} & \textbf{Soglia} & \textbf{Portata} & \textbf{Sensibilità} \\
  \hline
  Metro a nastro & $\qty{0.1}{cm}$ & $\qty{300.0}{cm}$ & $\qty{0.1}{cm}$ \\
  \hline[dashed]
  Calibro ventesimale & $\qty{0.05}{mm}$ & $\qty{150.00}{mm}$ & $\qty{0.05}{mm}$ \\
  \hline[dashed]
  Oscilloscopio & $\qty{0.01}{s}$ & $\qty{99.99}{s}$ & $\qty{0.01}{s}$ \\
  \hline
\end{tblr}
\begin{tblr}{
  width=\textwidth,
  colspec={ X[m,j]X[3,m,j] },
  vlines,
}
  \hline
  \textbf{Altro} & \textbf{Descrizione/Note} \\
  \hline
  Tubo di Kundt & {
    Un cilindro in plastica nel quale facciamo propagare le onde.
  } \\
  \hline[dashed]
  Oscilloscopio & {
    Permette di visualizzare la forma d'onda emessa e quella
    rilevata dal microfono.
  } \\
  \hline[dashed]
  Generatore di funzioni d'onda & {
    Da cui possiamo regolare frequenza, ampiezza,
    e tipo delle onde generate.
  } \\
  \hline[dashed]
  Microfono a condensatore & {
    Utilizzato per rilevare le onde.
    } \\
  \hline[dashed]
  Pistone & {
    Finalizzato alla chiusura di un'estremità del tubo.
    } \\
  \hline
\end{tblr}
\end{center}

\pagebreak
\section{Misurazione delle frequenze di risonanza con tubo aperto}

\subsection{Esperienza e procedimento di misura}

\begin{enumerate}
  \item
    Con il metro a nastro misuriamo la lunghezza del tubo $L$, mentre con il
    calibro ventesimale il suo diametro $d$. %TODO: interno o esterno?
  \item
    Inseriamo il microfono dentro al tubo in modo che riesca a rilevare le onde,
    quindi assicurandoci che non si trovi in corrispondenza di un nodo.
  \item
    Accesi l'oscilloscopio e il generatore di forme d'onda, regoliamo
    l'ampiezza in modo da poter percepire un suono.
  \item
    Aumentiamo la frequenza fino a trovare il primo massimo relativo nel
    grafico riportato dall'oscilloscopio: questa sarà la prima frequenza
    di risonanza, nonché l'armonica fondamentale del nostro tubo.
  \item
    Ripetiamo più volte il passaggio precedente, in modo da ottenere almeno
    cinque frequenze di risonanza.
\end{enumerate}
Il gruppo di lavoro ha effettuato le stesse procedure, ma mantenendo una
estremità del tubo chiusa tramite il pistone. In questo caso la lunghezza
del tubo da considerare è quella che va dall'estremità aperta al pistone.

\subsection{Analisi dei dati raccolti}
\emph{\textbf{Nota.}
Avendo valutato gli errori sulle grandezze misurate direttamente
come piccoli, casuali e indipendenti, per svolgere ogni calcolo
abbiamo utilizzato la tradizionale propagazione degli errori.
}

  Grazie al filmato possiamo graficare la variazione della massa in funzione del tempo
  e, successivamente, costruire tre distinte rette di regressione.

  Essendo note, oltre all'intervallo di tempo, tensione ed intensità della corrente,
  è nota anche la quantità di calore fornito: $Q = I \Delta V \Delta t$.

  Il calore latente dell'azoto liquido sarà:

\subsection{Conclusioni}

Come è possibile osservare comparando questi risultati a
quelli precedentemente ottenuti, il valore di $g$ risultante
è rimasto essenzialmente invariato (al netto della sua incertezza).

In conclusione, possiamo affermare ragionevolmente che,
rispetto alla sensibilità degli strumenti di misura,
il contributo dell'attrito è trascurabile.

\section{Misurazione delle frequenze di risonanza con tubo aperto}

\subsection{Esperienza e procedimento di misura}

\begin{enumerate}
  \item
    Posto il calorimetro sopra alla bilancia, avviamo la cattura del filmato.
  \item
    Dopo circa una decina di secondi (non è rilevante per la riuscita dell'esperienza),
    tramite il generatore forniamo calore all'azoto per mezzo della resistenza.
  \item
    Aspettato ... interrompiamo il flusso di calore e dopo
    un'altra decina di secondi terminiamo la registrazione video.
\end{enumerate}

\subsection{Analisi dei dati raccolti}
\emph{\textbf{Nota.}
Avendo valutato gli errori sulle grandezze misurate direttamente
come piccoli, casuali e indipendenti, per svolgere ogni calcolo
abbiamo utilizzato la tradizionale propagazione degli errori.
}

  Grazie al filmato possiamo graficare la variazione della massa in funzione del tempo
  e, successivamente, costruire tre distinte rette di regressione.

  Essendo note, oltre all'intervallo di tempo, tensione ed intensità della corrente,
  è nota anche la quantità di calore fornito: $Q = I \Delta V \Delta t$.

  Il calore latente dell'azoto liquido sarà:

\subsection{Conclusioni}

Come è possibile osservare comparando questi risultati a
quelli precedentemente ottenuti, il valore di $g$ risultante
è rimasto essenzialmente invariato (al netto della sua incertezza).

In conclusione, possiamo affermare ragionevolmente che,
rispetto alla sensibilità degli strumenti di misura,
il contributo dell'attrito è trascurabile.

\section{Misurazione delle frequenze di risonanza con tubo aperto}

\subsection{Esperienza e procedimento di misura}

\begin{enumerate}
  \item
    Posto il calorimetro sopra alla bilancia, avviamo la cattura del filmato.
  \item
    Dopo circa una decina di secondi (non è rilevante per la riuscita dell'esperienza),
    tramite il generatore forniamo calore all'azoto per mezzo della resistenza.
  \item
    Aspettato ... interrompiamo il flusso di calore e dopo
    un'altra decina di secondi terminiamo la registrazione video.
\end{enumerate}

\subsection{Analisi dei dati raccolti}
\emph{\textbf{Nota.}
Avendo valutato gli errori sulle grandezze misurate direttamente
come piccoli, casuali e indipendenti, per svolgere ogni calcolo
abbiamo utilizzato la tradizionale propagazione degli errori.
}

  Grazie al filmato possiamo graficare la variazione della massa in funzione del tempo
  e, successivamente, costruire tre distinte rette di regressione.

  Essendo note, oltre all'intervallo di tempo, tensione ed intensità della corrente,
  è nota anche la quantità di calore fornito: $Q = I \Delta V \Delta t$.

  Il calore latente dell'azoto liquido sarà:

\subsection{Conclusioni}

Come è possibile osservare comparando questi risultati a
quelli precedentemente ottenuti, il valore di $g$ risultante
è rimasto essenzialmente invariato (al netto della sua incertezza).

In conclusione, possiamo affermare ragionevolmente che,
rispetto alla sensibilità degli strumenti di misura,
il contributo dell'attrito è trascurabile.

\end{document}
