\documentclass{article}
\usepackage[utf8]{inputenc}
\usepackage[italian]{babel}
\usepackage{amsmath}
\usepackage{amssymb}
\usepackage{siunitx}
\usepackage{tabularray}
\usepackage{graphicx}
\usepackage{float}
\usepackage{minted}
\usepackage[page]{appendix}
\newcommand*{\diam}{\varnothing}
\newcommand*{\best}[1]{{#1}_\text{best}}
\newcommand*{\bestp}[1]{{\left(#1\right)}_\text{best}}
\newcommand*{\pbest}[1]{\left({#1}_\text{best}\right)}
\newcommand*{\pbestp}[1]{\left({\left(#1\right)}_\text{best}\right)}
\newcommand*{\errrel}[1]{\frac{\delta #1}{{#1}_\text{best}}}
\title{
    Laboratorio di Fisica 1\\
    R6: Misura dei calori specifici di materiali ignoti
}
\author{Gruppo 17: Bergamaschi Riccardo, Graiani Elia, Moglia Simone}
\date{6/12/2023 – 13/12/2023}
\makeindex
\begin{document}

\maketitle

\begin{abstract}
    Il gruppo di lavoro ha misurato il calore specifico di tre solidi distinti
    per risalirne alla natura.
\end{abstract}

\section{Materiali e strumenti di misura utilizzati}
\begin{center}
    \begin{tblr}{ |Q[l,m]|Q[c,m]|Q[c,m]|Q[c,m]| }
        \hline
        \textbf{Strumento di misura} & \textbf{\:\:\:\:\:Soglia\:\:\:\:\:} & \textbf{Portata} & \textbf{Sensibilità} \\
        \hline
        Termometro digitale & $\qty{00}{\degree C}$ & $\qty{00}{\degree C}$ & $\qty{0.1}{\degree C}$ \\
        % \hline[dashed]
        % Termometro a mercurio & $\qty{0.1}{\degree C}?$ & $\qty{100}{\degree C}?$ & $\qty{0.1}{\degree C}?$ \\
        \hline[dashed]
        Barometro & $\qty{0}{hPa}?$ & $\qty{14000}{hPa}$ & $\qty{1}{hPa}$ \\
        \hline[dashed]
        Cilindro graduato & $\qty{1}{mL}$ & $\qty{100}{mL}$ & $\qty{1}{mL}$ \\
        \hline[dashed]
        Bilancia di precisione & $\qty{0.50}{g}$ & $\qty{4100.00}{g}$ & $\qty{0.01}{g}$ \\
        \hline
        \hline
        \textbf{Altro} & \SetCell[c=3]{l} \textbf{Descrizione/Note} \\
        \hline
        Calorimetro & \SetCell[c=3]{l} {Isolato termicamente, quasi adiabatico.} \\
        \hline[dashed]
        {Fornelletto e pentolino} & \SetCell[c=3]{l} {Per scaldare acqua e campioni.} \\
        \hline[dashed]
        Tre campioni solidi & \SetCell[c=3]{l} {Li chiameremo $A$, $B$ e $C$.} \\
        \hline
    \end{tblr}
\end{center}

\section{Misurazione della massa equivalente}
    
\subsection{Esperienza e procedimento di misura}
    
\begin{enumerate}
    \item
        Versiamo in un cilindro graduato $\qty{100}{mL}$ di acqua distillata
        ($c=\qty{4186}{J \per kg\,K}$) e, dopo averne misurato la massa con
        la bilancia di precisione, la scaldiamo in un pentolino.
    \item
        Ripetiamo il passaggio precedente, ma, invece di scaldarla, questa volta
        versiamo l'acqua a temperatura ambiente nel calorimetro.
        
    \emph{
        \textbf{Osservazione.} È meglio che le masse si equivalgano, e che la loro
        somma sia uguale all'acqua che utilizzeremo nella seconda parte dell'esperimento,
        in modo che il calorimetro si bagni allo stesso modo.
        }
        
    \item
        Quando l'acqua raggiunge lo stato di ebollizione, che corrisponde a 
        $\qty{100}{\degree C}$, salvo correzioni dovute alla pressione diversa da
        $\qty{1}{atm}$, la versiamo nel calorimetro e mescoliamo lentamente
        per evitare che l'acqua calda resti in alto. Il termometro digitale ci
        darà il valore della temperatura in funzione del tempo.
\end{enumerate}
        
\subsection{Analisi dei dati raccolti e conclusioni}
Per le leggi della termodinamica noi sappiamo che:
    \[
        m_\text{calda} c_\text{acqua} (T_\text{calda}-T_\text{eq}) =
        (m_\text{fredda} c_\text{acqua} + C_\text{calorimetro})(T_\text{eq}-T_\text{fredda})
    \]

Invece che misurare $C_\text{calorimetro}$ in $\unit{J\per K}$, possiamo considerare a
quanta acqua equivale il calorimetro dal punto di vista termico, ovvero la quantità di
acqua che assorbirebbe lo stesso calore del calorimetro. Quindi:
    \[
        m_\text{calda} (T_\text{calda}-T_\text{eq}) =
        (m_\text{fredda} + m_\text{equiv})(T_\text{eq}-T_\text{fredda})
    \]

    \emph{
        \textbf{Osservazione.} La massa equivalente ($m_\text{equiv}$) ci dà anche un idea di
        quanto il calorimetro disturbi la misura.
        }

Eseguendo una regressione lineare sui dati raccolti dal termometro digitale, rappresentati nel    %TODO: spiegare la regressione?
seguente grafico, abbiamo trovato calcolato il valore di $T_\text{eq}$. Dunque:
    \[
        m_\text{equiv} = \frac{m_\text{calda} (T_\text{calda}-T_\text{eq})}{(T_\text{eq}-T_\text{fredda}) - m_\text{fredda}}
    \]
ovvero $m_\text{equiv} = \qty{00}{g}$. Ora che abbiamo ottenuto questo valore,    %TODO: calcolare il valore
possiamo calcolare i calori specifici dei metalli di cui sono composti i campioni.



\section{Misurazione del calore specifico dei materiali ignoti}
    
\subsection{Esperienza e procedimento di misura}

\begin{enumerate}
    \item
        Versiamo nel pentolino una quantità d'acqua tale permettere l'immersione
        completa dei campioni in essa e la scaldiamo. Per fare ciò più velocemente
        e assicurarci di essere in stato di ebollizione, regoliamo la temperatura
        della piastra a $T>\qty{100}{\degree C}$.
    \item
        Misuriamo $\qty{200}{mL}$ di acqua, distillata ed a temperatura ambiente,
        e la versiamo nel calorimetro.
    \item
        Per ogni solido ($A$, $B$ e $C$):
    \begin{enumerate}
        \item
            Ne misuriamo la massa con la bilancia di precisione
        \item
            Una volta che l'acqua nel pentolino si trova in corrispondenza della
            transizione di fase, lo immergiamo in essa in modo che raggiunga la
            $T$ del sistema.
        \item
            Quando anch'esso raggiunge la temperatura di $\qty{100}{\degree C}$,
            lo spostiamo nel calorimetro e mescoliamo nuovamente. Come prima, sarà
            il termometro digitale a darci il valore di $T$ in funzione del tempo.
    \end{enumerate}
\end{enumerate}

\subsection{Analisi dei dati raccolti e conclusioni}
Grazie alle leggi della termodinamica sappiamo che:
    \[
        m_\text{met} c_\text{met} (T_\text{met}-T_\text{eq}) =
        (c_\text{acqua} m_\text{acqua} + C_\text{calorimetro}) (T_\text{eq}-T_\text{acqua})
    \]   
o meglio:
    \[
        m_\text{met} c_\text{met} (T_\text{met}-T_\text{eq}) =
        (m_\text{acqua} + m_\text{equiv}) c_\text{acqua} (T_\text{eq}-T_\text{acqua})
    \]

Eseguendo una regressione lineare sui dati raccolti dal termometro digitale, rappresentati nel    %TODO: spiegare la regressione?
seguente grafico, abbiamo trovato calcolato il valore di $T_\text{eq}$. Dunque:
    \[
        c_\text{met} = \frac{(m_\text{acqua} + m_\text{equiv}) c_\text{acqua} (T_\text{eq}-T_\text{acqua})}
        {m_\text{met} (T_\text{met}-T_\text{eq})}
    \]

Nella seguente tabella riportiamo i valori ottenuti per ogni solido con le relative incertezze, che abbiamo calcolato con la
propagazione standard degli errori in quanto piccole, sistematiche e indipendenti.

\begin{center}
    \begin{tabular}{ |c|c|c| }
        \hline
        \emph{Campione} & $m$ $\unit{g}$ & $c$ $\unit{J {kg}^{-1} K^{-1} } $ \\
        \hline
        $A$ & 12.43 ± 0.01 & 0.00 ± 0.00 \\
        $B$ & 28.73 ± 0.01 & 0.00 ± 0.00 \\    %TODO: calcolare i valori
        $C$ & 44.86 ± 0.01 & 0.00 ± 0.00 \\
        \hline
    \end{tabular}
\end{center}









\section{Scritto}
    "Il calorimetro è abbastanza adiabatico e questo lo valuterete nell'ultima parte
    dell'esperimento."


    Come ultima cosa misureremo la discesa esponenziale della temperatura dell'acqua
    dentro al calorimetro e in particolare il suo tempo caratteristico, ovvero la quantità
    di tempo $\tau$ che impiega l'acqua all'interno del calorimetro ad abbassare la sua
    temperatura di $(T_0 - T_\text{amb})e$ volte.

    La legge che segue questa discesa è: \[(T-T_\text{amb.})=(T_0-T_\text{amb.}) e^{-t/\tau}\]

    Il parametro $\tau$ descrive quanto bene il calorimetro trattenga il calore
    (quindi sia adiabatico). Per misurarlo abbiamo scaldato $\qty{200}{g}$ d'acqua, che abbiamo
    poi lasciato raffreddare nel calorimetro in partenza vuoto, registrandone la $T$.

    L'equazione della regressione lineare che abbiamo utilizzato è:
    \[\ln(T-T_\text{amb.})=\ln(T_0-T_\text{amb.})-\frac{1}{\tau}t\]

\section{Conclusioni}
Per valutare numericamente la consistenza tra i due valori di $k$ ottenuti,
abbiamo calcolato il seguente valore (numero puro):
\[
    \varepsilon =
    \frac{
        \left|\left(k_\text{statica}\right)_\text{best} - \left(k_\text{dinamica}\right)_\text{best}\right|
    }{
        \delta k_\text{statica} + \delta k_\text{dinamica}
    }
\]
Allora $k_\text{statica}$ e $k_\text{dinamica}$ sono consistenti se e solo se $\varepsilon \le 1$.

Nel nostro caso, $\varepsilon = 1.33$. Il gruppo di lavoro ha ipotizzato che
questa inconsistenza (comunque contenuta, seppur non trascurabile) fra le due
misure possa essere ragionevolmente giustificata dalla difficoltà incontrata
nel ridurre al minimo le oscillazioni in direzione perpendicolare a $\vec{g}$;
considerato inoltre che la posizione dei fototraguardi non era ottimale, ciò
potrebbe avere ulteriormente influenzato la distribuzione dei tempi. È in
effetti possibile osservare che le distribuzioni da noi ottenute non sono,
il più delle volte, del tutto simmetriche: la moda sembra essersi spostata
leggermente a sinistra – un possibile sintomo dell'influenza di un
errore sistematico sulle misure.

\end{document}
