\documentclass{article}
\usepackage[utf8]{inputenc}
\usepackage[italian]{babel}
\usepackage{amsmath}
\usepackage{amssymb}
\usepackage{siunitx}
\usepackage{tabularray}
\usepackage{graphicx}
\usepackage{float}
\usepackage{minted}
\usepackage{caption}
\usepackage[page]{appendix}
\usepackage[bottom]{footmisc}
\newcommand*{\diam}{\varnothing}
\newcommand*{\best}[1]{{#1}_\text{best}}
\newcommand*{\bestp}[1]{{\left(#1\right)}_\text{best}}
\newcommand*{\pbest}[1]{\left({#1}_\text{best}\right)}
\newcommand*{\pbestp}[1]{\left({\left(#1\right)}_\text{best}\right)}
\newcommand*{\errrel}[1]{\frac{\delta #1}{{#1}_\text{best}}}
\newcommand*{\Th}{^{232}_{\;\;90} \text{Th}}
\renewcommand{\appendixpagename}{Appendici}
\title{
    Laboratorio di Fisica 1\\
    R4: Misura di variabili aleatorie
}
\author{Gruppo 17: Bergamaschi Riccardo, Graiani Elia, Moglia Simone}
\date{8/11/2023 – 15/11/2023}
\makeindex
\begin{document}

\maketitle

\begin{abstract}
    Il gruppo di lavoro ha misurato due variabili aleatorie, osservando come queste
    rispecchino le rispettive distribuzioni teoriche (di Bernoulli e di Poisson).
\end{abstract}

\section{Processo di Bernoulli}

\subsection{Dati sperimentali}
Eseguiamo 400 lanci di sei dadi distinti\footnote{Li distinguiamo in base al colore},
registrandone tutti i risultati.
Per ogni possibile risultato $s\in\left[1;6\right]\cap\mathbb{N}$, possiamo così
definire una variabile aleatoria\footnote{\emph{Notazione.} Per noi $0\in\mathbb{N}$.}
$x_s\in\left[0;6\right]\cap\mathbb{N}$ come il
numero di dadi, fra i sei lanciati, con risultato pari ad $s$.
Possiamo considerare il lancio dei sei dadi come un processo di Bernoulli,
in quanto i risultati dei dadi sono indipendenti fra loro. Di conseguenza,
la distribuzione di probabilità di $x_s$ è data da:
\[
    p \left(x_s=k\right) =
        \binom{6}{k}
        \left(\frac{1}{6}\right)^k
        \left(\frac{5}{6}\right)^{6-k}
        \qquad\forall k\in\left[0;6\right]\cap\mathbb{N}
\]
Di seguito riportiamo gli istogrammi dei dati così raccolti, assieme ai valori attesi,
calcolati mediante la distribuzione teorica.

\begin{center}
    \begin{figure}[H]
        % trim={< v > ^}
        \includegraphics[trim={2cm .5cm 2.4cm 2.1cm},clip,width=.5\textwidth]{img/Dadi1.jpg}
        \includegraphics[trim={2cm .5cm 2.4cm 2.1cm},clip,width=.5\textwidth]{img/Dadi2.jpg}
        \includegraphics[trim={2cm .5cm 2.4cm 2.1cm},clip,width=.5\textwidth]{img/Dadi3.jpg}
        \includegraphics[trim={2cm .5cm 2.4cm 2.1cm},clip,width=.5\textwidth]{img/Dadi4.jpg}
        \includegraphics[trim={2cm .5cm 2.4cm 2.1cm},clip,width=.5\textwidth]{img/Dadi5.jpg}
        \includegraphics[trim={2cm .5cm 2.4cm 2.1cm},clip,width=.5\textwidth]{img/Dadi6.jpg}
    \end{figure}
\end{center}

Come è possibile osservare da questi grafici, i risultati riportati sembrano seguire
grossomodo la distribuzione teorica. Tuttavia, presentano deviazioni osservabili; il
gruppo di lavoro ritiene che ciò sia principalmente dovuto al ridotto numero di lanci.

% \subsubsection*{Onestà dei dadi}
% Avendo segnato tutti i risultati di ogni dado, possiamo inoltre stimare se i dadi che
% abbiamo utilizzato sono truccati o meno. Infatti, su un dado onesto ci aspettiamo
% che escano tutti i risultati possibili con equa probabilità. Di seguito riportiamo
% gli istogrammi dei valori usciti su ogni dado.

% \begin{figure*}
%     \caption*{...}
% \end{figure*}

\subsection{Simulazione}
Tramite un programma da noi scritto e compilato\footnote{\emph{Vedi} Appendice A},
simuliamo la stessa esperienza con $10^{12}$ lanci dei sei dadi, al fine di
verificare la legge dei grandi numeri.

Di seguito riportiamo, in un istogramma, i risultati della simulazione.

\begin{center}
    \begin{figure}[H]
        % trim={< v > ^}
        \includegraphics[trim={2cm .5cm 2cm 2.1cm},clip,width=\textwidth]{img/DadiSimul.png}
    \end{figure}
\end{center}

Come è possibile osservare dal grafico, i risultati della simulazione sono, in proporzione,
talmente vicini alla distribuzione teorica da risultare pressoché indistinguibili da essa.

Ciò è in accordo con la legge dei grandi numeri, secondo la quale, al crescere del
numero di prove, la distribuzione di probabilità rappresenta sempre meglio i
risultati ottenuti, normalizzati.

\pagebreak
\section{Processo di Poisson}
\subsection{Materiali e strumenti di misura utilizzati}
\begin{center}
    \begin{tblr}{ |Q[l,m]|Q[c,m]|Q[c,m]|Q[c,m]| }
        \hline
        \textbf{Strumento di misura} & \textbf{\:\:\:\:\:Soglia\:\:\:\:\:} & \textbf{Portata} & \textbf{Sensibilità} \\
        \hline
        {Contatore Geiger} & \qty{1}{conteggi \per s} & N./A. & \qty{1}{conteggi \per s} \\
        \hline[dashed]
        Metro a nastro & \qty{0.1}{cm} & \qty{300.0}{cm} & \qty{0.1}{cm} \\
        \hline
        \hline
        \textbf{Altro} & \SetCell[c=3]{l} \textbf{Descrizione/Note} \\
        \hline
        {Campione di $\Th$} & \SetCell[c=3]{l} {
            Componente di una lampada da campeggio
        } \\
        \hline
    \end{tblr}
\end{center}


\subsection{Esperienza e procedimento di misura}

Posizionato il contatore Geiger a una certa distanza $d_i$ dal campione di $\Th$
(con $i\in\left[1;4\right]\cap\mathbb{N}$),
definiamo una variabile aleatoria\footnote{
    L'unità di misura di $x_i$ (e, conseguentemente, anche di $\overline{x_i}$) è
    $\unit{s^{-1}}$. Detto altrimenti,
    $x_i\in\left\{n\,\unit{s^{-1}}:n\in\mathbb{N}\right\}$
} $x_i$ come il numero di raggi $\gamma$
emessi dal $\Th$ nell'arco di un secondo, nella direzione del contatore.
Allora, detta $\overline{x_i}$ la media teorica
% \footnote{
    % Che il parametro della distribuzione coincida con la media è facilmente
    % dimostrabile. Detto $\lambda$ quel parametro, vale:
    % \[\begin{aligned}
        % \overline{x_\gamma} &= \sum_{k=0}^{+\infty} k\,p(x_\gamma=k) =
        % \sum_{k=0}^{+\infty} k \frac{\lambda^k  e^{-\lambda}}{k!} =
        % e^{-\lambda} \sum_{k=0}^{+\infty} k \frac{\lambda^k}{k!} =
        % e^{-\lambda} \left(0 + \sum_{k=1}^{+\infty} k \frac{\lambda^k}{k!}\right) \\ &=
        % \lambda e^{-\lambda} \sum_{k=1}^{+\infty} \frac{\lambda^{k-1}}{(k-1)!} =
        % \lambda e^{-\lambda} \sum_{j=0}^{+\infty} \frac{\lambda^j}{j!} =
        % \lambda e^{-\lambda} e^\lambda =
        % \lambda\quad\square
    % \end{aligned}\]
% }
di $x_i$, la distribuzione di probabilità di $x_i$ è data da una Poissoniana:
\[
    p(x_i=k)=\frac{\overline{x_i}^k e^{-\overline{x_i}}}{k!}
    \qquad
    \forall k\in\mathbb{N}
\]
Il valore di $\overline{x_i}$ è legato alla distanza $d_i$, al raggio $r$
della finestra del contatore, al tempo di dimezzamento $T_\frac{1}{2}$ del
$\Th$ e al numero $N$ di atomi di $\Th$ secondo la seguente relazione\footnote{
    Il numero medio $\overline{X}$ di raggi $\gamma$ emessi dal campione è
    $\overline{X} = \tau^{-1}N = {T^{-1}_\frac{1}{2}}N\ln{2}$,
    con $\tau=\frac{1}{\ln{2}}T_\frac{1}{2}$ il tempo caratteristico del $\Th$.
    Tuttavia, poiché la superficie di acquisizione è $\pi r^2$, e non $4\pi d_i^2$,
    dobbiamo moltiplicare per il rapporto fra le aree:
    \[
        \overline{x_i} = \frac{\pi r^2}{4\pi d_i^2} \overline{X} =
        \frac{Nr^2\ln{2}}{4 d_i^2 T_\frac{1}{2}}
    \]
    Infine, dobbiamo ricordare che il contatore Geiger non rileva soltanto le
    radiazioni emesse dal campione. Possiamo tenere conto di tutti gli altri
    contributi introducendo un termine costante $\overline{x_0}$, che chiameremo
    “media dei conteggi relativi alla radioattività ambientale”.
    Si ottiene così la relazione riportata.
}:
\[
    \overline{x_i} = \frac{Nr^2\ln{2}}{4 d_i^2 T_\frac{1}{2}} + \overline{x_0}
\]
D'ora in avanti indicheremo con $\xi$ la costante $\frac{Nr^2\ln{2}}{4T_\frac{1}{2}}$
(unità di misura: $\unit{m^2\per s}$).
La relazione diventa allora:
\[\overline{x_i} = \xi d_i^{-2} + \overline{x_0}\]

Per ogni distanza $d_i$, il gruppo di lavoro ha acquisito ripetutamente il valore di
$x_i$ per un tempo complessivo di circa un'ora\footnote{
    Abbiamo scelto deliberatamente di acquisire esattamente 3657 secondi in quanto
    $3657$ minimizza la funzione
    $f(x)=\left\{\frac{x}{\pi}\right\}=\frac{x}{\pi} - \left\lfloor\frac{x}{\pi}\right\rfloor$
    meglio di $3600$.
} ($\qty{3657}{s}$); ha poi acquisito nuovamente $x_4$ col contatore Geiger rivolto in
direzione opposta, per avere una stima diretta di $\overline{x_0}$.

Di seguito riportiamo gli istogrammi dei dati così raccolti,
assieme ai valori attesi, calcolati mediante la distribuzione teorica.

\begin{center}
    \begin{figure}[H]
        % trim={< v > ^}
        \includegraphics[trim={2cm .5cm 2.4cm 2.1cm},clip,width=.5\textwidth]{img/Geiger2.jpg}
        \includegraphics[trim={2cm .5cm 2.4cm 2.1cm},clip,width=.5\textwidth]{img/Geiger1.jpg}
        \includegraphics[trim={2cm .5cm 2.4cm 2.1cm},clip,width=.5\textwidth]{img/Geiger4.jpg}
        \includegraphics[trim={2cm .5cm 2.4cm 2.1cm},clip,width=.5\textwidth]{img/Geiger5.jpg}
    \end{figure}\begin{figure}[H]
        \centering
        \includegraphics[trim={2cm .5cm 2.4cm 2.1cm},clip,width=.5\textwidth]{img/Geiger0.jpg}
        \caption*{Conteggi, nell'ordine, di $x_1$, $x_2$, $x_3$, $x_4$ e $x_0$.}
    \end{figure}
\end{center}

Come si può osservare da questi grafici, i risultati ottenuti si allineano molto bene
alle distribuzioni di Poisson.

In particolare, $\overline{x_0} = \left(194\pm7\right)\cdot10^{-3}\,\unit{s^{-1}}$.

Per valutare l'accuratezza della nostra stima di $\overline{x_0}$, possiamo effettuare
una regressione lineare (pesata) utilizzando l'equazione di $x_i$ in funzione di
$d_i^{-2}$: \[\overline{x_i} = \xi d_i^{-2} + \overline{x_0}\]
Di seguito riportiamo una tabella con i dati utilizzati per la regressione lineare,
assieme a un grafico della retta di regressione stessa.

\begin{center}
    \begin{tblr}{ |Q[c,m]|Q[c,m]|Q[c,m]|Q[c,m]| }
        \hline
        $i$ & $d_i\;\;(\unit{cm})$ & $d_i^{-2}\;\;(\unit{m^{-2}})$ & $\overline{x_i}$ \\
        \hline
        1 & $9.5\pm0.1$  & $111\pm2$      & $0.809\pm0.015$\\
        2 & $10.3\pm0.1$ & $94.3\pm1.8$   & $0.737\pm0.014$\\
        3 & $26.5\pm0.1$ & $14.24\pm0.11$ & $0.272\pm0.009$\\
        4 & $41.2\pm0.1$ & $5.89\pm0.03$  & $0.219\pm0.008$\\
        \hline
    \end{tblr}
    \begin{figure}[H]
        % trim={< v > ^}
        \includegraphics[trim={2cm .5cm 2cm 2.1cm},clip,width=\textwidth]{img/Regressione.png}
        \caption*{Regressione lineare. In rosa la regione di incercezza.}
    \end{figure}
\end{center}

Risultati della regressione lineare:
\begin{itemize}
    \item $\overline{x_0} = \left(188\pm6\right)\cdot10^{-3}\,\unit{s^{-1}}$ (intercetta)
    \item $\xi = \left(5.70\pm0.13\right)\cdot10^{-3}\,\unit{m^2\per s}$ (coefficiente angolare)
\end{itemize}

Per valutare numericamente la consistenza tra i due valori di $\overline{x_0}$
ottenuti ($\overline{x_0}_\text{\,diretto}$, misurato direttamente, e
$\overline{x_0}_\text{\,indiretto}$, ottenuto dalla regressione lineare),
abbiamo calcolato il seguente valore (numero puro):
\[
    \varepsilon =
    \frac{
        \left(\overline{x_0}_\text{\,diretto}\right)_\text{best} -
        \left(\overline{x_0}_\text{\,indiretto}\right)_\text{best}
    }{
        \delta\overline{x_0}_\text{\,diretto} +
        \delta\overline{x_0}_\text{\,indiretto}
    }
\]
Allora $\overline{x_0}_\text{\,diretto}$ e $\overline{x_0}_\text{\,indiretto}$
sono consistenti se e solo se $\left|\varepsilon\right|\le1$.

Nel nostro caso, $\varepsilon = +0.45$, per cui i due valori di $\overline{x_0}$
sono consistenti.

Dal valore di $\xi$ ottenuto mediante la regressione lineare, è possibile stimare
il numero $N$ di atomi (e, di conseguenza, la massa) di $\Th$ nel campione.

Dalle formule precedentemente esposte, segue:
\[
    N = \frac{4\,\xi\,T_\frac{1}{2}}{r^2\ln{2}}
    = \left(2.98\pm0.09\right)\cdot10^{20}
\]
da cui:
\[\begin{aligned}
    m &= m_{\left(\text{1 atomo di }\Th\right)}N
       = (90(m_p+m_e) + (232-90)m_n)N\\
      &= \left(1.16\pm0.04\right)\cdot10^{-4}\,\unit{kg}
       = \left(0.116\pm0.004\right)\unit{g}
       = \left(116\pm4\right)\unit{mg}.
\end{aligned}\]

\pagebreak
\begin{appendices}
    \section{Codice Rust per $10^{12}$ lanci di sei dadi}
    Qui riportiamo il codice Rust, da noi scritto, che ci ha permesso di
    lanciare virtualmente $6\cdot10^{12}$ dadi in maniera estremamente
    efficiente.

    \inputminted[linenos, mathescape]{rust}{src/main.rs}
\end{appendices}

\end{document}
