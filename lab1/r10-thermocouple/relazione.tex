\documentclass{article}
\usepackage[utf8]{inputenc}
\usepackage[italian]{babel}
\usepackage{amsmath}
\usepackage{amssymb}
\usepackage{siunitx}
\usepackage{tabularray}
\usepackage{graphicx}
\usepackage{float}
\usepackage{xfrac}
\usepackage{caption}    % for \caption*{}
\newcommand*{\diam}{\varnothing}
\newcommand*{\best}[1]{{#1}_\text{best}}
\newcommand*{\bestp}[1]{{\left(#1\right)}_\text{best}}
\newcommand*{\pbest}[1]{\left({#1}_\text{best}\right)}
\newcommand*{\pbestp}[1]{\left({\left(#1\right)}_\text{best}\right)}
\newcommand*{\errrel}[1]{\frac{\delta #1}{{#1}_\text{best}}}
%% <custom footnotes/>
%\newcounter{savefootnote}
%\newcounter{symfootnote}
%\newcommand{\symfootnote}[1]{%
%   \setcounter{savefootnote}{\value{footnote}}%
%   \setcounter{footnote}{\value{symfootnote}}%
%   \ifnum\value{footnote}>8\setcounter{footnote}{0}\fi%
%   \let\oldthefootnote=\thefootnote%
%   \renewcommand{\thefootnote}{\fnsymbol{footnote}}%
%   \footnote{#1}%
%   \let\thefootnote=\oldthefootnote%
%   \setcounter{symfootnote}{\value{footnote}}%
%   \setcounter{footnote}{\value{savefootnote}}%
%}
%% </custom footnotes>
\title{
  Laboratorio di Fisica 1\\
  R8: Taratura di una termocoppia
}
\author{Gruppo 15: Bergamaschi Riccardo, Moglia Simone, Graiani Elia}
\date{30/04/2024 – 07/05/2024}
\makeindex
\begin{document}

\maketitle

\begin{abstract}
  Il gruppo di lavoro ha determinato la curva di calibrazione di una
  termocoppia sfruttando punti fissi, ovvero temperature note,
  di svariate sostanze chimiche.
\end{abstract}

\setcounter{section}{-1}  % Count sections starting from 0
\section{Materiali e strumenti di misura utilizzati}
\begin{center}
  \begin{tblr}{
    width=\textwidth,
    colspec={ X[2,m,j]X[m,c]X[m,c]X[m,c] },
    vlines,
  }
    \hline
    \textbf{Strumento di misura} & \textbf{Soglia} & \textbf{Portata} & \textbf{Sensibilità} \\
    \hline
    Termocoppia (tipo K) & \qty{-6.03}{mV} & \qty{50.64}{mV} & \qty{0.01}{mV} \\
    \hline[dashed]
    Cronometro & \qty{0.01}{s} & N./A. & \qty{0.01}{s} \\
    \hline[dashed]
    Termometro ambientale & \qty{-10.0}{\degree C}? & \qty{50.64}{\degree C}? & \qty{0.5}{\degree C} \\
    \hline
  \end{tblr}
  \begin{tblr}{
    width=\textwidth,
    colspec={ X[2,m,j]X[3,m,j] },
    vlines,
  }
    \hline
    \textbf{Altro} & \textbf{Descrizione/Note} \\
    \hline
    Campioni di sostanze chimiche & {
      Azoto liquido, acqua distillata,
      etanolo, gallio, e indio.
    } \\
    \hline[dashed]
    Amplificatore di voltaggio & {
      Amplifica di un fattore 100 il voltaggio rilevato dalla
      termocoppia, rendendo possibile l'acquisizione dati.
    } \\
    \hline[dashed]
    Fornelletto e pentolino & Per scaldare i campioni. \\
    \hline[dashed]
    Cacciavite & Utilizzato per collegare la termocoppia all'interfaccia. \\
    \hline[dashed]
    { Guanto da forno, pinzette, presine e contenitori isolanti } & {
      Per maneggiare i campioni in sicurezza.
    } \\
    \hline
  \end{tblr}
\end{center}

\pagebreak
\section{Esperienza e procedimento di misura}

\begin{enumerate}
  \item[0.]
    Posizioniamo una giunzione della termocoppia (che d'ora in poi indicheremo
    come “giunzione fissa”) in un miscuglio %eterogeneo
    di acqua distillata (solida e liquida) alla temperatura
    costante di $(273.1\pm0.1)\,\unit{K}$.
  \item
    Per ogni punto fisso, individuiamo il voltaggio $\Delta V$ misurato dalla
    termocoppia, con la giunzione libera immersa nel campione,
    quando quest'ultimo effettua la transizione
    di fase. Tale fenomeno è individuabile nel grafico di $\Delta V$ in funzione
    del tempo in quanto si presenta come un plateau: la temperatura è infatti
    costante fino al termine della transizione di fase.
  \item
    Dopo ogni acquisizione, misuriamo la temperatura ambiente con il termometro
    ambientale, per assicurarci che non sia variata (al netto della sensibilità
    dello strumento). Per tutte le acquisizioni, abbiamo rilevato
    $(21.0\pm0.5)\,\unit{\degree C} = (294.1\pm0.5)\,\unit{K}$
\end{enumerate}
Di seguito indichiamo i passaggi necessari, caso per caso,
al raggiungimento dei diversi punti fissi, unitamente alle
rispettive temperature (note a priori).

\subsection*{Acqua (fusione) e azoto (ebollizione)}
\textbf{Temperature}: rispettivamente, $(273.1\pm0.1)\,\unit{K}$ e $(77.3\pm0.1)\,\unit{K}$
\vspace{1mm}

Data la considerevole quantità di ghiaccio e azoto liquido ed essendo
entrambe le temperature di transizione di fase minori della temperatura
ambiente, i passaggi di stato avvengono spontaneamente e per lungo tempo.

Questo ha permesso al gruppo di lavoro, in entrambi i casi,
di inserire direttamente la giunzione nella miscela tra le due fasi,
senza la necessità di svolgere passaggi ulteriori.

\subsection*{Acqua (ebollizione)}
\textbf{Temperatura}: $(373.1\pm0.1)\,\unit{K}$
\vspace{1mm}

L'unica differenza con il caso precedente è la spontaneità
della transizione di fase:
il gruppo di lavoro ha pertanto, preliminarmente, portato
a bollore una considerevole quantità d'acqua distillata,
scaldandola nel pentolino.

È stato poi sufficiente immergere la giunzione nell'acqua
in ebollizione.

\subsection*{Etanolo, indio e gallio (fusione)}
\textbf{Temperature}: rispettivamente, $(158.8\pm0.1)\,\unit{K}$,
$(302.9\pm0.1)\,\unit{K}$ e $(429.7\pm0.1)\,\unit{K}$
\vspace{1mm}

A differenza dei precedenti, in questi casi i campioni
hanno massa relativamente ridotta, per cui la transizione
di fase è breve. È necessario dunque svolgere i seguenti passaggi:
\begin{enumerate}
  \item
\end{enumerate}

\subsection{Indio}
Mettiamo il crogiolo di indio a scaldare a bagnomaria
nel pentolino; una volta fuso vi inseriamo la giunzione
ed lo facciamo raffreddare "naturalmente?", misurando
la ddp durante la solidificazione.

\subsection{Gallio}
Facciamo fondere il gallio nel pentolino, per poi
immergerlo nell'azoto liquido e scaldarlo nuovamente
nel pentolino, leggendo la ddp durante la fusione.
%facciamo questo perché:
% -la temperatura di fusione del Ga è intorno ai 30°C, dunque esso si trovava spesso in uno stato intermedio tra due fasi
% -bisogna evitare il fenomeno del sottoraffreddamento

\section{Analisi dei dati raccolti e conclusioni}
\subsection{Calcolo del momento d'inerzia del campione}

Essendo il momento d'inerzia additivo, abbiamo calcolato
$I_\text{CM}$ sommando i singoli momenti d'inerzia rispetto al comune
asse di simmetria dei cilindri e dei tronchi di cono che compongono il
campione, dove la massa di ciascuno di essi è stata facilmente
calcolata assumendo la densità del campione uniforme.
Di seguito riportiamo tali misure:

\begin{table}[H]
    \centering

    \begin{tblr}{
        vlines = {},
        hline{1,2,17} = {},
        hline{3,5-7,9,10,12-14,16} = {dashed},
        cell{1,2,5,6,9,12,13,16}{1-5} = {c,m},
        cell{3,7,10,14}{1-3,5} = {r=2}{c,m},
    }
        $\#$&\emph{Forma}&$h\;(\unit{mm})$&$d_{1,2}\;(\unit{mm})$&$I\;(10^{-6}\;\unit{kg\,m^2})$\\
        1  & Cilindro          & $30.45\pm0.05$ & $49.90\pm0.05$ & $154.6 \pm1.8 $ \\
        2  & {Tronco\\di cono} & $ 5.95\pm0.10$ & $49.90\pm0.05$ & $ 13.7 \pm0.5 $ \\
           &                   &                & $29.40\pm0.05$ &                 \\
        3  & Cilindro          & $ 9.20\pm0.10$ & $25.85\pm0.05$ & $  3.36\pm0.08$ \\
        4  & Cilindro          & $10.80\pm0.05$ & $18.65\pm0.05$ & $  1.07\pm0.02$ \\
        5  & {Tronco\\di cono} & $ 4.25\pm0.05$ & $34.55\pm0.05$ & $ 11.8 \pm0.4 $ \\
           &                   &                & $49.90\pm0.05$ &                 \\
        6  & Cilindro          & $52.95\pm0.05$ & $49.90\pm0.05$ & $269   \pm3   $ \\
        7  & {Tronco\\di cono} & $ 4.25\pm0.05$ & $49.90\pm0.05$ & $ 12.6 \pm0.4 $ \\
           &                   &                & $36.35\pm0.05$ &                 \\
        8  & Cilindro          & $10.80\pm0.05$ & $18.75\pm0.05$ & $  1.09\pm0.02$ \\
        9  & Cilindro          & $ 9.25\pm0.10$ & $25.90\pm0.05$ & $  3.41\pm0.08$ \\
        10 & {Tronco\\di cono} & $ 5.95\pm0.10$ & $29.10\pm0.05$ & $ 13.5 \pm0.5 $ \\
           &                   &                & $49.90\pm0.05$ &                 \\
        11 & Cilindro          & $30.40\pm0.05$ & $49.90\pm0.05$ & $154.4 \pm1.8 $ \\
    \end{tblr}
\end{table}

\begin{itemize}
    \item Massa totale: $M=(2214.57\pm0.01)\;\unit{g}$
    \item Volume totale: $V=(2.654\pm0.017)\cdot10^{-4}\;\unit{m^3}$
    \item Densità media: $\rho=(8.34\pm0.05)\cdot10^{-3}\;\unit{kg \per m^3}$
    \item Momento d'inerzia totale: $I_\text{CM}=(6.38\pm0.09)\cdot10^{-4}\;\unit{kg\,m^2}$
\end{itemize}

\subsection{Distribuzione dei tempi di caduta}

Riportiamo di seguito i grafici della distribuzione dei tempi di caduta $t_{L,\theta}$,
accompagnati alle relative misure di $L$ e $\theta$.

\begin{center}
    %
\end{center}

\subsection{Calcolo di $g$ mediante la dinamica del corpo rigido}

Fissato un sistema di riferimento cartesiano ortogonale solidale
al piano inclinato, con origine nel punto di partenza del campione,
asse $x$ parallelo alle guide e asse $y$ entrante nel piano inclinato,
possiamo scrivere la legge del moto del centro di massa e le
equazioni cardinali della dinamica del corpo rigido:
\[x_\text{CM}(t) = \frac{1}{2} a_\text{CM} t^2\]
\[\left\{\begin{aligned}
    &M g \sin\theta - F_s = M a_\text{CM} \\
    &M g \cos\theta - F_n = 0 \\
    &R M g \sin\theta = \left(I_\text{CM} + M R^2\right) \alpha
\end{aligned}\right.\]
dove $R$ è il raggio di contatto, $\vec{F}_s$ è la forza di attrito statico
tra il campione e le guide, mentre $\vec{F}_n$ è la reazione vincolare delle
guide, normale al piano.

Per poter descrivere il moto del campione come di rotolamento puro,
dobbiamo assicurarci che $F_s \le \mu_s F_n$, con $\mu_s$ il
coefficiente di attrito statico tra il corpo rigido e le guide.
Se questa condizione è verificata, possiamo utilizzare la relazione:
\[\alpha = \frac{a_\text{CM}}{R}\]

Risolvendo il sistema lineare e la disequazione di cui sopra si ottiene:
\[\left\{\begin{aligned}
    &a_\text{CM} = \frac{M R^2}{I_\text{CM} + M R^2} g\sin\theta \\
    &F_n = M g \cos\theta \\
    &F_s = \frac{I}{I + M R^2} M g \sin\theta \\
    & 0 \le \alpha \le \arctan\left(\mu_s \left(\frac{MR^2}{I_\text{CM}} + 1\right)\right) \\
\end{aligned}\right.\]

Ricordando ora che $L = x_\text{CM}(\bar{t}_{L,\theta}) + D + S$, dove $D$ è il diametro
più esterno del campione e $S$ è lo spessore del cuscinetto, possiamo ricavare:
\[
    \frac{2(L-D-S)}{\sin\theta}\left(\frac{I_\text{CM}}{M R^2} + 1\right) = g \cdot \bar{t}_{L,\theta}^2
\]

Possiamo pertanto determinare il modulo di $\vec{g}$ mediante una regressione
lineare pesata\footnote{
    La scelta di una regressione lineare \emph{pesata} è giustificata dal fatto
    che gli errori sull'ascissa, per quanto ridotti, sono diversi fra di loro.
}:
\begin{figure}[H]
    % <v>^
    % \includegraphics[trim={1cm 0.6cm 1cm 1cm},clip,width=\textwidth]{img/regressione.jpg}
    \caption*{\emph{
        In rosso la retta di regressione, in rosa la sua regione di incertezza. \\
        Nel grafico principale, le barre di errore lungo l'ascissa, date le loro
        dimensioni, non sono visibili.
    }}
\end{figure}

Di seguito riportiamo i risultati della regressione lineare:
\begin{itemize}
    \item Coefficiente angolare ($g$) $=(9.9\pm0.5)\;\unit{m \per s^2}$
    \item Intercetta $=(0\pm3)\;\unit{m}$ (compatibile con $0$, come ci si aspettava)
\end{itemize}

\end{document}
