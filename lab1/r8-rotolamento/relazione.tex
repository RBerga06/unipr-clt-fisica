\documentclass{article}
\usepackage[utf8]{inputenc}
\usepackage[italian]{babel}
\usepackage{amsmath}
\usepackage{amssymb}
\usepackage{siunitx}
\usepackage{tabularray}
\usepackage{graphicx}
\usepackage{float}
\usepackage{xfrac}
\usepackage{caption}    % for \caption*{}
\newcommand*{\diam}{\varnothing}
\newcommand*{\best}[1]{{#1}_\text{best}}
\newcommand*{\bestp}[1]{{\left(#1\right)}_\text{best}}
\newcommand*{\pbest}[1]{\left({#1}_\text{best}\right)}
\newcommand*{\pbestp}[1]{\left({\left(#1\right)}_\text{best}\right)}
\newcommand*{\errrel}[1]{\frac{\delta #1}{{#1}_\text{best}}}
%% <custom footnotes/>
%\newcounter{savefootnote}
%\newcounter{symfootnote}
%\newcommand{\symfootnote}[1]{%
%   \setcounter{savefootnote}{\value{footnote}}%
%   \setcounter{footnote}{\value{symfootnote}}%
%   \ifnum\value{footnote}>8\setcounter{footnote}{0}\fi%
%   \let\oldthefootnote=\thefootnote%
%   \renewcommand{\thefootnote}{\fnsymbol{footnote}}%
%   \footnote{#1}%
%   \let\thefootnote=\oldthefootnote%
%   \setcounter{symfootnote}{\value{footnote}}%
%   \setcounter{footnote}{\value{savefootnote}}%
%}
%% </custom footnotes>
\title{
    Laboratorio di Fisica 1\\
    R8: Misura di $\left|\vec{g}\right|$ mediante rotolamento puro
}
\author{Gruppo 17: Bergamaschi Riccardo, Graiani Elia, Moglia Simone}
\date{19/03/2024 – 9/04/2024}
\makeindex
\begin{document}

\maketitle

\begin{abstract}
    Il gruppo di lavoro ha misurato indirettamente il modulo del campo gravitazionale locale ($g$)
    studiando il moto di rotolamento di un corpo rigido.
\end{abstract}

\setcounter{section}{-1}  % Count sections starting from 0
\section{Materiali e strumenti di misura utilizzati}
\begin{center}
    \begin{tblr}{ |Q[l,m]|Q[c,m]|Q[c,m]|Q[c,m]| }
        \hline
        \textbf{Strumento di misura} & \textbf{\:\:\:\:\:Soglia\:\:\:\:\:} & \textbf{Portata} & \textbf{Sensibilità} \\
        \hline
        {Due fototraguardi con \\ contatore di impulsi} & \qty{1}{\micro s} & \qty{99999999}{\micro s} & \qty{1}{\micro s} \\
        \hline[dashed]
        Metro a nastro & \qty{0.1}{cm} & \qty{300.0}{cm} & \qty{0.1}{cm} \\
        \hline[dashed]
        Calibro ventesimale & \qty{0.05}{mm} & \qty{150.00}{mm} & \qty{0.05}{mm} \\
        \hline[dashed]
        Bilancia di precisione & \qty{0.01}{g} & \qty{4200.00}{g} & \qty{0.01}{g} \\
        \hline[dashed]
        Cellulare come goniometro & \qty{0.1}{\degree} & \qty{45.0}{\degree} & \qty{0.1}{\degree} \\
        \hline
        \hline
        \textbf{Altro} & \SetCell[c=3]{l} \textbf{Descrizione/Note} \\
        \hline
        Piano inclinato & \SetCell[c=3]{l} {
            Costituito da guide che permettono al \\
            campione di cadere da un fototraguardo \\
            all'altro con un moto di rotolamento puro.
        } \\
        \hline[dashed]
        Campione & \SetCell[c=3]{l} {
            Corpo rigido con simmetria assiale, \\
            assimilabile a una combinazione di \\
            cilindri e tronchi di cono coassiali.
        } \\
        \hline[dashed]
        Brugola e Lucidi & \SetCell[c=3]{l} {
            Utili per cambiare, rispettivamente, \\
            la distanza tra i fototraguardi e l'angolo \\
            di inclinazione delle guide.
        } \\
        \hline
    \end{tblr}
\end{center}

\section{Esperienza e procedimento di misura}
\begin{enumerate}
    \item
        Misuriamo la massa del campione con la bilancia di precisione
        e, con il calibro ventesimale, tutti i diametri e le altezze
        necessarie al calcolo del suo momento d'inerzia.
    \item
        Fissiamo la distanza $L$ tra i due fototraguardi
        e l'angolo $\theta$ di inclinazione delle guide
        rispetto a un piano normale a $\vec{g}$.
        Allora, acceso e impostato adeguatamente il contatore di impulsi,
        misuriamo 50 volte il tempo di caduta del campione.
    \item
        Ripetiamo il punto precedente per svariate combinazioni
        di $L$ e $\theta$.

\end{enumerate}

\emph{
    \textbf{Notazione.} Fissati $L$ e $\theta$,
    indicheremo con $\left(t_{L,\theta}\right)_i$
    ogni $i$-esima misura del tempo di caduta
    $\left(i\in\left[0;50\right)\cap\mathbb{N}\right)$,
    mentre con $\overline{t}_{L,\theta}$ il tempo di caduta medio.
    Calcoloremo l'errore su $\overline{t}_{L,\theta}$ in questo modo:\[
        \delta\!\left(\overline{t}_{L,\theta}\right) =
        \sigma_{\overline{t}_{L,\theta}} =
        \frac{\sigma_{t_{L,\theta}}}{\sqrt{50}}.
    \]
}

\subsection{Analisi dei dati raccolti e conclusioni}
Essendo il momento d'inerzia additivo, abbiamo calcolato
$I_\text{CM}$ sommando i singoli momenti d'inerzia rispetto al comune
asse di simmetria dei cilindri e dei tronchi di cono che compongono il
campione, dove la massa di ciascuno di essi è stata facilmente
calcolata assumendo la densità del campione uniforme.
Di seguito riportiamo tali misure:

\begin{itemize}
    \item Massa totale: $M=(2214.57\pm0.01)\;\unit{g}$
    \item Volume totale: $V=(2.654\pm0.017)\cdot10^{-4}\;\unit{m^3}$
    \item Densità media: $\rho=(8.34\pm0.05)\cdot10^{-3}\;\unit{kg \per m^3}$
\end{itemize}

\begin{table}[H]
    \centering

    \begin{tblr}{
        vlines = {},
        hline{1,2,17} = {},
        hline{3,5-7,9,10,12-14,16} = {dashed},
        cell{1,2,5,6,9,12,13,16}{1-5} = {c,m},
        cell{3,7,10,14}{1-3,5} = {r=2}{c,m},
    }
        $\#$&\emph{Forma}&$h\;(\unit{mm})$&$d_{1,2}\;(\unit{mm})$&$I\;(\unit{mg\,m^2})$\\
        1  & Cilindro          & $30.45\pm0.05$ & $49.90\pm0.05$ & $154.6 \pm1.8 $ \\
        2  & {Tronco\\di cono} & $ 5.95\pm0.10$ & $49.90\pm0.05$ & $ 13.7 \pm0.5 $ \\
           &                   &                & $29.40\pm0.05$ &                 \\
        3  & Cilindro          & $ 9.20\pm0.10$ & $25.85\pm0.05$ & $  3.36\pm0.08$ \\
        4  & Cilindro          & $10.80\pm0.05$ & $18.65\pm0.05$ & $  1.07\pm0.02$ \\
        5  & {Tronco\\di cono} & $ 4.25\pm0.05$ & $34.55\pm0.05$ & $ 11.8 \pm0.4 $ \\
           &                   &                & $49.90\pm0.05$ &                 \\
        6  & Cilindro          & $52.95\pm0.05$ & $49.90\pm0.05$ & $269   \pm3   $ \\
        7  & {Tronco\\di cono} & $ 4.25\pm0.05$ & $49.90\pm0.05$ & $ 12.6 \pm0.4 $ \\
           &                   &                & $36.35\pm0.05$ &                 \\
        8  & Cilindro          & $10.80\pm0.05$ & $18.75\pm0.05$ & $  1.09\pm0.02$ \\
        9  & Cilindro          & $ 9.25\pm0.10$ & $25.90\pm0.05$ & $  3.41\pm0.08$ \\
        10 & {Tronco\\di cono} & $ 5.95\pm0.10$ & $29.10\pm0.05$ & $ 13.5 \pm0.5 $ \\
           &                   &                & $49.90\pm0.05$ &                 \\
        11 & Cilindro          & $30.40\pm0.05$ & $49.90\pm0.05$ & $154.4 \pm1.8 $ \\
    \end{tblr}
\end{table}

Fissato un sistema di riferimento cartesiano ortogonale solidale
al piano inclinato, con origine nel punto di partenza del campione,
possiamo scrivere la legge del moto del centro di massa e le
equazioni cardinali del corpo rigido:
\[x = \frac{1}{2} a_\text{CM} t^2\]
Ma noi conosciamo anche la forza ed il momento risultanti sul
corpo\footnote{con $R$ indichiamo la distanza tra il suo centro di
massa e il punto di contatto}:
\[M g sin(\theta) - F_s = Ma_\text{cm}\]
\[F_s R = I_\text{cm} \frac{a_\text{cm}}{R}\]
\[M g R^2 sin(\theta) = (I_\text{cm} + MR^2) a_{cm}\]
La norma di $\vec{g}$ misurata indirettamente è allora ricavabile da:
\[\frac{2 L}{t^2} = \frac{M g R^2 \sin(\theta)}{I_\text{cm} + MR^2}\]
dove l'errore su $g$ segue dalla propagazione degli errori su $d_0,\diam_s$ e
$\overline{t_s}$ (supponendo gli errori piccoli, casuali e indipendenti). %SCRIVERE DA COSA SEGUE L'ERRORE

\end{document}
