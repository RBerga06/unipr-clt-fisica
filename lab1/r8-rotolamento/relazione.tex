\documentclass{article}
\usepackage[utf8]{inputenc}
\usepackage[italian]{babel}
\usepackage{amsmath}
\usepackage{amssymb}
\usepackage{siunitx}
\usepackage{tabularray}
\usepackage{graphicx}
\usepackage{float}
\usepackage{xfrac}
\newcommand*{\diam}{\varnothing}
\newcommand*{\best}[1]{{#1}_\text{best}}
\newcommand*{\bestp}[1]{{\left(#1\right)}_\text{best}}
\newcommand*{\pbest}[1]{\left({#1}_\text{best}\right)}
\newcommand*{\pbestp}[1]{\left({\left(#1\right)}_\text{best}\right)}
\newcommand*{\errrel}[1]{\frac{\delta #1}{{#1}_\text{best}}}
%% <custom footnotes/>
%\newcounter{savefootnote}
%\newcounter{symfootnote}
%\newcommand{\symfootnote}[1]{%
%   \setcounter{savefootnote}{\value{footnote}}%
%   \setcounter{footnote}{\value{symfootnote}}%
%   \ifnum\value{footnote}>8\setcounter{footnote}{0}\fi%
%   \let\oldthefootnote=\thefootnote%
%   \renewcommand{\thefootnote}{\fnsymbol{footnote}}%
%   \footnote{#1}%
%   \let\thefootnote=\oldthefootnote%
%   \setcounter{symfootnote}{\value{footnote}}%
%   \setcounter{footnote}{\value{savefootnote}}%
%}
%% </custom footnotes>
\title{
    Laboratorio di Fisica 1\\
    R8: Misura di $\left|\vec{g}\right|$ mediante rotolamento puro
}
\author{Gruppo 17: Bergamaschi Riccardo, Graiani Elia, Moglia Simone}
\date{19/03/2024 – 9/04/2024}
\makeindex
\begin{document}

\maketitle

\begin{abstract}
    Il gruppo di lavoro ha misurato indirettamente il modulo del campo gravitazionale locale ($g$)
    studiando il moto di rotolamento di un corpo rigido.
\end{abstract}

\setcounter{section}{-1}  % Count sections starting from 0
\section{Materiali e strumenti di misura utilizzati}
\begin{center}
    \begin{tblr}{ |Q[l,m]|Q[c,m]|Q[c,m]|Q[c,m]| }
        \hline
        \textbf{Strumento di misura} & \textbf{\:\:\:\:\:Soglia\:\:\:\:\:} & \textbf{Portata} & \textbf{Sensibilità} \\
        \hline
        {Due fototraguardi con \\ contatore di impulsi} & \qty{1}{\micro s} & \qty{99999999}{\micro s} & \qty{1}{\micro s} \\
        \hline[dashed]
        Metro a nastro & \qty{0.1}{cm} & \qty{300.0}{cm} & \qty{0.1}{cm} \\
        \hline[dashed]
        Calibro ventesimale & \qty{0.05}{mm} & \qty{150.00}{mm} & \qty{0.05}{mm} \\
        \hline[dashed]
        Bilancia di precisione & \qty{0.01}{g} & \qty{4200.00}{g} & \qty{0.01}{g} \\
        \hline[dashed]
        Cellulare come goniometro & \qty{0.1}{\degree} & \qty{45.0}{\degree} & \qty{0.1}{\degree} \\
        \hline
        \hline
        \textbf{Altro} & \SetCell[c=3]{l} \textbf{Descrizione/Note} \\
        \hline
        Piano inclinato & \SetCell[c=3]{l} {
            Costituito da guide che permettono al \\
            campione di cadere da un fototraguardo \\
            all'altro con un moto di rotolamento puro.
        } \\
        \hline[dashed]
        Campione & \SetCell[c=3]{l} {
            Corpo rigido con simmetria assiale, \\
            assimilabile a una combinazione di \\
            cilindri e tronchi di cono.
        } \\
        \hline[dashed]
        Brugola & \SetCell[c=3]{l} {
            Utile per cambiare la distanza \\
            tra i fototraguardi
        } \\
        \hline[dashed]
        Lucidi & \SetCell[c=3]{l} {
            Utili per cambiare l'angolo di \\
            inclinazione delle guide
        } \\
        \hline
    \end{tblr}
\end{center}

\section{Esperienza e procedimento di misura}
\begin{enumerate}
    \item
        Misuriamo la massa del campione con la bilancia di precisione
        e, con il calibro ventesimale, tutti i diametri e le altezze
        necessarie al calcolo del suo momento d'inerzia.
    \item
        Fissiamo la distanza $L$ tra i due fototraguardi
        e l'angolo $\theta$ di inclinazione delle guide
        rispetto a un piano normale a $\vec{g}$.
        Allora, acceso e impostato adeguatamente il contatore di impulsi,
        misuriamo 50 volte il tempo di caduta del campione.
    \item
        Ripetiamo il punto precedente per svariate combinazioni
        di $L$ e $\theta$.

\end{enumerate}

\emph{
    \textbf{Notazione.} Indicheremo con $\left(t_s\right)_i$
    ogni $i$-esima misura del tempo di caduta
    $\left(i\in\left[0;50\right)\cap\mathbb{N}\right)$,
    mentre con $\overline{t_s}$ il tempo di caduta medio.
    In particolare:\[
        \delta\!\left(\overline{t_s}\right) = \sigma_{\overline{t_s}} =
        \frac{\sigma_{t_s}}{\sqrt{100}}   = \frac{\sigma_{t_s}}{10}.
    \]
}

\subsection{Analisi dei dati raccolti e conclusioni}
Essendo il momento d'inerzia additivo, abbiamo calcolato
$I_\text{CM}$ sommando i momenti d'inerzia rispetto all'asse
di simmetria

Fissato un sistema di riferimento solidale all'apparato di misura,
con origine nel punto di partenza del campione, possiamo scrivere
la legge del moto del campione, indicando con $l$ la distanza tra
i due fototraguardi:
\[l = \frac{1}{2} a_\text{cm} t^2\]
Ma noi conosciamo anche la forza ed il momento risultanti sul
corpo\footnote{con $R$ indichiamo la distanza tra il suo centro di
massa e il punto di contatto}:
\[M g sin(\theta) - F_s = Ma_\text{cm}\]
\[F_s R = I_\text{cm} \frac{a_\text{cm}}{R}\]
\[M g R^2 sin(\theta) = (I_\text{cm} + MR^2) a_{cm}\]
La norma di $\vec{g}$ misurata indirettamente è allora ricavabile da:
\[\frac{2 l}{t^2} = \frac{M g R^2 sin(\theta)}{I_\text{cm} + MR^2}\]
dove l'errore su $g$ segue dalla propagazione degli errori su $d_0,\diam_s$ e
$\overline{t_s}$ (supponendo gli errori piccoli, casuali e indipendenti). %SCRIVERE DA COSA SEGUE L'ERRORE


Di seguito riportiamo gli istogrammi dei tempi e i valori di $g$ così ottenuti.
Per confrontare queste misure indirette ($g_A$ e $g_B$) con il valore atteso
($g_\text{attesa}=\qty{9.806}{m\per s^2}$), valutiamo, per ogni sferetta $s$, la seguente quantità
(adimensionale):\[\varepsilon_s = \frac{\bestp{g_s} - g_\text{attesa}}{\delta g_s}
\] Allora la misura $g_s$ da noi ottenuta è compatibile con $g_\text{attesa}$ se e solo se
$\left|\varepsilon_s\right|\le1$; inoltre, dal segno di $\varepsilon_s$ è possibile
determinare se $g_s$ misurata è una sovrastima ($\varepsilon_s>0$) o una sottostima
($\varepsilon_s<0$) del valore atteso.

\begin{figure}[H]
    % trim={< v > ^}
    %\includegraphics[trim={2cm .5cm 2.4cm 2.1cm},clip,width=.5\textwidth]{img/TempiA.jpg}
    %\includegraphics[trim={2cm .5cm 2.4cm 2.1cm},clip,width=.5\textwidth]{img/TempiB.jpg}
    \caption{Istogrammi dei dati raccolti ($t_A$ e $t_B$)}
\end{figure}
\vspace{.5cm}
\begin{center}
    \begin{tblr}{ |Q[c,m]|Q[c,m]|Q[c,m]|Q[c,m]|Q[c,m]| }
        \hline
            $s$ &
            $\diam_s\:(\unit{mm})$ &
            $\overline{t_s}\:(\unit{ms})$ &
            $g_s\:(\unit{m\per s^2})$ &
            $\varepsilon_s$ \\
        \hline
        A & $24.62\pm0.01$ & $503.63\pm0.02$ & $9.82\pm0.02$ & $+0.67$ \\
        \hline[dashed]
        B & $22.23\pm0.01$ & $503.93\pm0.02$ & $9.82\pm0.02$ & $+0.57$ \\
        \hline
    \end{tblr}
\end{center}

Le misure di $g$ ottenute sono pertanto ampiamente consistenti con il valore atteso.

\emph{
    \textbf{Osservazione.} Dai valori di $\varepsilon$ non emerge una differenza
    significativa fra le due sferette. In particolare, sembra che l'attrito viscoso
    dell'aria abbia agito in maniera trascurabile (come ci aspettavamo).\\
    Tuttavia, dopo una più attenta analisi, è comunque possibile notare una tendenza:
    in media, la sferetta con raggio maggiore ha percorso la stessa distanza in un
    tempo leggermente minore.
    Pertanto, ciò potrebbe suggerire un effetto molto ridotto dell'attrito dell'aria;
    questa tendenza però non è rispecchiata dai valore di $\varepsilon$,
    probabilmente a causa di una sovrastima della distanza $d_0$. Si noti infatti che,
    dal segno degli $\varepsilon$, entrambe le misure di $g$ sono risultate sovrastime,
    mentre, nell'equazione (\ref{eq:1}), $d_0$ è al numeratore.
}
\end{document}
