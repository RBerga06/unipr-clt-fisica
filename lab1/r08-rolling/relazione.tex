\documentclass{article}
\usepackage[utf8]{inputenc}
\usepackage[italian]{babel}
\usepackage{amsmath}
\usepackage{amssymb}
\usepackage{siunitx}
\usepackage{tabularray}
\usepackage{graphicx}
\usepackage{float}
\usepackage{xfrac}
\usepackage{caption}    % for \caption*{}
\usepackage[bottom]{footmisc}  % forza le note a pié di pagina in basso
\usepackage[labelformat=simple, justification=centering]{subfig}
\renewcommand{\thesubfigure}{}
\newcommand*{\diam}{\varnothing}
\newcommand*{\best}[1]{{#1}_\text{best}}
\newcommand*{\bestp}[1]{{\left(#1\right)}_\text{best}}
\newcommand*{\pbest}[1]{\left({#1}_\text{best}\right)}
\newcommand*{\pbestp}[1]{\left({\left(#1\right)}_\text{best}\right)}
\newcommand*{\errrel}[1]{\frac{\delta #1}{{#1}_\text{best}}}
%% <custom footnotes/>
%\newcounter{savefootnote}
%\newcounter{symfootnote}
%\newcommand{\symfootnote}[1]{%
%   \setcounter{savefootnote}{\value{footnote}}%
%   \setcounter{footnote}{\value{symfootnote}}%
%   \ifnum\value{footnote}>8\setcounter{footnote}{0}\fi%
%   \let\oldthefootnote=\thefootnote%
%   \renewcommand{\thefootnote}{\fnsymbol{footnote}}%
%   \footnote{#1}%
%   \let\thefootnote=\oldthefootnote%
%   \setcounter{symfootnote}{\value{footnote}}%
%   \setcounter{footnote}{\value{savefootnote}}%
%}
%% </custom footnotes>
\title{
    Laboratorio di Fisica 1\\
    R8: Misura di $\left|\vec{g}\right|$ mediante rotolamento puro
}
\author{Gruppo 15: Bergamaschi Riccardo, Graiani Elia, Moglia Simone}
\date{19/03/2024 – 9/04/2024}
\makeindex
\begin{document}

\maketitle

\begin{abstract}
    Il gruppo di lavoro ha misurato indirettamente il modulo del campo gravitazionale locale ($g$)
    studiando il moto di rotolamento di un corpo rigido.
\end{abstract}

\setcounter{section}{-1}  % Count sections starting from 0
\section{Materiali e strumenti di misura utilizzati}
\begin{center}
    \begin{tblr}{
        width=\textwidth,
        colspec={ X[2,m,j]X[m,c]X[m,c]X[m,c] },
        vlines,
    }
        \hline
        \textbf{Strumento di misura} & \textbf{Soglia} & \textbf{Portata} & \textbf{Sensibilità} \\
        \hline
        {Sistema a contatti elettrici con contatore di impulsi} & \qty{1}{\micro s} & \qty{99999999}{\micro s} & \qty{1}{\micro s} \\
        \hline[dashed]
        Metro a nastro & \qty{0.1}{cm} & \qty{300.0}{cm} & \qty{0.1}{cm} \\
        \hline[dashed]
        Calibro ventesimale & \qty{0.05}{mm} & \qty{150.00}{mm} & \qty{0.05}{mm} \\
        \hline[dashed]
        Bilancia di precisione & \qty{0.01}{g} & \qty{4200.00}{g} & \qty{0.01}{g} \\
        \hline[dashed]
        Cellulare come goniometro & \qty{0.1}{\degree} & \qty{45.0}{\degree} & \qty{0.1}{\degree} \\
        \hline
    \end{tblr}
    \begin{tblr}{
        width=\textwidth,
        colspec={ X[m,j]X[3,m,j] },
        vlines,
    }
        \hline
        \textbf{Altro} & \textbf{Descrizione/Note} \\
        \hline
        Piano inclinato & {
            Costituito da guide che permettono al
            campione di cadere da un contatto elettrico
            all'altro con un moto di rotolamento puro.
        } \\
        \hline[dashed]
        Campione & {
            Corpo rigido con simmetria assiale,
            assimilabile a una combinazione di
            cilindri e tronchi di cono coassiali.
        } \\
        \hline[dashed]
        Cuscinetto & {
            Posto a coprire il secondo contatto
            elettrico, attutisce l'impatto del campione
            contro di esso.
        } \\
        \hline[dashed]
        Brugola e lucidi & {
            Utili per cambiare, rispettivamente,
            la distanza tra i contatti e l'angolo
            di inclinazione delle guide.
        } \\
        \hline
    \end{tblr}
\end{center}

\section{Esperienza e procedimento di misura}
\begin{enumerate}
    \item
        Misuriamo la massa del campione con la bilancia di precisione
        e, con il calibro ventesimale, tutti i diametri e le altezze
        necessarie al calcolo del suo momento d'inerzia.
    \item
        Fissiamo la distanza $L$ tra i due contatti elettrici
        e l'angolo $\theta$ di inclinazione delle guide
        rispetto a un piano normale a $\vec{g}$.
        Allora, acceso e impostato adeguatamente il contatore
        di impulsi, misuriamo 50 volte il tempo di caduta del
        campione $t_{L,\theta}$.
    \item
        Ripetiamo il punto precedente per svariate combinazioni
        di $L$ e $\theta$.

\end{enumerate}

\section{Analisi dei dati raccolti}

\emph{
    \textbf{Nota.} In tutta questa sezione, le incertezze su tutte le misure
    indirette sono state calcolate mediante l'usuale propagazione degli
    errori.
}

\subsection{Calcolo del momento d'inerzia del campione}

Essendo additivo, abbiamo calcolato $I_\text{CM}$ sommando
i singoli momenti d'inerzia\footnotemark[1] (rispetto al comune
asse di simmetria) dei segmenti % cilindri e dei tronchi di cono
che compongono il campione,
la cui massa % dove la massa di ciascuno di essi
è stata facilmente calcolata assumendo $\rho$ uniforme:

% Qui riportiamo i risultati delle misure di cui sopra (dirette e indirette):

\begin{table}[H]
    \centering

    \begin{tblr}{
        vlines = {},
        hline{1,2,17} = {},
        hline{3,5-7,9,10,12-14,16} = {dashed},
        cell{1,2,5,6,9,12,13,16}{1-5} = {c,m},
        cell{3,7,10,14}{1-3,5} = {r=2}{c,m},
    }
        $\#$&\emph{Forma}&$h\;(\unit{mm})$&$d_{1,2}\;(\unit{mm})$&$I\;(10^{-6}\;\unit{kg\,m^2})$\\
        1  & Cilindro          & $30.45\pm0.05$ & $49.90\pm0.05$ & $154.6 \pm1.8 $ \\
        2  & {Tronco\\di cono} & $ 5.95\pm0.10$ & $49.90\pm0.05$ & $ 13.7 \pm0.5 $ \\
           &                   &                & $29.40\pm0.05$ &                 \\
        3  & Cilindro          & $ 9.20\pm0.10$ & $25.85\pm0.05$ & $  3.36\pm0.08$ \\
        4  & Cilindro          & $10.80\pm0.05$ & $18.65\pm0.05$ & $  1.07\pm0.02$ \\
        5  & {Tronco\\di cono} & $ 4.25\pm0.05$ & $34.55\pm0.05$ & $ 11.8 \pm0.4 $ \\
           &                   &                & $49.90\pm0.05$ &                 \\
        6  & Cilindro          & $52.95\pm0.05$ & $49.90\pm0.05$ & $269   \pm3   $ \\
        7  & {Tronco\\di cono} & $ 4.25\pm0.05$ & $49.90\pm0.05$ & $ 12.6 \pm0.4 $ \\
           &                   &                & $36.35\pm0.05$ &                 \\
        8  & Cilindro          & $10.80\pm0.05$ & $18.75\pm0.05$ & $  1.09\pm0.02$ \\
        9  & Cilindro          & $ 9.25\pm0.10$ & $25.90\pm0.05$ & $  3.41\pm0.08$ \\
        10 & {Tronco\\di cono} & $ 5.95\pm0.10$ & $29.10\pm0.05$ & $ 13.5 \pm0.5 $ \\
           &                   &                & $49.90\pm0.05$ &                 \\
        11 & Cilindro          & $30.40\pm0.05$ & $49.90\pm0.05$ & $154.4 \pm1.8 $ \\
    \end{tblr}
\end{table}

\begin{itemize}
    \item Massa totale (misurata direttamente): $M=(2214.57\pm0.01)\;\unit{g}$
    \item Volume totale: $V=(2.654\pm0.017)\cdot10^{-4}\;\unit{m^3}$
    \item Densità media: $\rho=(8.34\pm0.05)\cdot10^{-3}\;\unit{kg \per m^3}$
    \item Momento d'inerzia totale: $I_\text{CM}=(6.38\pm0.09)\cdot10^{-4}\;\unit{kg\,m^2}$
\end{itemize}

\footnotetext[1]{
    Per calcolare i volumi e i momenti d'inerzia abbiamo utilizzato le seguenti formule
    (dove $r_{1,2} = \frac{1}{2} d_{1,2}$ sono i rispettivi raggi):
    \[\begin{aligned}
        &V_\text{cilindro} = \pi h r^2 \qquad
        &V_\text{tronco di cono} = \frac{\pi}{3} h \frac{r_1^3 - r_2^3}{r_1 - r_2} \\
        &I_\text{cilindro} = \frac{\pi}{2} \rho h r^4 \qquad
        &I_\text{tronco di cono} = \frac{\pi}{10} \rho h \frac{r_1^5 - r_2^5}{r_1 - r_2}
    \end{aligned}\]
}

\subsection{Distribuzione dei tempi di caduta}

Questi sono i grafici della distribuzione dei tempi di caduta $t_{L,\theta}$,
accompagnati alle relative misure di $L$ e $\theta$.

\begin{figure}[H]
    \centering
    \subfloat[][
        $L=(55.6\pm0.1)\;\unit{cm}$

        $\theta=(3.8\pm0.1)\unit{\degree}$
    ]{\includegraphics[trim={2.1cm 0.7cm 2.1cm 2cm},clip,width=0.47\textwidth]{img/G0.jpg}}\hfil
    \subfloat[][
        $L=(70.5\pm0.1)\;\unit{cm}$

        $\theta=(3.8\pm0.1)\unit{\degree}$
    ]{\includegraphics[trim={2.1cm 0.7cm 2.1cm 2cm},clip,width=0.47\textwidth]{img/G1.jpg}}\hfil
    \subfloat[][
        $L=(85.4\pm0.1)\;\unit{cm}$

        $\theta=(3.8\pm0.1)\unit{\degree}$
    ]{\includegraphics[trim={2.1cm 0.7cm 2.1cm 2cm},clip,width=0.47\textwidth]{img/G2.jpg}}\hfil
    \subfloat[][
        $L=(100.2\pm0.1)\;\unit{cm}$

        $\theta=(3.8\pm0.1)\unit{\degree}$
    ]{\includegraphics[trim={2.1cm 0.7cm 2.1cm 2cm},clip,width=0.47\textwidth]{img/G3.jpg}}\hfil
\end{figure}\begin{figure}[H]
    \centering
    \subfloat[][
        $L=(70.5\pm0.1)\;\unit{cm}$

        $\theta=(2.9\pm0.1)\unit{\degree}$
    ]{\includegraphics[trim={2.1cm 0.7cm 2.1cm 2cm},clip,width=0.47\textwidth]{img/G4.jpg}}\hfil
    \subfloat[][
        $L=(85.4\pm0.1)\;\unit{cm}$

        $\theta=(2.9\pm0.1)\unit{\degree}$
    ]{\includegraphics[trim={2.1cm 0.7cm 2.1cm 2cm},clip,width=0.47\textwidth]{img/G5.jpg}}
    \subfloat[][
        $L=(100.2\pm0.1)\;\unit{cm}$

        $\theta=(2.9\pm0.1)\unit{\degree}$
    ]{\includegraphics[trim={2.1cm 0.7cm 2.1cm 2cm},clip,width=0.47\textwidth]{img/G6.jpg}}\hfil
    \subfloat[][
        $L=(55.6\pm0.1)\;\unit{cm}$

        $\theta=(2.1\pm0.1)\unit{\degree}$
    ]{\includegraphics[trim={2.1cm 0.7cm 2.1cm 2cm},clip,width=0.47\textwidth]{img/G7.jpg}}\hfil
    \subfloat[][
        $L=(85.4\pm0.1)\;\unit{cm}$

        $\theta=(2.1\pm0.1)\unit{\degree}$
    ]{\includegraphics[trim={2.1cm 0.7cm 2.1cm 2cm},clip,width=0.47\textwidth]{img/G8.jpg}}\hfil
\end{figure}

Come è possibile osservare da questi grafici, le distribuzioni dei tempi di caduta
sono abbastanza assimilabili a distribuzioni gaussiane.
Questo ci permette di utilizzare nei calcoli successivi la media aritmetica $\bar{t}_{L,\theta}$
di ciascun set di dati, indicando come errore:
\[\sigma_{\bar{t}_{L,\theta}} = \frac{\sigma_{t_{L,\theta}}}{\sqrt{50}}\]

\emph{\textbf{Nota.}
    Si può osservare come le distribuzioni dei tempi di caduta si
    presentino come un po' asimmetriche: in particolare, la moda è
    spesso “spostata” verso tempi più bassi. Secondo il gruppo di lavoro,
    questo riflette il fatto che, se il corpo rigido viene lasciato cadere
    quando non è perfettamente allineato con le guide, può scivolare,
    allungando il tempo di caduta.
}

\subsection{Calcolo di $g$ mediante la dinamica del corpo rigido}

Fissato un sistema di riferimento cartesiano ortogonale solidale
al piano inclinato, con origine nel punto di partenza del campione,
asse $x$ parallelo alle guide e asse $y$ entrante nel piano inclinato,
possiamo scrivere la legge del moto del centro di massa e le
equazioni cardinali della dinamica del corpo rigido:
\[x_\text{CM}(t) = \frac{1}{2} a_\text{CM} t^2\]
\[\left\{\begin{aligned}
    &M g \sin\theta - F_s = M a_\text{CM} \\
    &M g \cos\theta - F_n = 0 \\
    &R M g \sin\theta = \left(I_\text{CM} + M R^2\right) \alpha
\end{aligned}\right.\]
dove $\alpha$ è l'accelerazione angolare,
$R = \frac{1}{2}(25.88\pm0.05)\;\unit{mm} = (12.94\pm0.03)\;\unit{mm}$
è il raggio di contatto, $\vec{F}_s$ è la forza di attrito statico
tra il campione e le guide e $\vec{F}_n$ è la reazione vincolare delle
guide, normale al piano.

Per poter descrivere il moto del campione come di rotolamento puro,
dobbiamo assicurarci che $F_s \le \mu_s F_n$, con $\mu_s$ il
coefficiente di attrito statico tra il corpo rigido e le guide.
Se questa condizione è verificata, possiamo utilizzare la relazione:
\[\alpha = \frac{a_\text{CM}}{R}\]

Risolvendo il sistema lineare e la disequazione di cui sopra si ottiene:
\[\left\{\begin{aligned}
    &a_\text{CM} = \frac{M R^2}{I_\text{CM} + M R^2} g\sin\theta \\
    &F_n = M g \cos\theta \\
    &F_s = \frac{I}{I + M R^2} M g \sin\theta \\
    & 0 \le \theta \le \arctan\left(\mu_s \left(\frac{MR^2}{I_\text{CM}} + 1\right)\right) \\
\end{aligned}\right.\]

\emph{\textbf{Osservazione.}
    Utilizzando il coefficiente di attrito statico (tra ottone e acciaio,
    i materiali, rispettivamente, del corpo rigido e delle guide)
    $\mu_s = 0.51\pm0.01$ riportato in letteratura,
    la disequazione diventa $0 \le \theta \le (38.8 \pm 0.7)\unit{\degree}$.
    È immediato notare che tutti i valori di $\theta$ che abbiamo utilizzato
    soddisfano questa condizione, permettendoci di applicare questo
    modello fisico al moto del corpo.
}

\pagebreak

Ricordando ora che $L = x_\text{CM}(\bar{t}_{L,\theta}) + D + S$,
dove $D = (49.90\pm0.05)\;\unit{mm}$ è il diametro più esterno del campione e
$S = (0.6\pm0.1)\;\unit{cm}$ è lo spessore del cuscinetto, possiamo ricavare:
\[
    \frac{2(L-D-S)}{\sin\theta}\left(\frac{I_\text{CM}}{M R^2} + 1\right) = g \cdot \bar{t}_{L,\theta}^2
\]

Possiamo pertanto determinare il modulo di $\vec{g}$ mediante una regressione
lineare pesata\footnote[2]{
    La scelta di una regressione lineare \emph{pesata} è giustificata dal fatto
    che gli errori sull'ascissa, per quanto ridotti, sono diversi fra di loro.
}:
\begin{figure}[H]
    % <v>^
    \includegraphics[trim={1cm 0.6cm 1cm 1cm},clip,width=\textwidth]{img/regressione.jpg}
    \caption*{\emph{
        In rosso la retta di regressione, in rosa la sua regione di incertezza. \\
        Le barre di errore lungo l'ascissa, date le loro dimensioni, non sono visibili.
    }}
\end{figure}

Di seguito riportiamo i risultati della regressione lineare:
\begin{itemize}
    \item Intercetta $=(0\pm3)\;\unit{m}$
    \item Coefficiente angolare ($g$) $=(9.9\pm0.5)\;\unit{m \per s^2}$
\end{itemize}

\pagebreak

\section{Conclusioni}
I risultati della regressione lineare sono chiaramente compatibili
con i valori attesi. Infatti:
\begin{itemize}
    \item Secondo il modello fisico utilizzato, l'intercetta dovrebbe essere nulla;
          in effetti, $(0\pm3)\;\unit{m}$ è compatibile con $\qty{0}{m}$.
    \item Il valore di $g$ atteso è $\qty{9.806}{m\per s^2}$; si può osservare
          facilmente che il valore misurato, $(9.9\pm0.5)\;\unit{m \per s^2}$, è
          compatibile con esso.
\end{itemize}

Possiamo pertanto concludere che l'esperienza ha avuto successo: mediante l'apparato
sperimentale abbiamo ottenuto una misura di $g$ compatibile con quella attesa.

\end{document}
