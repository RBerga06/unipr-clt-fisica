\documentclass{article}
\usepackage[utf8]{inputenc}
\usepackage[italian]{babel}
\usepackage{amsmath}
\usepackage{amssymb}
\usepackage{siunitx}
\usepackage{tabularray}
\usepackage{graphicx}
\usepackage{float}
% \usepackage{minted}
\usepackage[bottom]{footmisc}
\usepackage[page]{appendix}
\newcommand*{\diam}{\varnothing}
\newcommand*{\best}[1]{{#1}_\text{best}}
\newcommand*{\bestp}[1]{{\left(#1\right)}_\text{best}}
\newcommand*{\pbest}[1]{\left({#1}_\text{best}\right)}
\newcommand*{\pbestp}[1]{\left({\left(#1\right)}_\text{best}\right)}
\newcommand*{\errrel}[1]{\frac{\delta #1}{{#1}_\text{best}}}
\title{
  Laboratorio di Fisica 1\\
  R7: Misura di $\left|\vec{g}\right|$ mediante pendolo fisico
}
\author{Gruppo 15: Bergamaschi Riccardo, Graiani Elia, Moglia Simone}
\date{05/03/2024 – 12/03/2024}
\makeindex
\begin{document}

\maketitle

\begin{abstract}
  Il gruppo di lavoro ha misurato il modulo del campo gravitazionale locale
  ($g$) studiando il moto oscillatorio di un pendolo fisico.
\end{abstract}

\section{Materiali e strumenti di misura utilizzati}
\begin{center}
  \begin{tblr}{ |Q[l,m]|Q[c,m]|Q[c,m]|Q[c,m]| }
    \hline
    \textbf{Strumento di misura} & \textbf{\:\:\:\:\:Soglia\:\:\:\:\:} & \textbf{Portata} & \textbf{Sensibilità} \\
    \hline
    Sensore di rotazione & $\qty{0.002}{rad}$ & N./A. & $\qty{0.002}{rad}$ \\
    \hline[dashed]
    Cronometro & $\qty{0.001}{s}$ & N./A. & $\qty{0.001}{s}$ \\
    % TODO: Se nel rotolamento abbiamo messo che il “goniometro” era nel cellulare,
    %   qua il cronometro è nel computer: ha senso indicarlo?
    \hline[dashed]
    Micrometro ad asta filettata & $\qty{0.01}{mm}$ & $\qty{25.00}{mm}$ & $\qty{0.01}{mm}$ \\
    \hline[dashed]
    Calibro ventesimale & $\qty{0.05}{mm}$ & $\qty{150.00}{mm}$ & $\qty{0.05}{mm}$ \\
    \hline[dashed]
    Metro & $\qty{0.1}{cm}$ & $\qty{300.0}{cm}$ & $\qty{0.1}{cm}$ \\
    \hline[dashed]
    Bilancia di precisione & $\qty{0.01}{g}$ & $\qty{6200.00}{g}$ & $\qty{0.01}{g}$ \\
    \hline
    \hline
    \textbf{Altro} & \SetCell[c=3]{l} \textbf{Descrizione/Note} \\
    \hline
    {Asta e rotore} & \SetCell[c=3]{l} {
      L'asta, fissata ortogonalmente al rotore, \\
      è libera di ruotare attorno ad un suo estremo. \\
      Il rotore è invece innestato del sensore di rotazione.
      % TODO: è parte del sensore di rotazione?
    } \\
    \hline[dashed]
    {Tre cilindri \\ (con masse e raggi distinti)} & \SetCell[c=3]{l} {
      Presentano un foro centrale lungo l'asse di \\
      simmetria. Li indicheremo con $A$, $B$ e $C$.
      % Indicheremo con $A,B,C$ i tre cilindri \\
      % e con $0$, $A+B$, $A+C$, $B+C$ e $A+B+C$ \\
      % le loro combinazioni.
      } \\
    \hline
  \end{tblr}
\end{center}

\section{Esperienza e procedimento di misura}

\emph{
  \textbf{Definizione.} Con il termine “\emph{configurazione}”
  indicheremo d'ora in poi l'insieme delle posizioni di una
  qualunque combinazione di cilindri lungo l'asta, misurate
  rispetto all'estremo fissato al rotore.
}

\begin{enumerate}
  \item
      Misuriamo le masse dei cilindri con la bilancia di precisione,
      i rispettivi diametri (interni ed esterni) con il calibro ventesimale
      e le altezze con il micrometro ad asta filettata.
  \item
      Mediante il metro a nastro misuriamo la lunghezza dell'asta e,
      servendoci del micrometro, i diametri di asta e rotore.
  \item
      % TODO: 12?
      Ripetiamo 12 volte i seguenti passi:
  \begin{enumerate}
    \item
      Scelta arbitrariamente una configurazione,
      fissiamo all'asta i cilindri coinvolti,
      servendoci del foro centrale.
      Misuriamo poi, mediante il metro a nastro,
      le posizioni dei cilindri lungo l'asta
      rispetto al suo estremo libero.
    \item
      Servendoci dell'apposito programma, avviamo
      l'acquisizione dell'angolo in funzione del tempo
      ($\theta(t)$, lo definiremo formalmente più avanti).
    \item
      Inclinando l'asta rispetto alla sua posizione di equilibrio
      di un angolo prefissato $\theta_0$,
      sufficientemente piccolo\footnote{
Questa condizione sull'angolo $\theta_0$ ci permette di approssimare
$\sin(\theta) \sim \theta\quad\forall \theta \in [-\theta_0, \theta_0]$.
      }, diamo inizio al moto del pendolo.
      Acquisiamo dati fino all'arresto del moto.\footnote{
Abbiamo operato questa scelta perché, come verrà spiegato
successivamente, l'angolo misurato dallo strumento quando
il moto ha ampiezza minore della sensibilità dello strumento
offre al gruppo di lavoro informazioni importanti relativamente
alla bontà della misura.
      }
  \end{enumerate}
\end{enumerate}

\section{Analisi dei dati raccolti e conclusioni}
\emph{\textbf{Nota.}
Avendo valutato gli errori sulle grandezze misurate direttamente
come piccoli, casuali e indipendenti, per svolgere ogni calcolo
abbiamo utilizzato la tradizionale propagazione degli errori.
}

\subsection{Misura di $\left|\vec{g}\right|$}

% TODO: cambiamo in SdR polare piano, dove D diventa r_\text{CM}
%   e \vec{D} diventa (r_\text{CM}; \theta).
%   Questo semplificherebbe un pochino la notazione.
Scegliamo un sistema di riferimento cartesiano ortogonale fisso, con
versore $\hat{\imath}$ giacente sul piano di oscillazione del centro di
massa del pendolo, versore $\hat{\jmath}$ parallelo a $\vec{g}$ e
versore $\hat{k}$ diretto lungo l'asse di rotazione del sistema.

Allora, detto $\theta$ lo spostamento angolare del centro di massa
rispetto alla sua posizione di equilibrio,
vale la seconda equazione cardinale della dinamica:
\[\sum\tau^\text{ext}_z = \frac{d}{dt} L_z = I_z^\text{tot} \ddot{\theta}\]

\vspace{2mm}
\emph{
  \textbf{Nota.} In questa sezione abbiamo trascurato la presenza di
  attriti, ma chiaramente gli attriti ci sono e il moto è smorzato.
  Nella sezione successiva tratteremo proprio questo fenomeno,
  determinando, alla luce dei dati raccolti, quanto influisca
  sul valore di $g$.
}
\vspace{2mm}

Poiché l'unica forza esterna al sistema che compie un momento lungo
$\hat{k}$ è la forza peso, si ha:
\[
  \sum\vec{\tau}^\text{\,ext}_z =
  \vec{D}\times M\vec{g} = -MDg\sin(\theta)\hat{k}.
\]
dove $\vec{D}$ è la posizione del centro di massa rispetto
all'asse di rotazione.

L'equazione differenziale che descrive il moto del centro di massa
del pendolo fisico sarà allora:
\[ \ddot{\theta} = -\frac{MDg}{I_z^\text{tot}}\sin(\theta) \]

\subsubsection{L'approssimazione $\sin(\theta)\simeq\theta$}

Poiché $\sin(x) \sim x$ per $x\rightarrow 0$, come accennato in
precedenza possiamo semplificare il modello fisico approssimando
$\sin(\theta)$ a $\theta$. Questa operazione è valida solo
se $\theta-\sin(\theta) < \delta\theta$, ma, essendo chiaramente\footnote{
  Lo si può osservare facilmente mediante la conservazione dell'energia
  meccanica totale: se per $t=0$ si ha $\theta=\theta_0$ ed
  $E(0)=U_g=-MgD\cos(\theta_0)$, in ogni altro istante di tempo si avrà
  $E(t)=K-MgD\cos(\theta)\le E(0)=-MgD\cos(\theta_0)$, da cui
  $\cos(\theta)\ge\cos(\theta_0)$ ovvero $0\le\theta\le\theta_0$
  (poiché $0\le\theta<\pi$, intervallo in cui la funzione
  coseno è strettamente decrescente).
} $\theta(t)\le\theta_0\quad\forall t\in[0,+\infty)$,
la condizione è soddisfatta per ogni $\theta$ se è soddisfatta per
$\theta_0$.

Essendo, nel nostro caso, $\delta\theta=\qty{0.02}{rad}$, possiamo
utilizzare $\theta_0^\text{max} = \qty{0.49}{rad}$, in quanto:
\[
  \qty{0.49}{rad} - \sin(\qty{0.49}{rad}) \simeq \qty{0.019}{rad}
  \qquad
  \qty{0.50}{rad} - \sin(\qty{0.50}{rad}) \simeq \qty{0.021}{rad}
\]

In conclusione, l'approssimazione $\sin(\theta)\simeq\theta$ può essere
utilizzata \emph{sic} nel modello fisico con la cautela di verificare
che $\theta_0\le\qty{0.49}{rad}$. Posta dunque questa condizione,
possiamo risolvere:
\[ \ddot{\theta} = -\frac{MDg}{I_z^\text{tot}} \theta \]
Questa equazione descrive un moto armonico. Le soluzioni sono infatti
del tipo:
\[
  \theta(t) = \theta_0\cos(\omega t)
  \quad\text{dove}\quad
  \omega = \sqrt{\frac{MDg}{I_z^\text{tot}}}\quad\text{è detta “pulsazione”}.
\]
Possiamo tuttavia facilmente esprimere $\omega$ in funzione del periodo
$T$ del moto oscillatorio, più semplice da calcolare dai dati acquisiti.
Vale infatti:
\[
  \omega = \frac{2\pi}{T}
  \qquad\text{e quindi}\qquad
  \frac{I_z^\text{tot}}{MD} = g \frac{T^2}{4\pi^2}
\]


\subsubsection{Il calcolo di $I_z^\text{tot}$ e $Mr_\text{CM}$}

% TODO: Definire $Gamma$
La formula utilizzata per il calcolo di $I_z^\text{tot}$ riflette la composizione
del sistema, sfruttando la proprietà additiva del momento d'inerzia:
\[I_z^\text{tot} = I_{z,\text{rotore}} + I_{z,\text{asta}} + \sum_{\gamma\in\Gamma} I_{z,\gamma}\]

Chiaramente, per calcolare i momenti d'inerzia rispetto all'asse di
rotazione è necessario applicare il teorema di Huygens-Steiner
a quelli calcolati sui rispettivi centri di massa:
\[  % TODO: Non dovrebbe essere - anziché + dentro alle (...)² ?
  I_{z,\text{asta}} = I_\text{CM,asta} + m_\text{asta}\left(\frac{L_\text{asta} + \diam_\text{rotore}}{2}\right)^2
\]\[
  I_{z,(i,d)} = I_{\text{CM},i} + m_i\left(d + \frac{h_i - \diam_\text{rotore}}{2}\right)^2\qquad\forall(i,d)\in\Gamma
\]

Per calcolare il termine $M r_\text{CM}$, si osservi che, per la
definizione di posizione del centro di massa, la massa totale si
semplifica:
\[\begin{aligned}
  Mr_\text{CM} &= M\cdot \frac{1}{M}\left(
    m_\text{rotore}\cdot 0 + m_\text{asta} r_\text{CM,asta} +
    \sum_{(i,d)\in\Gamma}{m_i r_{\text{CM},i}}
  \right) \\&= m_\text{asta}\left(\frac{L_\text{asta} + \diam_\text{rotore}}{2}\right) +
    \sum_{(i,d)\in\Gamma}{m_i \left(d + \frac{h_i - \diam_\text{rotore}}{2}\right)}
\end{aligned}\]

Di seguito riportiamo le misure, dirette e indirette, utilizzate per il calcolo dei momenti d'inerzia:

\begin{center}
    \begin{tblr}{ |c|c|c|c|c| }
        \hline
        Oggetto & $l\;\;(\unit{cm})$ & $\diam\;\;(\unit{mm})$ & $m\;\;(\unit{g})$ & $I_\text{CM}\;\;(10^{-5}\,\unit{kg\,m^2})$ \\
        \hline
        Asta & $60.0\pm0.1$ & $5.94\pm0.01$ & $45.82\pm0.01$ & $568.5\pm1.5$ \\
        \hline[dashed]
        Rotore & N./A. & $13.41\pm0.01$ & $22.4\pm0.1^*$ & $0.058\pm0.001^*$ \\
        \hline
    \end{tblr}
\end{center}

\emph{$[^*]$ Valori dati}

\begin{center}
\begin{tblr}{ |c|c|c|c|c|c| }
    \hline
    $i$ & $m_i\;\;(\unit{g})$ & $d_i^\text{\,ext}\;\;(\unit{mm})$ & $d_i^\text{\,int}\;\;(\unit{mm})$ & $h_i\;\;(\unit{mm})$ & $I_{\text{CM},i}\;(\unit{mg\,m^2})$ \\
    \hline
    A & $115.95\pm0.01$ & $29.95 \pm 0.05$ & $6.20 \pm 0.05$ & $19.93 \pm 0.01$ & $10.62\pm0.03$ \\
    \hline[dashed]
    B & $115.86\pm0.01$ & $29.95 \pm 0.05$ & $6.20 \pm 0.05$ & $19.89 \pm 0.01$ & $10.59\pm0.03$\\
    \hline[dashed]
    C & $71.46\pm0.01$ & $29.95 \pm 0.05$ & $6.20 \pm 0.05$ & $12.08 \pm 0.01$ & $5.047\pm0.018$\\
    \hline
\end{tblr}  % TODO: Tabella troppo larga
\end{center}

\subsubsection{Il calcolo di $g$}

Utilizzando le formule di cui sopra, il gruppo di lavoro ha calcolato,
per ogni configurazione $\Gamma$, i valori di $\frac{I_z^\text{tot}}{MD}$
e $\frac{T^2}{4\pi^2}$, riportati nel grafico seguente.

Come è possibile osservare dalla relazione che le lega, la dipendenza
tra queste due grandezze è lineare: questo ci permette di determinare
il valore di $g$ come coefficiente angolare di una retta di regressione.

\begin{figure}[H]
  \includegraphics[trim={2cm 1cm 2cm 2.1cm},clip,width=\textwidth]{img/regressione.png}
\end{figure}

I risultati della regressione lineare sono i seguenti:
\begin{enumerate}
  \item Intercetta $= (0.003 \pm 0.005)\;\unit{m}$ (compatibile con $0$)
  \item Coefficiente angolare $g = (9.68 \pm 0.13)\;\unit{m\per s^2}$
    (compatibile con $g_\text{atteso} = \qty{9.805}{m\per s^2}$)
\end{enumerate}

\pagebreak
\subsection{Misura dello smorzamento}

Il moto del pendolo fisico è condizionato dalla presenza
di attriti, che ne modificano ampiezza e periodo.
In particolare, il modello matematico di riferimento è descritto da:
\[\theta(t) = \theta_0\cos(\omega t)\,e^{-\lambda t}\]
dove $\lambda$ è un parametro costante legato allo smorzamento
del moto.

\begin{center}
    \begin{figure}[H]
        % trim={< v > ^}
        % \includegraphics[trim={2cm 1cm 2cm 2.1cm},clip,width=\textwidth]{img/Exp1.jpg}
        \caption[]{\emph{
            I dati di un'acquisizione di $\theta(t)$,
            come raccolti dal sensore di rotazione,
            riportati su una larga scala temporale.
            Si può chiaramente notare lo smorzamento del moto.
        }}
    \end{figure}
\end{center}

Per stimare $\lambda$, il gruppo di lavoro ha proceduto
sull'acquisizione in \emph{Figura 2} come segue:
\begin{enumerate}
    \item
        Per prima cosa, abbiamo calcolato $\left|\theta(t)\right|$.
        Ciò ci ha permesso di trattare massimi e minimi “insieme”,
        evitando di ripetere l'analisi.
    \item
        Poi, abbiamo individuato i picchi dei nostri dati, ovvero
        gli insiemi di punti della forma
        $\left\{t_i,t_{i+1},\dots,t_j\right\}\times\left\{\left|\theta_k\right|\right\}$
        tali che $\left|\theta_{i-1}\right| < \left|\theta_k\right| > \left|\theta_{j+1}\right|$.
    \item
        Per ogni picco, ne abbiamo calcolato il punto medio,
        prendendo come $\delta t_\text{picco}$ la semidispersione $\frac{1}{2}(t_j - t_i) + \delta t$.
    \item
        Infine, abbiamo graficato i punti così trovati
        su scala logaritmica e
        abbiamo effettuato una regressione lineare (pesata\footnote{
            $\delta\!\ln{\left|\theta\right|}$, infatti, varia molto,
            nonostante $\delta\!\left|\theta\right|$ sia costante:
            ciò è conseguenza della propagazione degli errori.
            È inoltre possibile osservarlo nella \emph{Figura 3}.
        })
        sulle nuove ordinate.
\end{enumerate}

\begin{center}
    \begin{figure}[H]
        % trim={< v > ^}
        % \includegraphics[trim={2cm 1cm 2cm 2.1cm},clip,width=\textwidth]{img/Exp4.jpg}
        \caption[]{\emph{
            $\left|\theta(t)\right|$, su scala logaritmica.
            Sono riportate anche le barre di errore.
            In blu, una retta di regressione lineare
            sull'intervallo di dati in nero.
        }}
    \end{figure}
\end{center}

Dai risultati della regressione lineare emerge che
\[\lambda = \left(46.67\pm0.11\right)\unit{mHz}\]

Abbiamo infine valutato il contributo dell'attrito sul periodo
dell'oscillazione. Vale infatti:
\[\omega_0^2 = \omega^2 + \lambda^2\]
dove $\omega=\frac{2\pi}{T}$ è la pulsazione misurata
mentre $\omega_0$ è la pulsazione in assenza di attrito.

Si ottiene allora:
\[T_0 = \sqrt{\frac{1}{\frac{1}{T^2} + \left(\frac{\lambda}{2\pi}\right)^2}}\]
dove $T$ è il periodo misurato mentre
$T_0=\frac{2\pi}{\omega_0}$ è il periodo in assenza di attrito.

Per questa acquisizione:
\[T = \left(409.96\pm0.04\right)\unit{ms}\]

da cui segue:
\[T_0=\left(409.95\pm0.04\right)\unit{ms}\]

In conclusione, possiamo affermare ragionevolmente che,
rispetto alla sensibilità degli strumenti di misura,
il contributo dell'attrito è trascurabile.

\end{document}
