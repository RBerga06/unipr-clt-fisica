\documentclass{article}
\usepackage[utf8]{inputenc}
\usepackage[italian]{babel}
\usepackage{amsmath}
\usepackage{amssymb}
\usepackage{siunitx}
\usepackage{tabularray}
\usepackage{graphicx}
\usepackage{float}
% \usepackage{minted}
\usepackage[bottom]{footmisc}
\usepackage[page]{appendix}
\newcommand*{\diam}{\varnothing}
\newcommand*{\best}[1]{{#1}_\text{best}}
\newcommand*{\bestp}[1]{{\left(#1\right)}_\text{best}}
\newcommand*{\pbest}[1]{\left({#1}_\text{best}\right)}
\newcommand*{\pbestp}[1]{\left({\left(#1\right)}_\text{best}\right)}
\newcommand*{\errrel}[1]{\frac{\delta #1}{{#1}_\text{best}}}
\title{
    Laboratorio di Fisica 1\\
    R7: Misura di $\left|\vec{g}\right|$ mediante pendolo fisico
}
\author{Gruppo 15: Bergamaschi Riccardo, Graiani Elia, Moglia Simone}
\date{05/03/2024 – 12/03/2024}
\makeindex
\begin{document}

\maketitle

\begin{abstract}
    Il gruppo di lavoro ha misurato il modulo del campo gravitazionale locale
    ($g$) studiando il moto oscillatorio di un pendolo fisico.
\end{abstract}

\section{Materiali e strumenti di misura utilizzati}
\begin{center}
    \begin{tblr}{ |Q[l,m]|Q[c,m]|Q[c,m]|Q[c,m]| }
        \hline
        \textbf{Strumento di misura} & \textbf{\:\:\:\:\:Soglia\:\:\:\:\:} & \textbf{Portata} & \textbf{Sensibilità} \\
        \hline
        Sensore di rotazione & $\qty{0.002}{rad}$ & N./A. & $\qty{0.002}{rad}$ \\
        \hline[dashed]
        Cronometro & $\qty{0.001}{s}$ & N./A. & $\qty{0.001}{s}$ \\
        \hline[dashed]
        Micrometro ad asta filettata & $\qty{0.01}{mm}$ & $\qty{25.00}{mm}$ & $\qty{0.01}{mm}$ \\
        \hline[dashed]
        Calibro ventesimale & $\qty{0.05}{mm}$ & $\qty{150.00}{mm}$ & $\qty{0.05}{mm}$ \\
        \hline[dashed]
        Metro & $\qty{0.1}{cm}$ & $\qty{300.0}{cm}$ & $\qty{0.1}{cm}$ \\
        \hline[dashed]
        Bilancia di precisione & $\qty{0.01}{g}$ & $\qty{6200.00}{g}$ & $\qty{0.01}{g}$ \\
        \hline
        \hline
        \textbf{Altro} & \SetCell[c=3]{l} \textbf{Descrizione/Note} \\
        \hline
        {Rotore e asta} & \SetCell[c=3]{l} {
            % "L'ASTA È IL SOGGETTO DEL PENDOLO"
            L'asta, fissata ortogonalmente al rotore ad un \\
            estremo, è libera di ruotare grazie ad esso.
        } \\
        \hline[dashed]
        {Tre cilindri \\ (con masse e raggi distinti)} & \SetCell[c=3]{l} {
            Presentano un foro centrale lungo l'asse di \\
            simmetria. Indicheremo con $A,B,C$ i tre cilindri \\
            e con $0$, $A+B$, $A+C$, $B+C$ e $A+B+C$ \\
            le loro combinazioni.
            } \\
        \hline
    \end{tblr}
\end{center}

\section{Esperienza e procedimento di misura}

\begin{enumerate}
    \item
        Misuriamo le masse dei cilindri con la bilancia di precisione,
        i rispettivi diametri (interni ed esterni) con il calibro ventesimale
        e le altezze con il micrometro ad asta filettata.
    \item
        Con il metro a nastro misuriamo la lunghezza dell'asta e con il
        micrometro il suo diametro, nonché il diametro del rotore.
    \item
        Per ogni configurazione di cilindri:  % TODO: Riformulare
    \begin{enumerate}
        \item
            Fissiamo i cilindri scelti all'asta attraverso il foro centrale
            e ne misuriamo la distanza dal rotore.
        \item
            Avviamo l'acquisizione dell'angolo in funzione del tempo
            ($\theta(t)$, lo definiremo formalmente più avanti).
        \item
            Ruotando l'asta di un angolo prefissato $\theta_0$,
            sufficientemente piccolo\footnote{
                Questa condizione sull'angolo $\theta_0$ ci permette di
                approssimare $\sin(x) \sim x$. %TODO: fix bisciolino
            },
            diamo inizio al moto armonico del pendolo.
            Acquisiamo dati fino all'arresto del moto.
    \end{enumerate}
\end{enumerate}

\section{Analisi dei dati raccolti e conclusioni}
\emph{\textbf{Nota.}
Avendo valutato gli errori sulle grandezze misurate direttamente
come piccoli, casuali e indipendenti, per svolgere ogni calcolo
abbiamo utilizzato la tradizionale propagazione degli errori.
}

\subsection{Misura di $\left|\vec{g}\right|$}

Di seguito riportiamo i momenti d'inerzia costanti per tutto l'esperimento:

\begin{center}
    \begin{tblr}{ |c|c|c|c|c| }
        \hline
        Oggetto & $l\;\;(\unit{cm})$ & $\diam\;\;(\unit{mm})$ & $m\;\;(\unit{g})$ & $I\;\;(10^{-5}\,\unit{kg\,m^2})$ \\
        \hline
        Asta & $60.0\pm0.1$ & $5.94\pm0.01$ & $45.82\pm0.01$ & $568.5\pm1.5$ \\
        \hline[dashed]
        Rotore & N./A. & $13.41\pm0.01$ & $22.4\pm0.1^*$ & $0.058\pm0.001^*$ \\
        \hline
    \end{tblr}
\end{center}

\emph{$[^*]$ Valori dati}

Di seguito riportiamo massa, diametri (interni ed esterni) e altezza dei tre cilindri.

\begin{center}
\begin{tblr}{ |c|c|c|c|c| }
    \hline
    $i$ & $m_i\;\;(\unit{g})$ & $d_i^\text{ext}\;\;(\unit{mm})$ & $d_i^\text{int}\;\;(\unit{mm})$ & $h_i\;\;(\unit{mm})$ \\
    \hline
    A & $115.95 \pm 0.01$ & $29.95 \pm 0.05$ & $6.20 \pm 0.05$ & $19.93 \pm 0.01$ \\
    \hline[dashed]
    B & $115.86 \pm 0.01$ & $29.95 \pm 0.05$ & $6.20 \pm 0.05$ & $19.89 \pm 0.01$ \\
    \hline[dashed]
    C & $71.46 \pm 0.01$ & $29.95 \pm 0.05$ & $6.20 \pm 0.05$ & $12.08 \pm 0.01$ \\
    \hline
\end{tblr}
\end{center}

Sappiamo che il moto armonico del pendolo segue la legge $\sum \tau^\text{ext} = I\alpha$ in quanto compie una rotazione. \\
Detta $D$ la posizione del centro di massa rispetto all'asse di rotazione, possiamo scrivere $-Mg\sin(\theta)D = I\alpha$. \\
Approssimando $\sin(\theta)$ a $\theta$ ed esprimendo l'accelerazione angolare come derivata seconda dello spostamento angolare:
\[ \frac{d^2\theta}{dt^2} = -\frac{MgD}{I}\,\theta \]
da cui ricaviamo l'equazione della retta di regressione, il cui grafico verrà riportato in seguito.
\[ \left(\frac{2\pi}{T}\right)^2 = \frac{MDg}{I} \]

%TODO: AGGIUNGERE GRAFICO DELLA REGRESSIONE


Per valutare numericamente la consistenza dei risultati ottenuti con i valori
$G$ riportati in letteratura ($G_l$),
abbiamo calcolato, per ogni filo $j$, il seguente valore (numero puro):
\[\varepsilon = \frac{{G_j}_\text{best} - {G_l}_\text{best}}
                     {\delta G_j + \delta G_l}\]
Allora $G_j$ è consistente con $G_l$ se e solo se $\left|\varepsilon\right|\le 1$.

\begin{center}
\begin{tblr}{ |c|c|c|c|c| }
    \hline
    $j$ & $G_i$ ($\unit{GPa}$) & Materiale & $G_l$ ($\unit{GPa}$) & $\varepsilon$ \\
    \hline
    1 & $82\pm4$ &         &          & $-0.468$ \\
    2 & $80\pm3$ & Acciaio & $84\pm1$ & $-0.978$ \\
    3 & $81\pm2$ &         &          & $-0.809$ \\
    \hline[dashed]
    4 & $45.4\pm1.1$ & Rame& $43\pm1$ & $+1.174$ \\
    \hline
\end{tblr}
\end{center}

L'inconsistenza non trascurabile tra i valori di $G$ per il filo di
rame potrebbe essere dovuta alle cattive condizioni del filo stesso.

Infatti, il gruppo di lavoro lo ha reciso da una bobina, per poi
srotolarlo: queste operazioni hanno lasciato imperfezioni visibili
ad occhio nudo sul filo, come, ad esempio, piccole piegature.

Riteniamo che queste imperfezioni potrebbero avere influenzato
le nostre misure in maniera non trascurabile.

\pagebreak
\subsection{Misura dello smorzamento}

Il moto del pendolo fisico è condizionato dalla presenza
di attriti, che ne modificano ampiezza e periodo.
In particolare, il modello matematico di riferimento è descritto da:
\[\theta(t) = \theta_0\cos(\omega t)\,e^{-\lambda t}\]
dove $\lambda$ è un parametro costante legato allo smorzamento
del moto.

\begin{center}
    \begin{figure}[H]
        % trim={< v > ^}
        % \includegraphics[trim={2cm 1cm 2cm 2.1cm},clip,width=\textwidth]{img/Exp1.jpg}
        \caption[]{\emph{
            I dati di un'acquisizione di $\theta(t)$,
            come raccolti dal sensore di rotazione,
            riportati su una larga scala temporale.
            Si può chiaramente notare lo smorzamento del moto.
        }}
    \end{figure}
\end{center}

Per stimare $\lambda$, il gruppo di lavoro ha proceduto
sull'acquisizione in \emph{Figura 2} come segue:
\begin{enumerate}
    \item
        Per prima cosa, abbiamo calcolato $\left|\theta(t)\right|$.
        Ciò ci ha permesso di trattare massimi e minimi “insieme”,
        evitando di ripetere l'analisi.
    \item
        Poi, abbiamo individuato i picchi dei nostri dati, ovvero
        gli insiemi di punti della forma
        $\left\{t_i,t_{i+1},\dots,t_j\right\}\times\left\{\left|\theta_k\right|\right\}$
        tali che $\left|\theta_{i-1}\right| < \left|\theta_k\right| > \left|\theta_{j+1}\right|$.
    \item
        Per ogni picco, ne abbiamo calcolato il punto medio,
        prendendo come $\delta t_\text{picco}$ la semidispersione $\frac{1}{2}(t_j - t_i) + \delta t$.
    \item
        Infine, abbiamo graficato i punti così trovati
        su scala logaritmica e
        abbiamo effettuato una regressione lineare (pesata\footnote{
            $\delta\!\ln{\left|\theta\right|}$, infatti, varia molto,
            nonostante $\delta\!\left|\theta\right|$ sia costante:
            ciò è conseguenza della propagazione degli errori.
            È inoltre possibile osservarlo nella \emph{Figura 3}.
        })
        sulle nuove ordinate.
\end{enumerate}

\begin{center}
    \begin{figure}[H]
        % trim={< v > ^}
        % \includegraphics[trim={2cm 1cm 2cm 2.1cm},clip,width=\textwidth]{img/Exp4.jpg}
        \caption[]{\emph{
            $\left|\theta(t)\right|$, su scala logaritmica.
            Sono riportate anche le barre di errore.
            In blu, una retta di regressione lineare
            sull'intervallo di dati in nero.
        }}
    \end{figure}
\end{center}

Dai risultati della regressione lineare emerge che
\[\lambda = \left(46.67\pm0.11\right)\unit{mHz}\]

Abbiamo infine valutato il contributo dell'attrito sul periodo
dell'oscillazione. Vale infatti:
\[\omega_0^2 = \omega^2 + \lambda^2\]
dove $\omega=\frac{2\pi}{T}$ è la pulsazione misurata
mentre $\omega_0$ è la pulsazione in assenza di attrito.

Si ottiene allora:
\[T_0 = \sqrt{\frac{1}{\frac{1}{T^2} + \left(\frac{\lambda}{2\pi}\right)^2}}\]
dove $T$ è il periodo misurato mentre
$T_0=\frac{2\pi}{\omega_0}$ è il periodo in assenza di attrito.

Per questa acquisizione:
\[T = \left(409.96\pm0.04\right)\unit{ms}\]

da cui segue:
\[T_0=\left(409.95\pm0.04\right)\unit{ms}\]

In conclusione, possiamo affermare ragionevolmente che,
rispetto alla sensibilità degli strumenti di misura,
il contributo dell'attrito è trascurabile.

\end{document}
