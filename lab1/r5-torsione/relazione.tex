\documentclass{article}
\usepackage[utf8]{inputenc}
\usepackage[italian]{babel}
\usepackage{amsmath}
\usepackage{amssymb}
\usepackage{siunitx}
\usepackage{tabularray}
\usepackage{graphicx}
\usepackage{float}
\usepackage{minted}
\usepackage[page]{appendix}
\newcommand*{\diam}{\varnothing}
\newcommand*{\best}[1]{{#1}_\text{best}}
\newcommand*{\bestp}[1]{{\left(#1\right)}_\text{best}}
\newcommand*{\pbest}[1]{\left({#1}_\text{best}\right)}
\newcommand*{\pbestp}[1]{\left({\left(#1\right)}_\text{best}\right)}
\newcommand*{\errrel}[1]{\frac{\delta #1}{{#1}_\text{best}}}
\newcommand*{\Th}{^{232}_{\;\;90} \text{Th}}
\title{
    Laboratorio di Fisica 1\\
    R5: Misura del modulo di scorrimento di un filo
}
\author{Gruppo 17: Bergamaschi Riccardo, Graiani Elia, Moglia Simone}
\date{22/11/2023 – 29/11/2023}
\makeindex
\begin{document}

\maketitle

\begin{abstract}
    Il gruppo di lavoro ha misurato la costante di torsione di diversi fili,
    in modo da ricavarne il modulo di scorrimento.
\end{abstract}

\section{Materiali e strumenti di misura utilizzati}
\begin{center}
    \begin{tblr}{ |Q[l,m]|Q[c,m]|Q[c,m]|Q[c,m]| }
        \hline
        \textbf{Strumento di misura} & \textbf{\:\:\:\:\:Soglia\:\:\:\:\:} & \textbf{Portata} & \textbf{Sensibilità} \\
        \hline
        Sensore di rotazione & $\qty{00}{rad}?$ & $\qty{00}{rad}?$ & $\qty{00}{rad}?$ \\
        \hline[dashed]
        Metro & $\qty{0.1}{cm}$ & $\qty{300.0}{cm}$ & $\qty{0.1}{cm}$ \\
        \hline[dashed]
        Calibro ventesimale & $\qty{0.05}{mm}?$ & $\qty{150.00}{mm}$ & $\qty{0.05}{mm}$ \\
        \hline[dashed]
        Micrometro ad asta filettata & $\qty{0.01}{mm}$ & $\qty{25.00}{mm}$ & $\qty{0.01}{mm}$ \\
        \hline[dashed]
        Bilancia di precisione & $\qty{0.01}{g}?$ & $\qty{6200.00}{g}?$ & $\qty{0.01}{g}?$ \\
        \hline
        \hline
        \textbf{Altro} & \SetCell[c=3]{l} \textbf{Descrizione/Note} \\
        \hline
        {Quattro fili} & \SetCell[c=3]{l} {
            In particolare tre in alluminio e uno \\         %TODO: erano veramente in alluminio?
            in rame, principalmente distiguibili \\
            per il diametro.} \\
        \hline[dashed]
        {Tre cilindri \\ (con masse distinte)} & \SetCell[c=3]{l} {
            Indicheremo con $A,B,C$ i tre cilindri e \\
            con $\diam$, $A+B$, $A+C$, $B+C$ e $A+B+C$ \\
            le loro combinazioni. Tutti questi saranno \\
            qui chiamati “gravi”.} \\                        %TODO: dare un nome diverso da "gravi"
        {Pendolo di torsione} & \SetCell[c=3]{l} {           %"IL FILO È IL SOGGETTO DEL PENDOLO"
            Formato da un disco che ruotando provoca \\
            una deformazione elastica al filo ad esso \\
            fissato.} \\
             
        \hline
        \hline
    \end{tblr}
\end{center}

\section{Esperienza e procedimento di misura}

\begin{enumerate}
    \item
        Misuriamo la massa dei cilindri con la bilancia di precisione
        e il loro diametro con il calibro ventesimale.
    \item
        Con il metro a nastro e il micrometro ad asta filettata misuriamo
        rispettivamente la lunghezza e il diametro dei fili.
    \item Per ogni filo:
    \begin{enumerate}
        \item
            Lo sistemiamo nel pendolo in modo tale che l'estremità
            superiore sia libera di ruotare.
        \item
            Tramite una torsione 
            Grazie al sensore otteniamo l'angolo di rotazione rispetto
            a quello di partenza in funzione del tempo.
        \item
            Aggiungiamo progressivamente al disco del pendolo di torsione i cilindri e le varie combinazioni,
            in modo da ottenere più risultati differenti. Ripetiamo il procedimento per ogni filo.
    \end{enumerate}
\end{enumerate}








Il valore di $\overline{x_i}$ è legato alla distanza $d_i$, al raggio $r$
della finestra del contatore, al tempo di dimezzamento $T_\frac{1}{2}$ del
$\Th$ e al numero $N$ di atomi di $\Th$ secondo la seguente relazione\footnote{
    Il numero medio $\overline{X}$ di raggi $\gamma$ emessi dal campione è:
    \[
        \overline{X} = \frac{N}{\tau} = \frac{N\ln{2}}{T_\frac{1}{2}}
    \]
    con $\tau=\frac{1}{\ln{2}}T_\frac{1}{2}$ il tempo caratteristico del $\Th$.
    Tuttavia, poiché la superficie di acquisizione è $\pi r^2$, e non $4\pi d_i^2$,
    dobbiamo moltiplicare per il rapporto fra le aree:
    \[
        \overline{x_i} = \frac{\pi r^2}{4\pi d_i^2} \overline{X} =
        \frac{Nr^2\ln{2}}{4 d_i^2 T_\frac{1}{2}}
    \]
    Infine, dobbiamo ricordare che il contatore Geiger non rileva soltanto le
    radiazioni emesse dal campione. Possiamo tenere conto di tutti gli altri
    contributi introducendo un termine costante $\overline{x_0}$, che chiameremo
    “media dei conteggi relativi alla radioattività ambientale”.
    Si ottiene così la relazione riportata.
}:
\[
    \overline{x_i} = \frac{Nr^2\ln{2}}{4 d_i^2 T_\frac{1}{2}} + \overline{x_0}
\]
D'ora in avanti indicheremo con $\xi$ la costante $\frac{Nr^2\ln{2}}{4T_\frac{1}{2}}$
(unità di misura: $\unit{m^2\per s}$).
La relazione diventa allora:
\[\overline{x_i} = \xi d_i^{-2} + \overline{x_0}\]

Per ogni distanza $d_i$, il gruppo di lavoro ha acquisito ripetutamente il valore di
$x_i$ per un tempo complessivo di circa un'ora\footnote{
    Abbiamo scelto deliberatamente di acquisire esattamente 3657 secondi in quanto
    $3657$ minimizza la funzione
    $f(x)=\left\{\frac{x}{\pi}\right\}=\frac{x}{\pi} - \left\lfloor\frac{x}{\pi}\right\rfloor$
    meglio di $3600$.
} ($\qty{3657}{s}$); ha poi acquisito nuovamente $x_4$ col contatore Geiger rivolto in
direzione opposta, per avere una stima diretta di $\overline{x_0}$.

Di seguito riportiamo gli istogrammi dei dati così raccolti,
assieme ai valori attesi, calcolati mediante la distribuzione teorica.


Come si può osservare da questi grafici, i risultati ottenuti si allineano molto bene
alle distribuzioni di Poisson.
Per valutare l'accuratezza della nostra stima di $\overline{x_0}$, possiamo effettuare
una regressione lineare (pesata) utilizzando l'equazione di $x_i$ in funzione di
$d_i^{-2}$: \[\overline{x_i} = \xi d_i^{-2} + \overline{x_0}\]
Di seguito riportiamo una tabella con i dati utilizzati per la regressione lineare,
assieme a un grafico della retta di regressione stessa.


\begin{center}
    \begin{tblr}{ |Q[c,m]|Q[c,m]|Q[c,m]|Q[c,m]| }
        \hline
        $i$ & $d_i\;\;(\unit{cm})$ & $d_i^{-2}\;\;(\unit{m^{-2}})$ & $\overline{x_i}$ \\
        \hline
        1 & $9.5\pm0.1$  & $111\pm2$      & $0.809\pm0.015$\\
        2 & $10.3\pm0.1$ & $94.3\pm1.8$   & $0.737\pm0.014$\\
        3 & $26.5\pm0.1$ & $14.24\pm0.11$ & $0.272\pm0.009$\\
        4 & $41.2\pm0.1$ & $5.89\pm0.03$  & $0.219\pm0.008$\\
        \hline
    \end{tblr}

\end{center}

Risultati della regressione lineare:
\begin{itemize}
    \item $\overline{x_0} = 0.188\pm0.006$
    \item $\xi = \left(5.70\pm0.13\right)\cdot10^{-3}\,\unit{m^2\per s}$
\end{itemize}

Per valutare numericamente la consistenza tra i due valori di $\overline{x_0}$ ottenuti,
abbiamo calcolato il seguente valore (numero puro):
\[
    \varepsilon =
    \frac{
        \left(x_{0\,\text{misurato}}\right)_\text{best} - \left(x_{0\,\text{misurato}}\right)_\text{best}
    }{
        \delta x_{0\,\text{misurato}} + \delta x_{0\,\text{misurato}}
    }
\]
Allora $x_{0\,\text{misurato}}$ e $x_{0\,\text{misurato}}$ sono consistenti se e solo se $\left|\varepsilon\right|\le1$.

Nel nostro caso, $\varepsilon = 1.33$. Il gruppo di lavoro ha ipotizzato che
questa inconsistenza (comunque contenuta, seppur non trascurabile) fra le due
misure possa essere ragionevolmente giustificata dalla difficoltà incontrata
nel ridurre al minimo le oscillazioni in direzione perpendicolare a $\vec{g}$;
considerato inoltre che la posizione dei fototraguardi non era ottimale, ciò
potrebbe avere ulteriormente influenzato la distribuzione dei tempi. È in
effetti possibile osservare che le distribuzioni da noi ottenute non sono,
il più delle volte, del tutto simmetriche: la moda sembra essersi spostata
leggermente a sinistra – un possibile sintomo dell'influenza di un
errore sistematico sulle misure.

\pagebreak
\end{document}
