\documentclass{article}
\usepackage[utf8]{inputenc}
\usepackage[italian]{babel}
\usepackage{amsmath}
\usepackage{siunitx}
\usepackage{tabularray}
\title{
    Laboratorio di Fisica 1\\
    R2: Misura costante elastica di una molla
}
\author{Gruppo 17: Bergamaschi Riccardo, Graiani Elia, Moglia Simone}
\date{04/10/2023 – 11/10/2023}
\makeindex
\begin{document}

\maketitle

\begin{abstract}
    Il gruppo di lavoro ha misurato la costante elastica di una molla con due metodi distinti.
\end{abstract}

\section{Materiali e strumenti di misura utilizzati}
\begin{center}
    \begin{tblr}{ |Q[l,m]|Q[c,m]|Q[c,m]|Q[c,m]| }
        \hline
        \textbf{Strumento di misura} & \textbf{\:\:\:\:Soglia\:\:\:\:} & \textbf{Portata} & \textbf{Sensibilità} \\
        \hline
        {Fototraguardo con \\ contatore di impulsi} & \qty{1}{\micro s} & \qty{99999999}{\micro s} & \qty{1}{\micro s} \\
        \hline[dashed]
        Righello & \qty{0.1}{cm} & \qty{60.0}{cm} & \qty{0.1}{cm} \\
        \hline[dashed]
        Bilancia di precisione & \qty{0.01}{g} & \qty{6200.00}{g} & \qty{0.01}{g} \\
        \hline
        \hline
        \textbf{Altro} & \SetCell[c=3]{l} \textbf{Descrizione/Note} \\
        \hline
        Molla e gancio & \SetCell[c=3]{l} {
            Un estremo della molla è vincolato ad un supporto \\
            fisso, mentre all'altro è appeso un gancio per \\
            agevolare il caricamento dei campioni
        } \\
        \hline[dashed]
        3 campioni solidi & \SetCell[c=3]{l} Con masse distinte \\
        \hline[dashed]
        Specchio & \SetCell[c=3]{l} {
            Posizionato dietro al righello, permette di \\
            ridurre eventuali errori di lettura dovuti \\
            all'effetto di parallasse
        } \\
        \hline[dashed]
        Livella & \SetCell[c=3]{l} {
            Utile per assicurarsi che il fototraguardo \\
            sia orizzontale
        } \\
        \hline
    \end{tblr}
\end{center}

% *questi strumenti di misura, seppur disponibili, non sono stati
% utilizzati a causa della loro elevata sensiblità.

\section{Esperienza e procedimento di misura}
\subsection{Misurazione della costante elastica nel caso statico}
\begin{enumerate}
    \item Fissiamo il righello davanti allo specchio, parallelo alla direzione del campo gravitazionale locale e solidale all’estremo fisso della molla. Individuiamo un punto del sistema, solidale all’estremo libero della molla, che terremo come riferimento per misurare l’allungamento della molla: ne misuriamo allora la posizione $x_\text{0}$
    \item Consideriamo i tre campioni singolarmente, e poi tutte le loro combinazioni:
    \begin{itemize}
        \item Ne misuriamo la massa $m_\text{i}$ con la bilancia di precisione
        (nel caso di combinazioni di più campioni, ne misuriamo la massa complessiva);
        \item Appeso il grave alla molla, ne misuriamo l'allungamento $(\Delta x)_\text{i}$, sottraendo $x_\text{0}$ alla misura $x_\text{i}$ della sua posizione ($\delta (\Delta x)_\text{i} = \delta  x_\text{0} + \delta  x_\text{i}$). Per ridurre ulteriormente la probabilità di commettere un errore di parallasse, ripetiamo il procedimento tre volte, tenendo solamente la misura più vicina alla media.
    \end{itemize}
\end{enumerate}
\subsection{Misurazione della costante elastica nel caso dinamico}
\begin{enumerate}
    \item Accendiamo il contatore di impulsi e lo impostiamo su \emph{Universal Counter} e su 20 oscillazioni;
    \item Consideriamo i tre campioni $A, B, C$ e $A+B$:
    \begin{itemize}
        \item Appeso il campione alla molla, allineiamo i due fototraguardi aiutandoci con la livella, in modo tale che possano rilevare le oscillazioni;
        \item Tiriamo il campione verso il basso e poi lo rilasciamo, in modo che il sistema molla inizi a oscillare con direzione parallela al campo gravitazionale locale;
        \item Una volta verificato che l’oscillazione sia stabile, facciamo partire il contatore di impulsi, che misurerà il tempo impiegato per compiere 20 oscillazioni;
    \end{itemize}
\end{enumerate}

\section{Dati raccolti e conclusioni}
Di seguito sono riportate tutte le misure effettuate direttamente, così come quelle calcolate come descritto.

\begin{center}
    \begin{tabular}{ |c|c|c|c| }
        \hline
        \emph{Parallelepipedo} & $x$ (mm) & $y$ (mm) & $z$ (mm) \\
        \hline
        Misura 1 & 39.90 ± 0.05 & 64.60 ± 0.05 & 5.01 ± 0.01 \\
        Misura 2 & 39.90 ± 0.05 & 64.40 ± 0.05 & 4.99 ± 0.01 \\
        Misura 3 & 39.90 ± 0.05 & 64.40 ± 0.05 & 4.98 ± 0.01 \\
        \hline
        Misura tenuta & 39.90 ± 0.05 & 64.40 ± 0.05 & 4.99 ± 0.01 \\
        \hline
    \end{tabular}
    \newline
    \vspace*{0.4 cm}
    \newline
    \begin{tabular}{ |c|c|c| }
        \hline
        \emph{Cilindro} 1 & $h$ (mm) & $d$ (mm) \\
        \hline
        Misura 1 & 24.83 ± 0.01 & 27.95 ± 0.05 \\
        Misura 2 & 24.82 ± 0.01 & 28.05 ± 0.05 \\
        Misura 3 & 24.83 ± 0.01 & 28.00 ± 0.05 \\
        \hline
        Misura tenuta & 24.83 ± 0.01 & 28.00 ± 0.05 \\
        \hline
    \end{tabular}
    \newline
    \vspace*{0.4 cm}
    \newline
    \begin{tabular}{ |c|c| }
        \hline
        \emph{Sfera} & $d$ (mm) \\
        \hline
        Misura 1 & 20.63 ± 0.01 \\
        Misura 2 & 20.63 ± 0.01 \\
        Misura 3 & 20.64 ± 0.01 \\
        \hline
        Misura tenuta & 20.63 ± 0.01 \\
        \hline
    \end{tabular}
    \newline
    \vspace*{0.4 cm}
    \newline
    \begin{tabular}{ |c|c|c| }
        \hline
        \emph{Cilindro 2} & $h$ (mm) & $d$ (mm) \\
        \hline
        Misura 1 & 77.75 ± 0.05 & 6.97 ± 0.01 \\
        Misura 2 & 77.80 ± 0.05 & 6.97 ± 0.01 \\
        Misura 3 & 77.80 ± 0.05 & 6.98 ± 0.01 \\
        \hline
        Misura tenuta & 77.80 ± 0.05 & 6.97 ± 0.01 \\
        \hline
    \end{tabular}
    \newline
    \vspace*{0.4 cm}
    \newline
    \begin{tabular}{ |c|c|c|c| }
        \hline
        \emph{Campione} & $m$ (g) & $V$ ($\text{cm}^3$) & $\rho\enspace(\text{g} / \text{cm}^3)$\\
        \hline
        Parallelepipedo & 107.40 ± 0.01 & 12.87 ± 0.05 & 8.34 ± 0.03 \\
        Cilindro 1 & 41.21 ± 0.01 & 15.29 ± 0.06 & 2.695 ± 0.011 \\
        Sfera & 35.81 ± 0.01 & 4.597 ± 0.007 & 7.789 ± 0.014 \\
        Cilindro 2 & 8.00 ± 0.01 & 2.97 ± 0.01 & 2.695 ± 0.013 \\
        \hline
    \end{tabular}
    \newline
    \vspace*{0.4 cm}
    \newline
    \begin{tabular}{ |c|c|c|c|c| }
        \hline
        \emph{Campione} & $\rho\enspace(\text{g} / \text{cm}^3)$ & Materiale & $\rho_\text{lett.}\enspace(\text{g} / \text{cm}^3)$ & $\varepsilon$ \\
        \hline
        Parallelepipedo & 8.34 ± 0.03 & Ottone giallo (\emph{high brass}) & 8.47 ± 0.01 & 2.5 \\
        Cilindro 1 & 2.695 ± 0.011 & Lega di Al laminato 3003 & 2.73 ± 0.01 & 1.7 \\
        Sfera & 7.789 ± 0.014 & Acciaio & 7.8 ± 0.1 & 0.1 \\
        Cilindro 2 & 2.695 ± 0.013 & Lega di Al laminato 3003 & 2.73 ± 0.01 & 1.5 \\
        \hline
    \end{tabular}
\end{center}

L'inconsistenza non trascurabile tra $\rho$ (le nostre misure) e $\rho_\text{lett.}$ è dovuta principalmente al fatto che si tratta di leghe; probabilmente, i nostri campioni presentavano concentrazioni diverse dei vari elementi.

\end{document}
