\documentclass{article}
\usepackage[utf8]{inputenc}
\usepackage[italian]{babel}
\usepackage{amsmath}
\title{
    Laboratorio di Fisica 1\\
    R1: Misura indiretta della densità di solidi
}
\author{Riccardo Bergamaschi, Elia Graiani, Simone Moglia}
\date{27/09/2023}
\makeindex
\begin{document}

\maketitle

\begin{abstract}
    Il gruppo di lavoro ha misurato la densità di solidi ignoti
    per individuarne la natura.
\end{abstract}

\section{Materiali utilizzati e strumenti di misura}
Abbiamo misurato la densità di quattro campioni solidi:
un parallelepipedo, una sfera e due cilindri.
Di seguito gli strumenti di misura utilizzati:

\begin{center}
    \begin{tabular}{ |c|c|c|c| }
        \hline
        Nome & Soglia & Portata & Sensibilità \\
        \hline
        Micrometro ad asta filettata & 0.01 mm & 25.00 mm & 0.01 mm \\
        \hline
        Calibro ventesimale & 0.05 mm & 150.00 mm & 0.05 mm \\
        \hline
        Bilancia di precisione & 0.01 g & 2000.00 g & 0.01 g \\
        \hline
        Metro a nastro* & 0.1 cm & 300 cm & 0.1 cm \\
        \hline
        Cilindro graduato* & 1 mL & ??? & 1 mL \\
        \hline
    \end{tabular}
\end{center}

*questi strumenti di misura, seppur disponibili, non sono stati
utilizzati a causa della loro elevata sensiblità.

\section{Desrizione dell'esperimento e del procedimento di misura}
\begin{enumerate}
    \item Misuriamo per ogni campione la sua massa $m$ con la bilancia di precisione.
    \item Misuriamo tre volte per ogni campione le distanze necessarie al calcolo
          del suo volume, tenendo come valore migliore quello più vicino alla media
          delle misure e come incertezza la sensibilità degli strumenti utilizzati.
          Quando possibile, utilizziamo il micrometro; altrimenti, il calibro ventesimale.
    \item Per ogni campione, ne calcoliamo il volume $V$ (e la sua incertezza), in base alla forma:
    \begin{itemize}
        \item Parallelepipedo:
            \[V_\text{best} = x_\text{best} y_\text{best} z_\text{best}\]
            \[
                \frac{\delta V}{V_\text{best}} = \frac{\delta x}{x_\text{best}} + \frac{\delta y}{y_\text{best}} + \frac{\delta z}{z_\text{best}}
            \]
        \item Cilindri:
            \[V_\text{best} = \pi \left(\frac{d_\text{best}}{2}\right)^2 h_\text{best}\]
            \[
                \frac{\delta V}{V_\text{best}} = 2 \cdot \frac{\delta d}{d_\text{best}} + \frac{\delta h}{h_\text{best}}
            \]
        \item Sfera:
            \[V_\text{best} = \frac{4 \pi}{3} \left(\frac{d_\text{best}}{2}\right)^3\]
            \[
                \frac{\delta V}{V_\text{best}} = 3 \cdot \frac{\delta d}{d_\text{best}}
            \]
    \end{itemize}
    \item Sempre tenendo conto delle incertezze, troviamo la densità $\rho$ (e il relativo errore) del campione:
        \[\rho = \frac{m}{V}\]
        \[\frac{\delta \rho}{\rho_\text{best}} = \frac{\delta m}{m_\text{best}} + \frac{\delta V}{V_\text{best}}\]
    \item Infine, cerchiamo di capire di che materiale siano composti i vari campioni, confrontando
    i valori di $rho$ misurati con quelli indicati in letteratura ($\rho_\text{lett.}$).
\end{enumerate}

Di seguito sono riportate tutte le misure, dirette e indirette, effettuate.

\begin{center}
    \begin{tabular}{ |c|c|c|c| }
        \hline
        Parallelepipedo & x (mm) & y (mm) & z (mm) \\
        \hline
        Misura 1 & 39.90 ± 0.05 & 64.60 ± 0.05 & 5.01 ± 0.01 \\
        Misura 2 & 39.90 ± 0.05 & 64.40 ± 0.05 & 4.99 ± 0.01 \\
        Misura 3 & 39.90 ± 0.05 & 64.40 ± 0.05 & 4.98 ± 0.01 \\
        \hline
        Misura tenuta & 39.90 ± 0.05 & 64.40 ± 0.05 & 4.99 ± 0.01 \\
        \hline
    \end{tabular}

    \begin{tabular}{ |c|c|c| }
        \hline
        Cilindro 1 & h (mm) & d (mm) \\
        \hline
        Misura 1 & 24.83 ± 0.01 & 27.95 ± 0.05 \\
        Misura 2 & 24.82 ± 0.01 & 28.05 ± 0.05 \\
        Misura 3 & 24.83 ± 0.01 & 28.00 ± 0.05 \\
        \hline
        Misura tenuta & 24.83 ± 0.01 & 28.00 ± 0.05 \\
        \hline
    \end{tabular}

    \begin{tabular}{ |c|c| }
        \hline
        Sfera & d (mm) \\
        \hline
        Misura 1 & 20.63 ± 0.01 \\
        Misura 2 & 20.63 ± 0.01 \\
        Misura 3 & 20.64 ± 0.01 \\
        \hline
        Misura tenuta & 20.63 ± 0.01 \\
        \hline
    \end{tabular}

    \begin{tabular}{ |c|c|c| }
        \hline
        Cilindro 2 & h (mm) & d (mm) \\
        \hline
        Misura 1 & 77.75 ± 0.05 & 6.97 ± 0.01 \\
        Misura 2 & 77.80 ± 0.05 & 6.97 ± 0.01 \\
        Misura 3 & 77.80 ± 0.05 & 6.98 ± 0.01 \\
        \hline
        Misura tenuta & 77.80 ± 0.05 & 6.97 ± 0.01 \\
        \hline
    \end{tabular}

    \begin{tabular}{ |c|c|c|c| }
        \hline
        Campione & m (g) & V ($cm^3$) & $\rho \left(\frac{g}{cm^3}\right)$ \\
        \hline
        Parallelepipedo & 107.40 ± 0.01 & 12.87 ± 0.05 & 8.34 ± 0.03 \\
        Cilindro 1 & 41.21 ± 0.01 & 15.29 ± 0.06 & 2.695 ± 0.011 \\
        Sfera & 35.81 ± 0.01 & 4.597 ± 0.007 & 7.789 ± 0.014 \\
        Cilindro 2 & 8.00 ± 0.01 & 2.97 ± 0.01 & 2.695 ± 0.013 \\
        \hline
    \end{tabular}

    \begin{tabular}{ |c|c|c|c|c| }
        \hline
        Campione & $\rho \left(\frac{g}{cm^3}\right)$ & Materiale & $\rho_\text{lett.} \left(\frac{g}{cm^3}\right)$ & $\frac{\left|\rho_\text{best} - \left(\rho_{lett.}\right)_\text{best}\right|}{\delta \rho + \delta \rho_\text{lett.}}$ \\
        \hline
        Parallelepipedo & 8.34 ± 0.03 & Ottone giallo (high brass) & 8.47 ± 0.01 & 2.5 \\
        \hline
        Cilindro 1 & 2.695 ± 0.011 & Lega di Al laminato 3003 & 2.73 ± 0.01 & 1.7 \\
        \hline
        Sfera & 7.789 ± 0.014 & Acciaio & 7.8 ± 0.1 & 0.1 \\
        \hline
        Cilindro 2 & 2.695 ± 0.013 & Lega di Al laminato 3003 & 2.73 ± 0.01 & 1.5 \\
        \hline
    \end{tabular}
\end{center}

L'inconsistenza non trascurabile tra $\rho$ (le nostre misure) e $\rho_\text{lett.}$ (i rispettivi valori riportati in letteratura) è dovuta principalmente al fatto che si tratta di leghe; probabilmente, i nostri campioni presentavano concentrazioni diverse dei vari elementi.

\end{document}
