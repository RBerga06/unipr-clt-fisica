\documentclass{article}
\usepackage[utf8]{inputenc}
\usepackage[italian]{babel}
\title{
    Laboratorio di Fisica 1\\
    R1: Misura indiretta della densità di solidi
}
\author{Riccardo Bergamaschi, Elia Graiani, Simone Moglia}
\date{27/09/2023}
\makeindex
\begin{document}

\maketitle

\begin{abstract}
    Il gruppo di lavoro ha misurato la densità di solidi ignoti
    per individuarne la natura.
\end{abstract}

\section{Materiali utilizzati e strumenti di misura}
Abbiamo misurato la densità di quattro campioni solidi:
un parallelepipedo, una sfera e due cilindri.
Di seguito gli strumenti di misura utilizzati:

\begin{center}
    \begin{tabular}{ |c|c|c|c| }
        \hline
        Nome & Soglia & Portata & Sensibilità \\
        \hline
        Micrometro ad asta filettata & 0.01 mm & 25.00 mm & 0.01 mm \\
        \hline
    \end{tabular}
\end{center}

\end{document}
