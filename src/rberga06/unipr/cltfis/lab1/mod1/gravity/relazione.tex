\documentclass{article}
\usepackage[utf8]{inputenc}
\usepackage[italian]{babel}
\usepackage{amsmath}
\usepackage{amssymb}
\usepackage{siunitx}
\usepackage{tabularray}
\usepackage{graphicx}
\usepackage{float}
\newcommand*{\diam}{\varnothing}
\newcommand*{\best}[1]{{#1}_\text{best}}
\newcommand*{\bestp}[1]{{\left(#1\right)}_\text{best}}
\newcommand*{\pbest}[1]{\left({#1}_\text{best}\right)}
\newcommand*{\pbestp}[1]{\left({\left(#1\right)}_\text{best}\right)}
\newcommand*{\errrel}[1]{\frac{\delta #1}{{#1}_\text{best}}}
%% <custom footnotes/>
%\newcounter{savefootnote}
%\newcounter{symfootnote}
%\newcommand{\symfootnote}[1]{%
%   \setcounter{savefootnote}{\value{footnote}}%
%   \setcounter{footnote}{\value{symfootnote}}%
%   \ifnum\value{footnote}>8\setcounter{footnote}{0}\fi%
%   \let\oldthefootnote=\thefootnote%
%   \renewcommand{\thefootnote}{\fnsymbol{footnote}}%
%   \footnote{#1}%
%   \let\thefootnote=\oldthefootnote%
%   \setcounter{symfootnote}{\value{footnote}}%
%   \setcounter{footnote}{\value{savefootnote}}%
%}
%% </custom footnotes>
\title{
    Laboratorio di Fisica 1\\
    R3: Misura del modulo del campo gravitazionale locale
}
\author{Gruppo 17: Bergamaschi Riccardo, Graiani Elia, Moglia Simone}
\date{18/10/2023 – 25/10/2023}
\makeindex
\begin{document}

\maketitle

\begin{abstract}
    Il gruppo di lavoro ha misurato indirettamente il modulo del campo gravitazionale locale ($\vec{g}$) con due metodi distinti:
    dapprima facendo cadere due palline da ferme, successivamente tenendo conto di distanze diverse e velocità iniziali diverse.
\end{abstract}

\setcounter{section}{-1}  % Count sections starting from 0
\section{Materiali e strumenti di misura utilizzati}
\begin{center}
    \begin{tblr}{ |Q[l,m]|Q[c,m]|Q[c,m]|Q[c,m]| }
        \hline
        \textbf{Strumento di misura} & \textbf{\:\:\:\:\:Soglia\:\:\:\:\:} & \textbf{Portata} & \textbf{Sensibilità} \\
        \hline
        {Due fototraguardi con \\ contatore di impulsi} & \qty{1}{\micro s} & \qty{99999999}{\micro s} & \qty{1}{\micro s} \\
        \hline[dashed]
        Metro a nastro & \qty{0.1}{cm} & \qty{300.0}{cm} & \qty{0.1}{cm} \\
        \hline[dashed]
        Micrometro ad asta filettata & \qty{0.01}{mm} & \qty{25.00}{mm} & \qty{0.01}{mm} \\
        \hline
        \hline
        \textbf{Altro} & \SetCell[c=3]{l} \textbf{Descrizione/Note} \\
        \hline
        {Sistema di sgancio \\ elettropneumatico} & \SetCell[c=3]{l} {
            Usato, tramite comando manuale, per \\
            rilasciare le sferette in modo riproducibile.
        } \\
        \hline[dashed]
        Livella & \SetCell[c=3]{l} {
            Utile per assicurarsi che i fototraguardi \\
            siano orizzontali.
        } \\
        \hline[dashed]
        Due sferette & \SetCell[c=3]{l} {Indicheremo con $A,B$ le due sferette.} \\
        \hline
    \end{tblr}
\end{center}

\section{Caduta libera dallo stato di quiete}
\subsection{Esperienza e procedimento di misura}
\begin{enumerate}
    \item
        Misuriamo con il metro a nastro la distanza $d_0$ tra il centro
        del pistone del sistema di sgancio e un fototraguardo e con il
        micrometro ad asta filettata i diametri $\diam_A,\diam_B$ delle
        due sferette.\\
        Nel nostro caso, $d_0 = \left(125.8\pm 0.3\right)\unit{cm}$.\\
        \emph{
            \textbf{Nota.} Il gruppo di lavoro ha stimato
            $\delta d_0 = \qty{0.3}{cm}$ nonostante la sensibilità
            del metro sia notevolmente inferiore, a causa della difficoltà
            nel determinare il centro del pistone e la posizione precisa del
            fascio a infrarossi del fototraguardo.
        }
    \item
        Acceso e impostato adeguatamente il contatore di impulsi,
        misuriamo 100 volte il tempo di caduta di ogni sferetta
        $s\in\left\{A;B\right\}$
        fra il sistema di sgancio e il fototraguardo.
\end{enumerate}

\emph{
    \textbf{Notazione.} Indicheremo con $\left(t_s\right)_i$
    ogni $i$-esima misura del tempo di caduta
    $\left(i\in\left[0;100\right)\cap\mathbb{N}\right)$,
    mentre con $\overline{t_s}$ il tempo di caduta medio.
    In particolare:\[
        \delta\!\left(\overline{t_s}\right) = \sigma_{\overline{t_s}} =
        \frac{\sigma_{t_s}}{\sqrt{100}}   = \frac{\sigma_{t_s}}{10}.
    \]
}


\emph{
    \textbf{Osservazione.} Poiché il pistone tiene ferma la sferetta
    per la parte centrale, mentre il fototraguardo rileva il passaggio
    della pallina nel momento in cui il suo punto più basso interrompe il
    fascio di luce (infrarossa), la distanza percorsa dalla sferetta $s$
    nel tempo misurato non sarà semplicemente $d_0$, bensì
    $d_0 - \frac{1}{2}\diam_s$.
}

\subsection{Analisi dei dati raccolti e conclusioni}
Fissato un sistema di riferimento solidale all'apparato di misura, con origine
nella posizione di partenza delle sferette e $\hat{x} = \hat{g}$, possiamo
scrivere la seguente legge del moto:
\[x(t) = \frac{1}{2}g t^2\]
La norma di $\vec{g}$ misurata indirettamente è allora ricavabile da:
\begin{equation}\label{eq:1}
    g_s = \frac{2d_0 - \diam_i}{\left(\overline{t_s}\right)^2}
    % \best{g} = \frac{2\best{\left(d_0\right)} - \bestp{\diam_i}}
    %                 {\bestp{\overline{t_i}}^2}
    % \qquad\wedge\qquad
    % \errrel{g} = \frac{2\left(\delta d_0\right) - \delta \diam_i}
    %                   {2\best{\left(d_0\right)} - \bestp{\diam_i}}
    %            + 2\frac{\sigma_{\overline{t_i}}}{\overline{t_i}}
\end{equation}
dove l'errore su $g_s$ segue dalla propagazione degli errori su $d_0,\diam_s$ e
$\overline{t_s}$.

Di seguito riportiamo gli istogrammi dei tempi e i valori di $g$ così ottenuti.
Per confrontare queste misure indirette ($g_A$ e $g_B$) con il valore atteso, ovvero
\[g=\left(9.806\pm0.001\right)\unit{m\per s^2},\]valutiamo, per ogni
sferetta $s$, la seguente quantità:\[
    \varepsilon_s = \frac{ \bestp{g_s}-\best{g} }
                         { \delta g_s + \delta g }
\] Allora la misura $g_s$ da noi ottenuta è compatibile con $g$ se e solo se
$\left|\varepsilon_s\right|\le1$; inoltre, dal segno di $\varepsilon_s$ è possibile
sapere se $g_s$ misurata è una sovrastima ($\varepsilon_s>0$) o una sottostima
($\varepsilon_s<0$).

\begin{figure}[H]
    % trim={< v > ^}
    \includegraphics[trim={2cm .5cm 2.4cm 2.1cm},clip,width=.5\textwidth]{TempiA.jpg}
    \includegraphics[trim={2cm .5cm 2.4cm 2.1cm},clip,width=.5\textwidth]{TempiB.jpg}
    \caption{Istogrammi dei dati raccolti ($t_A$ e $t_B$)}
\end{figure}
\vspace{.5cm}
\begin{center}
    \begin{tblr}{ |Q[c,m]|Q[c,m]|Q[c,m]|Q[c,m]|Q[c,m]| }
        \hline
            $s$ &
            $\diam_s\:(\unit{mm})$ &
            $\overline{t_s}\:(\unit{ms})$ &
            $g_s\:(\unit{m\per s^2})$ &
            $\varepsilon_s$ \\
        \hline
        A & $24.62\pm0.01$ & $503.63\pm0.02$ & $9.82\pm0.02$ & $+0.68$ \\
        \hline[dashed]
        B & $22.23\pm0.01$ & $503.93\pm0.02$ & $9.82\pm0.02$ & $+0.59$ \\
        \hline
    \end{tblr}
\end{center}

Le misure di $g$ ottenute sono pertanto ampiamente consistenti con il valore atteso.

\emph{
    \textbf{Osservazione.} Dai valori di $\varepsilon$ non emerge una differenza
    significativa fra le due sferette. In particolare, sembra che l'attrito viscoso
    dell'aria abbia agito in maniera trascurabile (come ci aspettavamo).\\
    Tuttavia, dopo una più attenta analisi, è comunque possibile notare una tendenza:
    in media, la sferetta con raggio maggiore ha percorso la stessa distanza in un
    tempo leggermente minore.
    Pertanto, ciò potrebbe suggerire un effetto molto ridotto dell'attrito dell'aria;
    questa tendenza però non è rispecchiata dai valore di $\varepsilon$,
    probabilmente a causa di una sovrastima della distanza $d_0$. Si noti infatti che
    dal segno degli $\varepsilon$, entrambe le misure di $g$ sono risultate sovrastime,
    mentre, nell'equazione (\ref{eq:1}), $d_0$ è al numeratore.
}

\section{Caduta libera con velocità iniziale}
\begin{enumerate}
    \item Consideriamo la distanza tra i due fototraguardi e impostiamo i fotodiodi
          del contatore su A+B.
    \item Usando solo
    \begin{enumerate}
        \item Appeso il campione alla molla, allineiamo i due fototraguardi
              aiutandoci con la livella, in modo tale che possano rilevare
              le oscillazioni nel modo più accurato possibile;
        \item Tiriamo leggermente il campione verso il basso e poi lo rilasciamo,
              in modo che il sistema molla inizi a oscillare con direzione
              il più possibile parallela a $\vec{g}$;
        \item Attesa la stabilizzazione dell’oscillazione, avviamo
              l'acquisizione della misura di un tempo (20 periodi)
              $20T_i$.
        \item Ripetiamo molte volte (in tutto $N_{20T_i}$) i punti
              (b) e (c). In particolare, $N_{20T_A} = N_{20T_B} = 25$
              e $N_{20T_C} = N_{20T_{A+B}} = 30$.
    \end{enumerate}
    \item Infine, misuriamo con la bilancia, separatamente,
          la massa della molla $m_m$ e la massa del gancio $m_g$.
\end{enumerate}

Infatti, nel caso dinamico, il contributo di queste masse
\emph{non} si annulla; in particolare, la massa del gancio
contribuisce appieno (in quanto è solidale col grave),
mentre la massa della molla contribuisce per circa
$\frac{1}{3}$. La massa effettiva da considerare per ogni grave
sarà allora:
\[\best{\left(\left(m_\text{eff}\right)_i\right)} = \best{\left(m_i\right)} + \best{\left(m_g\right)} + \frac{1}{3}\best{\left(m_m\right)}\]
\[\delta \left(m_\text{eff}\right)_i = \delta m_i + \delta m_g + \frac{1}{3}\delta m_m\]

Di seguito sono riportate le distribuzioni dei dati raccolti:

\begin{figure}[H]
    %\includegraphics[trim={2cm 1.8cm .7cm 1.5cm},width=.5\textwidth]{Dinamico1.jpg}
    %\includegraphics[trim={.7cm 1.8cm 2cm 1.5cm},width=.5\textwidth]{Dinamico2.jpg}
    \caption{Istogrammi dei periodi delle oscillazioni di $A$ e $B$}
\end{figure}\begin{figure}[H]
    %\includegraphics[trim={2cm 1.8cm .7cm 1.5cm},width=.5\textwidth]{Dinamico3.jpg}
    %\includegraphics[trim={.7cm 1.8cm 2cm 1.5cm},width=.5\textwidth]{Dinamico4.jpg}
    \caption{Istogrammi dei periodi delle oscillazioni di $C$ e $A+B$}
\end{figure}

Poiché i nostri dati hanno assunto distribuzioni grossolanamente
approssimabili a gaussiane, possiamo procedere al calcolo di $k$,
utilizzando, per ogni grave $i$, i seguenti valori:
\[
    \left(20T_i\right)_\text{best} = \overline{20T_i}
    \qquad\wedge\qquad
    \delta\left(20T_i\right) =
    \sigma_{\overline{20T_i}} =
    \frac{\sigma_{20T_i}}{\sqrt{N_{20T_i}}}
\]
dove $\overline{20T_i}$ e $\sigma_{20T_i}$ indicano rispettivamente
media e deviazione standard dei tempi.

Per determinare la costante elastica della molla, abbiamo effettuato
una regressione lineare (stavolta pesata) sui quadrati dei valori medi
dei tempi ($T_i^2$, con
$\delta T_i^2 = 5 \cdot 10^{-3} (20 T_i)_\text{best} \delta(20 T_i)$
)\footnote[2]{
    La formula per l'errore su $T_i^2$ segue direttamente dalla
    propagazione degli errori:
    \[
        \frac{\delta T_i^2}{\left(T_i^2\right)_\text{best}} = 2\frac{\delta T_i}{{\left(T_i\right)}_\text{best}}
        \qquad
        \delta T_i^2 = 2\left(T_i\right)_\text{best}\delta T_i
        \qquad
        \delta T_i^2 = \frac{\left(20T_i\right)_\text{best}(\delta 20T_i)}{200}
    \]
    da cui quanto riportato sopra.
    Si osservi che $\delta T_i^2$ dipende da
    $\left(20T_i\right)_\text{best}$:
    proprio questo è il motivo dietro alla scelta del metodo pesato
    per la regressione lineare.
} rispetto alla massa $\left(m_\text{eff}\right)_i$, facendo riferimento
alla relazione $T_i^2 = \frac{4\pi^2}{k} \left(m_\text{eff}\right)_i$. Allora, detto $b$ il
coefficiente angolare della retta di regressione, varrà:
\[
    k_\text{best}=\frac{4\pi^2}{b_\text{best}}
    \qquad\wedge\qquad
    \frac{\delta k}{k_\text{best}}=\frac{\delta b}{b_\text{best}}
\]
Si noti che, anche in questo caso, l'intercetta $a$ della retta dev'essere
compatibile con $0$.

Di seguito è riportata la retta di regressione, assieme ai risultati ottenuti:

\begin{figure}[H]
    %\includegraphics[trim={0 1.8cm 0 0},width=\textwidth]{DinamicoReg.jpg}
    \caption{
        La retta di regressione (in rosso)
        e la sua regione di incertezza (in rosa).
    }
\end{figure}

\begin{itemize}
    \item $a = \left(0.02\pm0.19\right)\unit{cm}$ (compatibile con 0)
    \item $
        b = \left(4.604\pm0.002\right)\cdot10^{-4}\;\unit{s^2\per g}
          = \left(46.04\pm0.02\right)\cdot10^{-2}\;\unit{s^2\per kg}
    $
    \item $k = \left(85.74\pm0.04\right)\unit{N\per m}$
\end{itemize}


\section{Conclusioni}
Per valutare numericamente la consistenza tra i due valori di $k$ ottenuti,
abbiamo calcolato il seguente valore (numero puro):
\[
    \varepsilon =
    \frac{
        \left|\left(k_\text{statica}\right)_\text{best} - \left(k_\text{dinamica}\right)_\text{best}\right|
    }{
        \delta k_\text{statica} + \delta k_\text{dinamica}
    }
\]
Allora $k_\text{statica}$ e $k_\text{dinamica}$ sono consistenti se e solo se $\varepsilon \le 1$.

Nel nostro caso, $\varepsilon = 1.33$. Il gruppo di lavoro ha ipotizzato che
questa inconsistenza (comunque contenuta, seppur non trascurabile) fra le due
misure possa essere ragionevolmente giustificata dalla difficoltà incontrata
nel ridurre al minimo le oscillazioni in direzione perpendicolare a $\vec{g}$;
considerato inoltre che la posizione dei fototraguardi non era ottimale, ciò
potrebbe avere ulteriormente influenzato la distribuzione dei tempi. È in
effetti possibile osservare che le distribuzioni da noi ottenute non sono,
il più delle volte, del tutto simmetriche: la moda sembra essersi spostata
leggermente a sinistra – un possibile sintomo dell'influenza di un
errore sistematico sulle misure.

\end{document}
