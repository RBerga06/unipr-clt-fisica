\documentclass{article}
\usepackage[utf8]{inputenc}
\usepackage[italian]{babel}
\usepackage{amsmath}
\usepackage{amsfonts}
\usepackage{amssymb}
\newcommand*{\qed}{\blacksquare}
\newcommand*{\M}[3]{\mathcal{M}_{#1\times#2} \left(#3\right)}
\newcommand*{\MR}[2]{\M{#1}{#2}{\mathbb{R}}}
\newcommand*{\MC}[2]{\M{#1}{#2}{\mathbb{C}}}
\newcommand*{\MK}[2]{\M{#1}{#2}{\mathbb{K}}}
\newcommand*{\T}[1]{{#1}^\text{T}}  % Trasposta di una matrice
\newcommand*{\m}[1]{\begin{bmatrix}#1\end{bmatrix}}
\DeclareMathOperator{\Tr}{Tr}  % Traccia di una matrice
\title{Geometria - Esercizi 4}
\author{Riccardo Bergamaschi}
\date{19/10/2023}
\begin{document}
\maketitle
\section*{Esercizio 1}
\begin{enumerate}
    \item $AB=\m{
        4 & 7 & -1 & 11 \\
        1 & 3 & -2 & 0 \\
    }$, $BA$ non definito
    \item $AB=\m{
        0 & 0 \\
        0 & 0 \\
    }$, $BA=\m{
        26 & -52 \\
        13 & -26 \\
    }$
    \item $AB=\m{3}$, $BA=\m{
        15 & -3 & 9 & 33 & -6 \\
        -10 & 2 & -6 & -22 & 4 \\
        -20 & 4 & -12 & -44 & 8 \\
        0 & 0 & 0 & 0 & 0 \\
        5 & -1 & 3 & 11 & -2 \\
    }$
    \item $AB=\m{
        11 & 2 & 12 \\
        8 & -2 & 6 \\
        -8 & 5 & -3 \\
    }$, $BA=\m{
        -2 & 8 & -2 \\
        0 & 3 & 0 \\
        1 & 17 & 5 \\
    }$
    \item $AB=\m{
        -2i+4 & 1+5i & 6-4i \\
        4i & i-2 & 1+2i \\
        4-6i & 1+4i & 4-i \\
    }$, $BA=\m{
        i-1 & 6-i & -1+3i \\
        0 & 4-4i & 2-4i \\
        2i & 5-6i & 1-i \\
    }$
\end{enumerate}

\section*{Esercizio 3}
\begin{enumerate}
    \item Poiché $A$ e $B$ sono ortogonali,
    $\left(A^\text{T}\right)^{-1}=A$ e $\left(B^\text{T}\right)^{-1}=B$.

    Allora $
        \left(\left(AB\right)^\text{T}\right)^{-1} =
        \left(A^\text{T}B^\text{T}\right)^{-1} =
        \left(A^\text{T}\right)^{-1}\left(B^\text{T}\right)^{-1} =
        AB
    $, e quindi $AB$ è anch'essa ortogonale. $\qed$
    \item $A$ è invertibile se e solo se esiste $B\in\MR{n}{n}$
    inversa di A, ovvero tale che $AB=BA=\text{Id}_n$.
    Essendo A ortogonale, per definizione $A^\text{T}$ è tale per cui
    $A A^\text{T} = A^\text{T} A = \text{Id}_n$. Di conseguenza,
    $A^\text{T}$ è l'inversa di $A$, ovvero $A^\text{T} = A^{-1}$.
    In particolare, $A$ è invertibile. $\qed$
    \item $A^{-1}=A^\text{T}$ è ortogonale, poiché
    $\left(\left(A^\text{T}\right)^\text{T}\right)^{-1}=A^{-1}=A^\text{T}$.
    $\qed$
\end{enumerate}

\section*{Esercizio 4}
\begin{enumerate}
    \item Poiché $A$ e $B$ sono unitarie,
    $\left(A^*\right)^{-1}=A$ e $\left(B^*\right)^{-1}=B$.

    Allora $
        \left(\left(AB\right)^*\right)^{-1} =
        \left(A^*B^*\right)^{-1} =
        \left(A^*\right)^{-1}\left(B^*\right)^{-1} =
        AB
    $, e quindi $AB$ è anch'essa unitaria. $\qed$
    \item $A$ è invertibile se e solo se esiste $B\in\MC{n}{n}$
    inversa di A, ovvero tale che $AB=BA=\text{Id}_n$.
    Essendo A unitaria, per definizione $A^*$ è tale per cui
    $A A^* = A^* A = \text{Id}_n$. Di conseguenza,
    $A^*$ è l'inversa di $A$, ovvero $A^* = A^{-1}$.
    In particolare, $A$ è invertibile. $\qed$
    \item $A^{-1}=A^*$ è ortogonale, poiché
    $\left(\left(A^*\right)^*\right)^{-1}=A^{-1}=A^*$.
    $\qed$
\end{enumerate}

\section*{Esercizio 5}
\begin{enumerate}
    \item $AA^\text{T}=
        \m{ 1 & 0 \\ -1 & 1 }
        \m{ 1 & 0 \\ -1 & 1 }^\text{T} =
        \m{ 1 & 0 \\ -1 & 1 }
        \m{ 1 & -1 \\ 0 & 1 } =
        \m{ 1 & -1 \\ -1 & 2 } \ne \text{Id}_2
    $, quindi $A$ non è ortogonale.
    \item $BB^\text{T}=
        \m{ 0 & 1 \\ -1 & 0 }
        \m{ 0 & 1 \\ -1 & 0 }^\text{T} =
        \m{ 0 & 1 \\ -1 & 0 }
        \m{ 0 & -1 \\ 1 & 0 } =
        \m{ 1 & 0 \\ 0 & 1 } = \text{Id}_2
    $, quindi $B$ è ortogonale.
    \item \begin{equation*}\begin{aligned}
        C\,C^\text{T} &=
        \m{
            \frac{1}{\sqrt{2}} & -\frac{1}{\sqrt{2}} & 0 \\
            0 & 0 & 1 \\
            \frac{1}{\sqrt{2}} & \frac{1}{\sqrt{2}} & 0 \\
        }
        \m{
            \frac{1}{\sqrt{2}} & -\frac{1}{\sqrt{2}} & 0 \\
            0 & 0 & 1 \\
            \frac{1}{\sqrt{2}} & \frac{1}{\sqrt{2}} & 0 \\
        }^\text{T} \\ &=
        \m{
            \frac{1}{\sqrt{2}} & -\frac{1}{\sqrt{2}} & 0 \\
            0 & 0 & 1 \\
            \frac{1}{\sqrt{2}} & \frac{1}{\sqrt{2}} & 0 \\
        }
        \m{
            \frac{1}{\sqrt{2}} & 0 & \frac{1}{\sqrt{2}} \\
            -\frac{1}{\sqrt{2}} & 0 & \frac{1}{\sqrt{2}} \\
            0 & 1 & 0 \\
        } \\ &=
        \m{
            1 & 0 & 0 \\
            0 & 1 & 0 \\
            0 & 0 & 1 \\
        } = \text{Id}_3
    \end{aligned}\end{equation*}
    Quindi $C$ è ortogonale.
    \item \begin{equation*}\begin{aligned}
        DD^\text{T} &=
        \m{
            0 & -1 & 0 & 0 \\
            1 & 0 & 0 & 0 \\
            0 & 0 & 1 & 0 \\
            0 & 0 & 0 & -1 \\
        }
        \m{
            0 & -1 & 0 & 0 \\
            1 & 0 & 0 & 0 \\
            0 & 0 & 1 & 0 \\
            0 & 0 & 0 & -1 \\
        }^\text{T} \\ &=
        \m{
            0 & -1 & 0 & 0 \\
            1 & 0 & 0 & 0 \\
            0 & 0 & 1 & 0 \\
            0 & 0 & 0 & -1 \\
        }
        \m{
            0 & 1 & 0 & 0 \\
            -1 & 0 & 0 & 0 \\
            0 & 0 & 1 & 0 \\
            0 & 0 & 0 & -1 \\
        } \\ &=
        \m{
            1 & 0 & 0 & 0 \\
            0 & 1 & 0 & 0 \\
            0 & 0 & 1 & 0 \\
            0 & 0 & 0 & 1 \\
        } = \text{Id}_4
    \end{aligned}\end{equation*}
    Quindi $D$ è ortogonale.
\end{enumerate}

\section*{Esercizio 6}
$W$ è un sottospazio vettoriale di $\MK{n}{q}$ su $\mathbb{K}$ solo se
è chiuso rispetto alla somma e alla moltiplicazione per scalare.
\begin{itemize}
    \item $W$ è chiuso rispetto alla somma. Siano $A_1,A_2\in W$; allora:
    \[BA_1=0_{\MK{m}{q}} \wedge BA_2=0_{\MK{m}{q}}\]
    \[B(A_1 + A_2) = BA_1 + BA_2 = 0_{\MK{m}{q}} + 0_{\MK{m}{q}} = 0_{\MK{m}{q}}\]
    Quindi $A_1 + A_2 \in W$.
    \item $W$ è chiuso rispetto alla moltiplicazione per scalare.
    Siano $A\in W$ e $\lambda\in\mathbb{K}$; allora $BA=0_{\MK{m}{q}}$ e:
    \[B(\lambda A)=\lambda BA = \lambda 0_{\MK{m}{q}} = 0_{\MK{m}{q}}\]
    Quindi $\lambda A \in W$.
\end{itemize}
Da queste due considerazioni, segue, per definizione, la tesi. $\qed$

\section*{Esercizio 7}
\[\begin{aligned}
    \Tr(B^{-1} A B) &=
    \Tr(B^{-1})\Tr(A)\Tr(B) =
    \Tr(B^{-1})\Tr(B)\Tr(A) \\ &=
    \Tr(B^{-1} B)\Tr(A) =
    \Tr(\text{Id}_n)\Tr(A) =
    \Tr(A)\hspace{10pt}\qed
\end{aligned}\]

\section*{Esercizio 8}
Essendo $A$ e $B$ ortogonali, valgono:
\[
    \left(A^\text{T}\right)^{-1}=A
    \hspace{10pt}\wedge\hspace{10pt}
    \left(B^\text{T}\right)^{-1}=B
    \hspace{10pt}\wedge\hspace{10pt}
    B^{-1}=B^\text{T}
\]
Allora:
\[\begin{aligned}
    \left(\T{\left(BA\T{B}\right)}\right)^{-1} &=
    \left(\T{\left(\T{B}\right)}\T{A}\T{B}\right)^{-1} =
    \left(B\T{A}\T{B}\right)^{-1} \\ &=
    \left(\T{B}\right)^{-1}\left(\T{A}\right)^{-1}B^{-1} =
    BA\T{B}
\end{aligned}\]
Quindi anche $BA\T{B}$ è ortogonale. $\qed$

\section*{Esercizio 9}
\[\begin{aligned}
    \det(\lambda A) &=
    \det\m{
        \lambda A^1,\lambda A^2,\dots,\lambda A^n
    } \\ &=
    \lambda \det\m{
        A^1,\lambda A^2,\dots,\lambda A^n
    } \\ &=
    \lambda^2 \det\m{
        A^1, A^2,\dots,\lambda A^n
    } \\ &= \dots \\ &=
    \lambda^n\det\m{
        A^1,A^2,\dots,A^n
    } \\ &=
    \lambda^n\det(A) \\
    &&\qed
\end{aligned}\]

\section*{Esercizio 10}
$A$ è antisimmetrica se e solo se $\T{A}=-A$. Allora:
\[
    \det(A) =
    \det\left(\T{A}\right) =
    \det(-A) =
    \det\left((-1)A\right) =
    (-1)^n \det(A)
\]
Essendo $n$ dispari, $(-1)^n = -1$. Otteniamo dunque l'equazione:
\[\begin{aligned}
    \det(A) &= -\det(A) \\
    2 \det(A) &= 0 \\
    \det(A) &= 0 \\
\end{aligned}\]
Da cui la tesi. $\qed$

\section*{Esercizio 11}
\begin{enumerate}
    \item $
        \det\m{2 & -3 \\ 1 & 4} =
        2 \cdot 4 - (-3) \cdot 1 = 8 + 3 = 11
    $
    \item $
        \det\m{
            3 & -1 & 0 \\
            0 & -4 & 1 \\
            2 & 1 & 1 \\
        } = \det\m{
            3 & -1 & 0 \\
            0 & -4 & 1 \\
            2 & 5 & 0 \\
        } = -1 \cdot \det\m{
            3 & -1 \\
            2 &  5 \\
        } = -17
    $
    \item \[\begin{aligned}
        \det\m{
            1 & 1 & 0 & 0 \\
            -1 & 2 & 0 & 0 \\
            0 & 0 & 1 & -7 \\
            0 & 0 & 0 & 3 \\
        } &= \det\m{
            1 & 1 & 0 & 0 \\
            0 & 3 & 0 & 0 \\
            0 & 0 & 1 & -7 \\
            0 & 0 & 0 & 3 \\
        } \\ &= \det\m{
            3 & 0 & 0 \\
            0 & 1 & -7 \\
            0 & 0 & 3 \\
        } = 3 \det\m{
            1 & -7 \\
            0 & 3 \\
        } = 9
    \end{aligned}\]
\end{enumerate}

\section*{Esercizio 12}
Una matrice quadrata è singolare se e solo se ha determinante $0$.
\[\begin{aligned}
    \det\m{
        2a & 1 & 3 \\
        -a & 5 & 1 \\
        2 & 6 & 4 \\
    } &= \det\m{
        5a & -14 & 0 \\
        -a & 5 & 1 \\
        4a+2 & -14 & 0 \\
    } = -\det\m{
        5a & -14 \\
        4a+2 & -14 \\
    } \\ &= 14(5a) - 14(4a+2) = 14(a-2)
\end{aligned}\]
Allora, la matrice non è singolare se e solo se $-14(a-2) \ne 0$, ovvero $a \ne 2$.

\section*{Esercizio 13}
$0_{\MR{2}{2}}$ è una matrice diagonale ma non invertibile; infatti:
\[\forall A\in\MR{2}{2}\hspace{15pt}0_{\MR{2}{2}}A=0_{\MR{2}{2}}\ne\text{Id}_2\]

\section*{Esercizio 14}
$\m{ 0 & 1 \\ -1 & 0 }$ è antisimmetrica e invertibile; infatti:
\[
    \det\m{ 0 & 1 \\ -1 & 0 } =
    0\cdot0 - 1\cdot(-1) = 1 \ne 0
\]

\section*{Esercizio 15}
In quanto diagonale, $0_{\MR{2}{2}}$ è anche simmetrica.
Inoltre, abbiamo già dimostrato che essa non è invertibile.

\section*{Esercizio 16}
Identico all'esercizio 13

\section*{Esercizio 17}
Identico all'esercizio 15

\section*{Esercizio 18}
Ogni matrice reale simmetrica è anche, vista come matrice complessa, Hermitiana.
Infatti, $\MR{n}{n}\subset\MC{n}{n}$ e,
fissata $A\in\MR{n}{n}$ simmetrica, vale
$A^* = \overline{\T{A}} = \overline{A} = A$.
Di conseguenza, $\m{ 0 & 1 \\ 1 & 0 }$ è
Hermitiana (in quanto simmetrica) e invertibile
(poiché $\det\m{ 0 & 1 \\ 1 & 0 } = -1$).

\end{document}
