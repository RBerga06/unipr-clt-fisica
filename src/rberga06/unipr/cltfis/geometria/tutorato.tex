\documentclass{article}
\usepackage[utf8]{inputenc}
\usepackage[italian]{babel}
\usepackage{amsmath}
\usepackage{amsfonts}
\usepackage{amssymb}
\usepackage{xfrac}
\makeatletter
\renewcommand*\env@matrix[1][*\c@MaxMatrixCols c]{%
  \hskip -\arraycolsep
  \let\@ifnextchar\new@ifnextchar
  \array{#1}}
\makeatother
\newcommand*{\qed}{\blacksquare}
\newcommand*{\M}[3]{\mathcal{M}_{#1\times#2} \left(#3\right)}
\newcommand*{\MR}[2]{\M{#1}{#2}{\mathbb{R}}}
\newcommand*{\MC}[2]{\M{#1}{#2}{\mathbb{C}}}
\newcommand*{\MK}[2]{\M{#1}{#2}{\mathbb{K}}}
\newcommand*{\T}[1]{{#1}^\text{T}}  % Trasposta di una matrice
\newcommand*{\m}[1]{\begin{bmatrix}#1\end{bmatrix}}
\newcommand*{\sys}[1]{\left\{\begin{array}{@{}l@{}}#1\end{array}\right.}
\DeclareMathOperator{\Tr}{Tr}  % Traccia di una matrice
\DeclareMathOperator{\rg}{rg}  % Rango di una matrice
\title{Esercizi tutorato in preparazione del primo parziale}
\author{Riccardo Bergamaschi}
\date{\today}
\begin{document}
\maketitle
\section{}
\subsection{}
\[\pi:\quad x+3y-z=4\qquad P=\begin{bmatrix} 1 \\ 0 \\ 3 \end{bmatrix}\]
\[r:\quad \begin{bmatrix}x\\y\\z\end{bmatrix}=\begin{bmatrix}1\\0\\3\end{bmatrix}+t\begin{bmatrix}1\\3\\-1\end{bmatrix}\]
\[\pi_1 \perp \pi\quad\wedge\quad\pi_1\parallel s_1\]
\[
    s_1:\left\{\begin{array}{@{}l@{}}
        x + 2y = 2 \\
        y - z = 2 \\
    \end{array}\right.\qquad\qquad P=\begin{bmatrix} 1 \\ 0 \\ 1
    \end{bmatrix}
\]
\[n_{\pi_1} = v_1\times n_{\pi}\]
\[
    v_1 =
    \begin{bmatrix}1\\2\\0\end{bmatrix} \times
    \begin{bmatrix}0\\1\\-1\end{bmatrix} =
    \begin{bmatrix}-2\\1\\1\end{bmatrix}
\]
\[
    n_{\pi_1} =
    \begin{bmatrix}-2\\1\\1\end{bmatrix} \times
    \begin{bmatrix}1\\3\\-1\end{bmatrix} =
    \begin{bmatrix}-4\\-1\\-7\end{bmatrix}
\]
\[\pi_1:\quad4x+y+7z=11\]

Punto c: $s_2$ incidente a $\pi:x+3y-z=4$, passante per $P=\begin{bmatrix}1\\2\\1\end{bmatrix}$.
\[
    r:\m{x\\y\\z} =\m{1\\2\\1}+tV
    \quad\text{tale che}\quad
    \left\langle V,n\right\rangle\ne 0
\]

\section{}
\subsection{}
\[r:\sys{x = 0 \\ y + z = 5} \qquad P = \m{1\\-2\\3}\]
\[\pi:4x+y+z=5\]
\[r_1:\sys{x-y=0\\x+y-z=2}\]

Mutua posizione fra $r$ ed $r_1$:
\[
    \m{
        [ccc|c]
        1&0&0&0\\
        0&1&1&5\\
        1&-1&0&0\\
        1&1&-1&2\\
    }\leadsto\m{
        [ccc|c]
        1&0&0&0\\
        1&1&-1&2\\
        1&-1&0&0\\
        0&1&1&5\\
    }\leadsto\m{
        [ccc|c]
        1&0&0&0\\
        1&1&-1&2\\
        1&-1&0&0\\
        0&1&1&5\\
    }
\]

\section{}
\subsection{}
\[\m{
        \alpha^2-4&\alpha^2+1&\alpha+1\\
        -5&0&-\alpha+5\\
        -\alpha-1&-3&-\alpha^2+4\\
    }\]\[
    \sys{
        \alpha^2-4=0\\
        \alpha^2+1=-(-5)\\
        \alpha+1=-(-\alpha-1)\\
        -\alpha+5=-(-3)
    }\]\[
    \sys{
        \alpha=\pm2\\
        \alpha=\pm2\\
        \alpha=2\\
    }
\]
\[\alpha=2\]
\subsection{}
\[
    A=\m{1&0&1\\1&1&1}\in\MR{2}{3}\qquad
    B=\m{-\alpha&2\\\alpha&-\alpha\\\alpha^2&-\alpha}\in\MR{3}{2}
\]\[
    \Tr(AB)=\Tr{\m{-\alpha+\alpha^2&2-\alpha\\
    \alpha^2&2-2\alpha}}=\alpha^2-3\alpha+2
\]
\[\alpha^2-3\alpha+2=0\qquad \alpha\in{1;2}\]
\[\Tr(AB)=0\quad\Leftrightarrow\quad\alpha\in{1;2}\]

\subsection{}
\[D=\m{i+2&\beta&i\\0&1&\beta\\i+2&i-2&i}\]
\[\begin{aligned}
    \det{D} &=
    \det\m{i+2&\beta&i\\0&1&\beta\\0&i-2-\beta&0} \\&=
    (i+2)\det\m{1&\beta\\i-2-\beta&0} \\&=
    (i+2)\beta(\beta+2-i)
\end{aligned}\]
\[
    \det{D}\ne0\quad\Leftrightarrow\quad
    i\notin\left\{ 0,-2+i \right\}
\]

\section{}
\subsection{}
Sia $\beta\in\mathbb{R}$. Determinare se il sistema è
compatibile:
\[\begin{aligned}\m{[ccc|c]
    1&1&0&\beta+1\\
    1&0&1&2\beta\\
    2&1&1&\beta-3\\
}&\leadsto\m{[ccc|c]
    1&1&0&\beta+1\\
    0&-1&1&\beta-1\\
    0&-1&1&-\beta-5\\
}\\&\leadsto\m{[ccc|c]
    1&1&0&\beta+1\\
    0&-1&1&\beta-1\\
    0&0&0&-2\beta-4\\
}
\end{aligned}\]
Il sistema è compatibile se e solo se $-2\beta-4=0$,
ovvero $\beta=-2$.

\subsection{}
Determinare il rango della matrice al variare di $\alpha\in\mathbb{R}$:

\[\begin{aligned}A=\m{
    1&-1&1&\alpha^2-\alpha-1\\
    0&1&\alpha&\alpha^2+2\alpha+1\\
    1&2&1&\alpha^2+2\alpha+2\\
    -1&1&\alpha-1&\alpha^2+3\alpha+1\\
}&\leadsto\m{
    1&-1&1&\alpha^2-\alpha-1\\
    0&1&\alpha&\alpha^2+2\alpha+1\\
    0&3&0&3\alpha+3\\
    0&0&\alpha&2\alpha^2+2\alpha\\
}\\&\leadsto\m{
    1&-1&1&\alpha^2-\alpha-1\\
    0&1&\alpha&\alpha^2+2\alpha+1\\
    0&0&-3\alpha&-3\alpha^2-3\alpha\\
    0&0&\alpha&2\alpha^2+2\alpha\\
}\\&\leadsto\m{
    1&-1&1&\alpha^2-\alpha-1\\
    0&1&\alpha&\alpha^2+2\alpha+1\\
    0&0&\alpha&\alpha^2+\alpha\\
    0&0&0&\alpha^2+\alpha\\
}
\end{aligned}\]
\[\rg{A}=\begin{cases}
    2&\alpha=0\\
    3&\alpha=-1\\
    4&\text{altrimenti}\\
\end{cases}\]

\section{Altri esercizi}
\subsection{}
Determinare la retta $s_2$ tale che:
\[s_2\parallel\pi_4\wedge P\in s_2\]

con:
\[
    s_2\parallel\pi_4\qquad
    \pi_4:x-y=3\qquad
    r:\sys{y=0\\y+z=4}
    \qquad P=\m{2\\1\\2}
\]

Allora, la soluzione sarà:
\[s_2:\sys{2y+z=4\\x-y=1}\]

\subsection{}
$17.3$

\[P_1=\m{1\\0\\1}\quad P_2=\m{1\\1\\1}\quad r:\sys{3x-2z=3\\3y-z=1}\]

\[V_{12}=P_1-P_2=\m{0\\-1\\0}\]
\[n=V\times V_{12}=\m{2\\1\\3}\times\m{0\\-1\\0}=\m{3\\0\\-2}\]

\[\pi: 3x-2z=1\]

\section{}
\[\begin{aligned}B=\m{
    0&1&1&1\\
    1&2&t+1&2\\
    0&3&t&3+t\\
    1&1&t&1+t\\
    1&0&1&1\\
}&\leadsto\m{
    1&2&t+1&2\\
    0&1&t1&1\\
    0&3&t&3+t\\
    0&-1&-1&t-1\\
    0&-2&-t&-1\\
}\end{aligned}\]


\end{document}