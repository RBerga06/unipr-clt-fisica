\documentclass{article}
\usepackage[utf8]{inputenc}
\usepackage[italian]{babel}
\usepackage{amsmath}
\usepackage{amsfonts}
\usepackage{amssymb}
\usepackage{xfrac}
\newcommand*{\qed}{\blacksquare}
\newcommand*{\oo}{\infty}
\newcommand*{\limit}[2]{\lim_{#1\rightarrow#2}}
\newcommand*{\sys}[1]{\left\{\begin{aligned}#1\end{aligned}\right.}
\begin{document}

\textbf{1. Data la funzione:}
\[
    f(x) = \begin{cases}
        x^2       & x \le 0 \\
        \sqrt{2x} & x > 0   \\
    \end{cases}
\]

\textbf{a. Determinare l'immagine attraverso $f$ di $S=[-2;1]$.}
L'immagine di $S$ attraverso $f$ è l'insieme di tutti i valori che $f$
assume in corrispondenza dei punti di $S$:
\[\begin{aligned}
    f(\left[-2;1\right])
    &= \left\{y\in\mathbb{R}\:|\:\exists x\in\left[-2;1\right]:y=f(x)\right\}
\end{aligned}\]
Risolviamo allora l'equazione $y = f(x)$, tenendo $y$ come parametro.
I valori cercati di $y$ saranno allora tutti e soli quelli per i quali questa
equazione ammette almeno una soluzione nell'intervallo specificato.
\[y = f(x)\]
\[\begin{aligned}
    \:\\
        \sys{&x\le0\\&y = x^2}
        \quad&\vee\quad
        \sys{&x > 0 \\ &y = \sqrt{2x}}
    \\\:\\
        \sys{&y\ge0\\&x\le 0\\&x=\pm\sqrt{y}}
        \quad&\vee\quad
        \sys{&y\ge0\\&x>0\\&y^2=2x}
    \\\:\\
        \sys{&y\ge0\\&x=-\sqrt{y}}
        \quad&\vee\quad
        \sys{&y\ge0\\&x=\frac{y^2}{2}}
    \\\:\\
        \sys{&y\ge0\\&x=-\sqrt{y}}
        \quad&\vee\quad
        \sys{&y\ge0\\&x=\frac{y^2}{2}}
\end{aligned}\]
Se $y<0$, sicuramente non ci sono soluzioni.
Altrimenti, le soluzioni $x$ esistono, ma noi dobbiamo imporre che almeno una si
trovi nell'intervallo $\left[-2;1\right]$. Le condizioni aggiuntive su $y$
saranno allora:
\[\begin{aligned}
    \sys{&x\in\left[-2;1\right]\\&x=-\sqrt{y}}
    \quad&\vee\quad
    \sys{&x\in\left[-2;1\right]\\&x=\frac{y^2}{2}}
    \\\:\\
    -2\le-\sqrt{y}\le1
    \quad&\vee\quad
    -2\le\frac{y^2}{2}\le1
    \\\:\\
    2\ge\sqrt{y}\ge-1
    \quad&\vee\quad
    -4\le y^2\le2
    \\\:\\
    2\ge\sqrt{y}
    \quad&\vee\quad
    y^2\le2
    \\\:\\
    y\le4
    \quad&\vee\quad
    -\sqrt{2}\le y\le\sqrt{2}
    \\\:\\
    y&\le 4
\end{aligned}\]
A questo punto ci resta solo da intersecare questi valori di $y$ con $y\ge0$
(la condizione trovata in precedenza): otteniamo allora $y\in\left[0;4\right]$.
L'immagine di $S$ tramite $f$ sarà allora:

\[\begin{aligned}
    f(\left[-2;1\right])
    = \left\{y\in\mathbb{R}\:|\:\exists x\in\left[-2;1\right]:y=f(x)\right\}
    = \left[0; 4\right]\qquad\qed
\end{aligned}\]

\end{document}